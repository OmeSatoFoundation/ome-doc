\begin{frame}
    \frametitle{今日の目標} 
    \begin{itemize}
        \item 1時間目\\
        {\footnotesize「ファイル」と「ディレクトリ」の基本を知ろう:コンピュータでのデータ管理\\
        「ターミナル」で「コマンド」を使ってみよう:テキストでコンピュータに命令\\}
        \item 2時間目\\
        {\footnotesize「ターミナル」で「コマンド」を使ってみよう:テキストでコンピュータに命令\\}
        \item 3時間目\\
        {\footnotesize「センサーボード」の基本について知ろう:コンピュータと周りの世界のやりとり\\
        「センサーボード」を使ってスクリプトを書いてみよう:コンピュータに自動で仕事をさせる\\}
        \item 4時間目\\
        {\footnotesize「センサーボード」を使ってスクリプトを書いてみよう(応用)自分で0からスクリプトを書いてみよう:新しいプログラムをひとりで作ろう\\}
    \end{itemize}
\end{frame}

\begin{frame}
    \frametitle{ファイルって何だろう}
    \begin{itemize}
        \item ファイルとは
        \begin{itemize}
            \item コンピュータに保存されるデータ
            \item 音楽、画像、文章などなど
        \end{itemize}
        \item 拡張子
        \begin{itemize}
            \item データ(ファイル)の種類を決めるもの
            \item データの名前.拡張子 で名前をつけるよ
        \end{itemize}
    \end{itemize}
    
    \begin{figure}[h]
    \centering
    \begin{minipage}[b]{0.32\columnwidth}
        \centering
        \includesvg[width=0.5\columnwidth]{images/chap03/image.svg}
        \caption{gazou.jpg}
    \end{minipage}
    \begin{minipage}[b]{0.32\columnwidth}
        \centering
        \includesvg[width=0.5\columnwidth]{images/chap03/oto.svg}
        \caption{oto.mp3}
    \end{minipage}
    \begin{minipage}[b]{0.32\columnwidth}
        \centering
        \includesvg[width=0.5\columnwidth]{images/chap03/douga.svg}
        \caption{douga.mp4}
    \end{minipage}
    \end{figure}
\end{frame}

\begin{frame}
    \frametitle{ディレクトリって何だろう?}
    \begin{itemize}
        \item ファイルを整理整頓するためのもの
        \item ディレクトリの中にファイルやディレクトリをしまうよ
        \item 1回目で習ったフォルダと同じものだよ
    \end{itemize}
\end{frame}

\begin{frame}
    \frametitle{ディレクトリの関係(1)}
    \begin{columns}
        \column{0.4\textwidth}
            \centering
            \includesvg[width=0.8\textwidth]{images/chap03/root_directory.svg}
        \column{0.6\textwidth}
            \begin{itemize}
                \item /は一番上のディレクトリをルートディレクトリと呼ぶよ
                \item コンピュータの設定を変える大事なファイルもあるので、間違って消さないように気を付けよう!
            \end{itemize}
    \end{columns}
\end{frame}

\begin{frame}
    \frametitle{ディレクトリの関係(2)}
    \begin{columns}
        \column{0.4\textwidth}
            \centering
            \includesvg[width=0.8\textwidth]{images/chap03/home_directory.svg}
        \column{0.6\textwidth}
            \begin{itemize}
                \item 自分がいま使っているディレクトリはカレントディレクトリ
                \item みんなはホームディレクトリ(/home/pi)で作業するよ
                \item ディレクトリの位置は一番上からどこにあるかまたはカレントディレクトリからどこにあるかで決まるよ
            \end{itemize}
    \end{columns}
\end{frame}

\begin{frame}
    \frametitle{ディレクトリの関係(3)}
    \begin{columns}
        \column{0.4\textwidth}
            \centering
            \includesvg[width=0.8\textwidth]{images/chap03/directory_arc.svg}
        \column{0.6\textwidth}
        {\small
        一番上から見たら /\\
        カレントディレクトリが/のとき ./ カレントディレクトリがhomeのとき ../\\
        カレントディレクトリがpiのとき ../../\\
        \vskip.1\baselineskip
        一番上から見たら /home カレントディレクトリが/のとき ./home\\
        カレントディレクトリがhomeのとき ./\\
        カレントディレクトリがpiのとき ../\\
        \vskip.1\baselineskip
        一番上から見たら /home/pi カレントディレクトリが/のとき ./home/pi\\
        カレントディレクトリがhomeのとき ./pi\\
        カレントディレクトリがpiのとき ./\\
        
        \begin{itemize}
            \item カレントディレクトリは ./ を使うよ
            \item 上のディレクトリ ../ を使うよ
            \item ディレクトリを区切るとき / を使うよ
        \end{itemize}
        }
    \end{columns}
\end{frame}

\begin{frame}
    \frametitle{問題をといてみよう!}
    \begin{itemize}
        \item 教科書2ページ 問題3-1(3問)
        \item 教科書4ページ 問題3-2(5問)
    \end{itemize}
\end{frame}

\begin{frame}
    \frametitle{ターミナルを使ってみよう}
    \begin{itemize}
        \item ファイルやディレクトリを使ってたくさんのことができるようになるよ
        \item 命令を直接、入力することができるよ
        \item キーボードをカタカタしているとかっこいいね!
    \end{itemize}
    \begin{figure}[h]
        \centering
        \includesvg[width=0.5\columnwidth]{images/chap03/terminal.svg}
    \end{figure}
\end{frame}

\begin{frame}
    \frametitle{コマンドを使おう}
    \begin{itemize}
        \item コマンドはコンピュータに対する命令のこと
    \end{itemize}
    \begin{figure}[h]
        \centering
        \includesvg[width=0.9\columnwidth]{images/chap03/command1.svg}
    \end{figure}
\end{frame}

\begin{frame}
    \frametitle{コマンドの使い方}
    \begin{figure}[h]
        \centering
        \includesvg[width=0.9\columnwidth]{images/chap03/command2.svg}
    \end{figure}
    \begin{itemize}
        \item コマンドは動作、引数は動作の対象
        \item すべて半角の英語だよ
        \item 間にスペース(空白)がいるよ
            \begin{itemize}
                \item ここではスペースは␣と書いてあるよ
                \item 打つときには半角のスペースを打ってね
            \end{itemize}
        \item オプションはあるときとないときがあるよ
    \end{itemize}
\end{frame}

\begin{frame}
    \frametitle{Tabキーで楽しよう}
    \begin{itemize}
        \item Tab(タブ)キーを押すとそれまでに入力した文字から残りの文字をコンピュータが推測してくれるよ
        \item いくつかあるときは候補を表示するよ
    \end{itemize}
    \begin{figure}[h]
        \centering
        \includesvg[width=0.9\columnwidth]{images/chap03/tab.svg}
    \end{figure}
\end{frame}

\begin{frame}
    \frametitle{コマンド(1)}
    \begin{itemize}
        \item pwd
        \begin{itemize}
            \item 自分がどのディレクトリにいるかわかるよ
        \end{itemize}
        \item ls -F ディレクトリ
        \begin{itemize}
            \item ディレクトリの中のファイルを見ることができるよ
        \end{itemize}
        \item cd ディレクトリ
        \begin{itemize}
            \item ディレクトリへ移動することができるよ
        \end{itemize}
        \item cat ファイル
        \begin{itemize}
            \item ファイルに書かれている文字を表示することができるよ
        \end{itemize}
    \end{itemize}
\end{frame}

\begin{frame}
    \frametitle{コマンド(2)}
    \begin{itemize}
        \item cp コピーするファイル コピーした後のファイル
        \begin{itemize}
            \item ファイルをコピーする
        \end{itemize}
        \item cp -r コピーするディレクトリ コピーした後のディレクトリ
        \begin{itemize}
            \item ディレクトリをコピーするとき
        \end{itemize}
        \item いまあるファイルやディレクトリと同じ名前にならないように気を付けよう
    \end{itemize}
\end{frame}

\begin{frame}
    \frametitle{問題をといてみよう!}
    \begin{itemize}
        \item 教科書6ページ 問題3-3
        \item 教科書7ページ 問題3-4
        \item 教科書8ページ 問題3-5
        \item 教科書10ページ 問題3-6
        \item 教科書11ページ 問題3-7 (2問)
    \end{itemize}
\end{frame}

\begin{frame}
    \frametitle{コマンド(3)}
    \begin{itemize}
        \item mv␣移動したいファイルやディレクトリ␣移動先
        \begin{itemize}
            \item ファイルやディレクトリを移動できるよ
        \end{itemize}
        \item mv␣ファイルやディレクトリの名前␣変えた後の名前
        \begin{itemize}
            \item ファイルやディレクトリの名前を変えることができるよ
        \end{itemize}
        \item less␣ファイルの名前
        \begin{itemize}
            \item ファイルに書かれている文字を一画面ずつ見ることができるよ
        \end{itemize}
    \end{itemize}
\end{frame}

\begin{frame}
    \frametitle{コマンド(4)}
    \begin{itemize}
        \item rm␣ファイル
        \begin{itemize}
            \item ファイルを消したいときは
        \end{itemize}
        \item rm␣-r␣ディレクトリ
        \begin{itemize}
            \item ディレクトリを消したいときは
        \end{itemize}
        \item mkdir␣ディレクトリ
        \begin{itemize}
            \item 新しいディレクトリを作ることができるよ
        \end{itemize}
    \end{itemize}
    \begin{figure}[h]
        \centering
        \includesvg[width=0.5\columnwidth]{images/chap03/command3.svg}
    \end{figure}
\end{frame}

\begin{frame}
    \frametitle{問題をといてみよう!}
    \begin{itemize}
        \item 教科書12ページ 問題3-8
        \item 教科書12ページ 問題3-9
        \item 教科書13ページ 問題3-10
        \item 教科書13ページ 問題3-11
        \item 教科書14ページ 問題3-12(2問)
        \item 教科書15ページ 問題3-13(2問)
        \item 教科書17ページ 問題3-14
        \item 教科書19ページから21ページ まとめ問題3-15
    \end{itemize}
\end{frame}
