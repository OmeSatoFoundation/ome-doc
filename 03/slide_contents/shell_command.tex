\begin{frame}
    \frametitle{今日の目標} 
    \begin{itemize}
        \item 1時間目\\
        {\footnotesize「ファイル」と「ディレクトリ」の基本を知ろう:コンピュータでのデータ管理\\
        「ターミナル」で「コマンド」を使ってみよう:テキストでコンピュータに命令\\}
        \item 2時間目\\
        {\footnotesize「ターミナル」で「コマンド」を使ってみよう:テキストでコンピュータに命令\\}
        \item 3時間目\\
        {\footnotesize「センサーボード」の基本について知ろう:コンピュータと周りの世界のやりとり\\
        「センサーボード」を使ってスクリプトを書いてみよう:コンピュータに自動で仕事をさせる\\}
        \item 4時間目\\
        {\footnotesize「センサーボード」を使ってスクリプトを書いてみよう(応用)自分で0からスクリプトを書いてみよう:新しいプログラムをひとりで作ろう\\}
    \end{itemize}
\end{frame}

\begin{frame}
    \frametitle{ターミナルを使ってみよう}
    \begin{itemize}
        \item ファイルやディレクトリを使ってたくさんのことができるようになるよ
        \item 命令を直接、入力することができるよ
        \item キーボードをカタカタしているとかっこいいね!
    \end{itemize}
    \begin{figure}[h]
        \centering
        \includesvg[width=0.5\columnwidth]{images/chap03/terminal.svg}
    \end{figure}
\end{frame}

\begin{frame}
    \frametitle{コマンドってなんだろう}
    \begin{itemize}
        \item コマンドはコンピュータに対する命令のこと
    \end{itemize}
    \begin{figure}[h]
        \centering
        \includesvg[width=0.9\columnwidth]{images/chap03/command1.svg}
    \end{figure}
\end{frame}

\begin{frame}
    \frametitle{コマンドの使い方}
    \begin{figure}[h]
        \centering
        \includesvg[width=0.9\columnwidth]{images/chap03/command2.svg}
    \end{figure}
    \begin{itemize}
        \item コマンドは動作、引数は動作の対象
        \item すべて半角の英語だよ
        \item 間にスペース(空白)がいるよ
            \begin{itemize}
                \item ここではスペースは␣と書いてあるよ
                \item 打つときには半角のスペースを打ってね
            \end{itemize}
        \item オプションはあるときとないときがあるよ
    \end{itemize}
\end{frame}

\begin{frame}
    \frametitle{コマンドを使ってみよう}
    \begin{itemize}
        \item pwd
        \begin{itemize}
            \item カレントディレクトリ(自分がどのディレクトリにいるか)がわかるよ
        \end{itemize}
        \item ls -F ディレクトリ
        \begin{itemize}
            \item ディレクトリの中のファイルを見ることができるよ
        \end{itemize}
    \end{itemize}
\end{frame}

\begin{frame}
    \frametitle{Tabキーで楽しよう}
    \begin{itemize}
        \item Tab(タブ)キーを押すとそれまでに入力した文字から残りの文字をコンピュータが推測してくれるよ
        \item いくつかあるときは候補を表示するよ
    \end{itemize}
    \begin{figure}[h]
        \centering
        \includesvg[width=0.9\columnwidth]{images/chap03/tab.svg}
    \end{figure}
\end{frame}

\begin{frame}
    \frametitle{問題をといてみよう!}
    \begin{itemize}
        \item 教科書4ページ 問題3-1
    \end{itemize}
\end{frame}


\begin{frame}
    \frametitle{ファイルって何だろう}
    \begin{itemize}
        \item ファイルとは
        \begin{itemize}
            \item コンピュータに保存されるデータ
            \item 音楽、画像、文章などなど
        \end{itemize}
        \item 拡張子
        \begin{itemize}
            \item データ(ファイル)の種類を決めるもの
            \item データの名前.拡張子 で名前をつけるよ
        \end{itemize}
    \end{itemize}
    
    \begin{figure}[h]
    \centering
    \begin{minipage}[b]{0.32\columnwidth}
        \centering
        \includesvg[width=0.5\columnwidth]{images/chap03/image.svg}
        \caption{gazou.jpg}
    \end{minipage}
    \begin{minipage}[b]{0.32\columnwidth}
        \centering
        \includesvg[width=0.5\columnwidth]{images/chap03/oto.svg}
        \caption{oto.mp3}
    \end{minipage}
    \begin{minipage}[b]{0.32\columnwidth}
        \centering
        \includesvg[width=0.5\columnwidth]{images/chap03/douga.svg}
        \caption{douga.mp4}
    \end{minipage}
    \end{figure}
\end{frame}

\begin{frame}
    \frametitle{ディレクトリってなんだろう?}
    \begin{itemize}
        \item ファイルを整理整頓するためのもの
        \item ディレクトリの中にファイルやディレクトリをしまうよ
        \item 1回目で習ったフォルダと同じものだよ
    \end{itemize}
\end{frame}

\begin{frame}
    \frametitle{ラズパイの主なディレクトリ}
    \begin{minipage}{0.3\hsize}
        {\footnotesize 
        \begin{forest}
          for tree={grow'=0,folder, scale=0.7}
          [/
            [bin/]
            [dev/]
            [etc/]
            [home/
              [あなたのユーザ名/]]
            [sbin/]
            [tmp/]
            [usr/
              [bin/]
              [lib/]
              [local/]]
            [var/]
          ]
        \end{forest}
        }
    \end{minipage}
    \begin{minipage}{0.65\hsize}
        \footnotesize
        \begin{itemize}
        \item [/bin] システムの動作に必要なコマンドの実行ファイルを置くためのディレクトリ
        \item [/dev] ディスクやキーボードなどのハードウェアを操作するためのファイルを格納する
        \item [/etc] ラズパイで動作するさまざまなアプリケーションの動作を設定するためのテキストファイルが置かれる
        \item [/home] ホームディレクトリが配置される
        \item [/sbin] shutdownなどの管理者用のコマンドを置くためのディレクトリ
        \item [/tmp] 一時的な作業ファイル(テンポラリファイル)を置くためのディレクトリ
        \item [/usr] 追加でアプリケーションをインストールした場合にはこのディレクトリの下に配置される
        \item [/var] 変化するデータを置くためのディレクトリ
        \end{itemize}
    \end{minipage}
\end{frame}

\begin{frame}
    \frametitle{問題をといてみよう!}
    \begin{itemize}
        \item 教科書6ページ 問題3-2
    \end{itemize}
\end{frame}

\begin{frame}
    \frametitle{ルートディレクトリとホームディレクトリ}
    \begin{minipage}{0.45\hsize}
        \footnotesize
        \centering
        \includesvg[width=0.7\textwidth]{images/chap03/root_directory.svg}
        \begin{itemize}
            \item 一番上のディレクトリをルートディレクトリ( / )と呼ぶよ
            \item コンピュータの設定を変える大事なファイルもあるので、間違って消さないように気を付けよう!
        \end{itemize}
    \end{minipage}
    \begin{minipage}{0.45\hsize}
        \footnotesize
        \centering
        \includesvg[width=0.7\textwidth]{images/chap03/home_directory.svg}
        \begin{itemize}
            \item みんなはホームディレクトリ\\( /home/あなたのユーザ名 )で作業するよ。ここではユーザ名をpiにしているよ
            \item ホームディレクトリは「\textasciitilde」(チルダ)を使って表せるよ
        \end{itemize}
    \end{minipage}
\end{frame}

\begin{frame}
    \frametitle{カレントディレクトリ}
    \begin{itemize}
        \item 自分がいま使っているディレクトリはカレントディレクトリだよ
        \item カレントディレクトリは「.」と書けるよ
        \item カレントディレクトリからひとつ上のディレクトリは「..」と書けるよ
    \end{itemize}
\end{frame}

\begin{frame}
    \frametitle{絶対パスと相対パス}
    ディレクトリの位置は、絶対パスと相対パスの2種類で指定できるよ\\
    \begin{itemize}
        \item 絶対パス
        \begin{itemize}
            \item ディレクトリの位置をルートディレクトリ(/)から見てどこにあるかで指定するよ
            \item 絶対パスの先頭には / がついているよ (/usr/bin など)
        \end{itemize}
        \item 相対パス
        \begin{itemize}
            \item ディレクトリの位置をカレントディレクトリから見てどこにあるかで指定するよ
            \item 相対パスには先頭に / がついていないよ (./usr/bin や usr/bin など)
        \end{itemize}
        \item ホームディレクトリ
        \begin{itemize}
            \item ホームディレクトリは「\textasciitilde」(チルダ)を使って表せるよ
            \item 「\textasciitilde」(チルダ)と「/home/あなたのユーザ名」は同じだよ
        \end{itemize}
    \end{itemize}
\end{frame}

\begin{frame}
    \frametitle{パスを指定しよう}
    \begin{figure}
        \begin{minipage}{0.2\hsize}
            \begin{forest}
            for tree={grow'=0,folder}
            [/, draw
              [home/
                [pi/]]]
            \end{forest}
        \end{minipage}
        \begin{minipage}{0.75\hsize}
            \footnotesize
            \begin{tabular}{ll}
                絶対パス & /からの相対パス \\ \hline
                / & . \\ \hline
                /home\quad/home/ & \begin{tabular}{l} home\quad./home  \\ home/\quad./home/ \end{tabular} \\ \hline 
                /home/pi\quad/home/pi/ & \begin{tabular}{l} home/pi\quad./home/pi  \\ home/pi/\quad./home/pi/ \end{tabular} \\ \hline
            \end{tabular}
        \end{minipage}
    \end{figure}
      
    \begin{figure}[H]
        \begin{minipage}{0.2\hsize}
            \begin{forest}
                for tree={grow'=0,folder}
                [/
                    [home/ , draw
                        [pi/]]]
            \end{forest}
        \end{minipage}
        \begin{minipage}{0.75\hsize}
        \footnotesize
            \begin{tabular}{ll}
                絶対パス & /home/からの相対パス \\ \hline
                / & \begin{tabular}{l} ..\quad../ \\\quad./..\quad./../ \end{tabular} \\ \hline
                /home\quad/home/ & \begin{tabular}{l} . \\./  \end{tabular} \\ \hline 
                /home/pi\quad/home/pi/ & \begin{tabular}{l} pi\quad./pi \\ pi/\quad./pi/ \end{tabular} \\ \hline
            \end{tabular}
        \end{minipage}
    \end{figure}
\end{frame}

\begin{frame}
    \frametitle{パスを指定しよう}   
    \begin{figure}[H]
        \begin{minipage}{0.2\hsize}
            \begin{forest}
                for tree={grow'=0,folder}
                [/
                    [home/
                        [pi/ , draw ]]]
            \end{forest}
        \end{minipage}
        \begin{minipage}{0.75\hsize}
        \footnotesize
            \begin{tabular}{ll}
                絶対パス & /home/pi/からの相対パス \\ \hline
                / & \begin{tabular}{l} ../..\quad../../ \\ ./../..\quad./../../ \end{tabular} \\ \hline
                /home\quad/home/ & \begin{tabular}{l} ..\quad../ \\ ./..\quad./../ \end{tabular} \\ \hline 
                /home/pi\quad/home/pi/ & \begin{tabular}{l} . \\ ./ \end{tabular} \\ \hline
            \end{tabular}
        \end{minipage}
    \end{figure}
\end{frame}

\begin{frame}
    \frametitle{問題をといてみよう!}
    \begin{itemize}
        \item 教科書9\textasciitilde10ページ 問題3-3
    \end{itemize}
\end{frame}

\begin{frame}
    \frametitle{ターミナルでファイルを操作してみよう}
    \begin{itemize}
        \item cd␣ディレクトリ
        \begin{itemize}
            \item カレントディレクトリを指定したディレクトリに移動するよ
        \end{itemize}
        \item cd
        \begin{itemize}
            \item ホームディレクトリに移動するよ
        \end{itemize}
        \item cat␣ファイル
        \begin{itemize}
            \item ファイルに書かれている文字を表示するよ
        \end{itemize}
        \item cp␣ファイル1␣ファイル2
        \begin{itemize}
            \item ファイル1をファイル2という名前でコピーするよ
        \end{itemize}
        \item cp␣-r␣ディレクトリ1␣ディレクトリ2
        \begin{itemize}
            \item ディレクトリ1をディレクトリ2という名前でコピーするよ
        \end{itemize}
        \item mv␣移動したいファイルやディレクトリ␣移動先
        \begin{itemize}
            \item ファイルやディレクトリを移動するよ
        \end{itemize}
        \item mv␣ファイルやディレクトリの名前␣変えたいの名前
        \begin{itemize}
            \item ファイルやディレクトリの名前を変えるよ
        \end{itemize}
    \end{itemize}
\end{frame}

\begin{frame}
    \frametitle{問題をといてみよう!}
    \begin{itemize}
        \item 教科書13ページ 問題3-4
        \item 教科書14ページ 問題3-5
    \end{itemize}
\end{frame}

\begin{frame}
    \frametitle{ターミナルでファイルを操作してみよう}
    \begin{itemize}
        \item less␣ファイル
        \begin{itemize}
            \item ファイルに書かれている文字を一画面ずつ見ることができるよ
            \item eを押すと一行進んで、yを押すと一行戻るよ
            \item qを押すと終わるよ
        \end{itemize}
        \item mousepad␣ファイル␣\&
        \begin{itemize}
            \item mousepadを使ってファイルを作成または編集するよ
        \end{itemize}
    \end{itemize}
\end{frame}

\begin{frame}
    \frametitle{ターミナルでファイルを操作してみよう}
    \begin{itemize}
        \item mkdir␣ディレクトリ
        \begin{itemize}
            \item 新しいディレクトリを作ることができるよ
        \end{itemize}
        \item rm␣ファイル
        \begin{itemize}
            \item ファイルを消したいときは
        \end{itemize}
        \item rm␣-r␣ディレクトリ
        \begin{itemize}
            \item ディレクトリを消したいときは
        \end{itemize}
    \end{itemize}
    \begin{figure}[h]
        \centering
        \includesvg[width=0.5\columnwidth]{images/chap03/command3.svg}
    \end{figure}
\end{frame}

\begin{frame}
    \frametitle{問題をといてみよう!}
    \begin{itemize}
        \item 教科書17ページ 問題3-6
    \end{itemize}
\end{frame}
