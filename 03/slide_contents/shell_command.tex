\begin{frame}
    \frametitle{今日の目標} 
    \begin{itemize}
        \item 1時間目\\
        {\footnotesize「ファイル」と「ディレクトリ」の基本を知ろう:コンピュータでのデータ管理\\
        「ターミナル」で「コマンド」を使ってみよう:テキストでコンピュータに命令\\}
        \item 2時間目\\
        {\footnotesize「ターミナル」で「コマンド」を使ってみよう:テキストでコンピュータに命令\\}
        \item 3時間目\\
        {\footnotesize「センサーボード」の基本について知ろう:コンピュータと周りの世界のやりとり\\
        「センサーボード」を使ってスクリプトを書いてみよう:コンピュータに自動で仕事をさせる\\}
        \item 4時間目\\
        {\footnotesize「センサーボード」を使ってスクリプトを書いてみよう(応用)自分で0からスクリプトを書いてみよう:新しいプログラムをひとりで作ろう\\}
    \end{itemize}
\end{frame}

\begin{frame}
    \frametitle{ターミナルを使ってみよう}
    \begin{itemize}
        \item ファイルやディレクトリを使ってたくさんのことができるようになるよ
        \item 命令を直接、入力することができるよ
        \item キーボードをカタカタしているとかっこいいね!
    \end{itemize}
    \begin{figure}[h]
        \centering
        \includesvg[width=0.5\columnwidth]{images/chap03/terminal.svg}
    \end{figure}
\end{frame}

\begin{frame}
    \frametitle{コマンドってなんだろう}
    \begin{itemize}
        \item コマンドはコンピュータに対する命令のこと
    \end{itemize}
    \begin{figure}[h]
        \centering
        \includesvg[width=0.9\columnwidth]{images/chap03/command1.svg}
    \end{figure}
\end{frame}

\begin{frame}
    \frametitle{コマンドの使い方}
    \begin{figure}[h]
        \centering
        \includesvg[width=0.9\columnwidth]{images/chap03/command2.svg}
    \end{figure}
    \begin{itemize}
        \item コマンドは動作、引数は動作の対象
        \item すべて半角の英語だよ
        \item 間にスペース(空白)がいるよ
            \begin{itemize}
                \item ここではスペースは␣と書いてあるよ
                \item 打つときには半角のスペースを打ってね
            \end{itemize}
        \item オプションはあるときとないときがあるよ
    \end{itemize}
\end{frame}

\begin{frame}
    \frametitle{コマンドを使ってみよう}
    \begin{itemize}
        \item pwd
        \begin{itemize}
            \item カレントディレクトリ(自分がどのディレクトリにいるか)がわかるよ
        \end{itemize}
        \item ls -F ディレクトリ
        \begin{itemize}
            \item ディレクトリの中のファイルを見ることができるよ
        \end{itemize}
    \end{itemize}
\end{frame}

\begin{frame}
    \frametitle{Tabキーで楽しよう}
    \begin{itemize}
        \item Tab(タブ)キーを押すとそれまでに入力した文字から残りの文字をコンピュータが推測してくれるよ
        \item いくつかあるときは候補を表示するよ
    \end{itemize}
    \begin{figure}[h]
        \centering
        \includesvg[width=0.9\columnwidth]{images/chap03/tab.svg}
    \end{figure}
\end{frame}

\begin{frame}
    \frametitle{問題をといてみよう!}
    \begin{itemize}
        \item 教科書4ページ 問題3-1
    \end{itemize}
\end{frame}


\begin{frame}
    \frametitle{ファイルって何だろう}
    \begin{itemize}
        \item ファイルとは
        \begin{itemize}
            \item コンピュータに保存されるデータ
            \item 音楽、画像、文章などなど
        \end{itemize}
        \item 拡張子
        \begin{itemize}
            \item データ(ファイル)の種類を決めるもの
            \item データの名前.拡張子 で名前をつけるよ
        \end{itemize}
    \end{itemize}
    
    \begin{figure}[h]
    \centering
    \begin{minipage}[b]{0.32\columnwidth}
        \centering
        \includesvg[width=0.5\columnwidth]{images/chap03/image.svg}
        \caption{gazou.jpg}
    \end{minipage}
    \begin{minipage}[b]{0.32\columnwidth}
        \centering
        \includesvg[width=0.5\columnwidth]{images/chap03/oto.svg}
        \caption{oto.mp3}
    \end{minipage}
    \begin{minipage}[b]{0.32\columnwidth}
        \centering
        \includesvg[width=0.5\columnwidth]{images/chap03/douga.svg}
        \caption{douga.mp4}
    \end{minipage}
    \end{figure}
\end{frame}

\begin{frame}
    \frametitle{ディレクトリってなんだろう?}
    \begin{itemize}
        \item ファイルを整理整頓するためのもの
        \item ディレクトリの中にファイルやディレクトリをしまうよ
        \item 1回目で習ったフォルダと同じものだよ
    \end{itemize}
\end{frame}

\begin{frame}
    \frametitle{ラズパイの主なディレクトリ}
    \begin{minipage}{0.3\hsize}
        {\footnotesize 
        \begin{forest}
          for tree={grow'=0,folder, scale=0.7}
          [/
            [bin/]
            [dev/]
            [etc/]
            [home/
              [あなたのユーザ名/]]
            [sbin/]
            [tmp/]
            [usr/
              [bin/]
              [lib/]
              [local/]]
            [var/]
          ]
        \end{forest}
        }
    \end{minipage}
    \begin{minipage}{0.65\hsize}
        \footnotesize
        \begin{itemize}
        \item [/bin] システムの動作に必要なコマンドの実行ファイルを置くためのディレクトリ
        \item [/dev] ディスクやキーボードなどのハードウェアを操作するためのファイルを格納する
        \item [/etc] ラズパイで動作するさまざまなアプリケーションの動作を設定するためのテキストファイルが置かれる
        \item [/home] ホームディレクトリが配置される
        \item [/sbin] shutdownなどの管理者用のコマンドを置くためのディレクトリ
        \item [/tmp] 一時的な作業ファイル(テンポラリファイル)を置くためのディレクトリ
        \item [/usr] 追加でアプリケーションをインストールした場合にはこのディレクトリの下に配置される
        \item [/var] 変化するデータを置くためのディレクトリ
        \end{itemize}
    \end{minipage}
\end{frame}

\begin{frame}
    \frametitle{問題をといてみよう!}
    \begin{itemize}
        \item 教科書6ページ 問題3-2
    \end{itemize}
\end{frame}

\begin{frame}
    \frametitle{ルートディレクトリとホームディレクトリ}
    \begin{minipage}{0.45\hsize}
        \footnotesize
        \centering
        \includesvg[width=0.7\textwidth]{images/chap03/root_directory.svg}
        \begin{itemize}
            \item 一番上のディレクトリをルートディレクトリ( / )と呼ぶよ
            \item コンピュータの設定を変える大事なファイルもあるので、間違って消さないように気を付けよう!
        \end{itemize}
    \end{minipage}
    \begin{minipage}{0.45\hsize}
        \footnotesize
        \centering
        \includesvg[width=0.7\textwidth]{images/chap03/home_directory.svg}
        \begin{itemize}
            \item みんなはホームディレクトリ\\( /home/あなたのユーザ名 )で作業するよ。ここではユーザ名をpiにしているよ
            \item ホームディレクトリは「\textasciitilde」(チルダ)を使って表せるよ
        \end{itemize}
    \end{minipage}
\end{frame}

\begin{frame}
    \frametitle{カレントディレクトリ}
    \begin{itemize}
        \item 自分がいま使っているディレクトリはカレントディレクトリだよ
        \item カレントディレクトリは「.」と書けるよ
        \item カレントディレクトリからひとつ上のディレクトリは「..」と書けるよ
    \end{itemize}
\end{frame}

\begin{frame}
    \frametitle{絶対パスと相対パス}
    ディレクトリの位置は、絶対パスと相対パスの2種類で指定できるよ\\
    \begin{itemize}
        \item 絶対パス
        \begin{itemize}
            \item ディレクトリの位置をルートディレクトリ(/)から見てどこにあるかで指定するよ
            \item 絶対パスの先頭には / がついているよ (/usr/bin など)
        \end{itemize}
        \item 相対パス
        \begin{itemize}
            \item ディレクトリの位置をカレントディレクトリから見てどこにあるかで指定するよ
            \item 相対パスには先頭に / がついていないよ (./usr/bin や usr/bin など)
        \end{itemize}
        \item ホームディレクトリ
        \begin{itemize}
            \item ホームディレクトリは「\textasciitilde」(チルダ)を使って表せるよ
            \item 「\textasciitilde」(チルダ)と「/home/あなたのユーザ名」は同じだよ
        \end{itemize}
    \end{itemize}
\end{frame}

\begin{frame}
    \frametitle{パスを指定しよう}
    \begin{figure}
        \begin{minipage}{0.2\hsize}
            \begin{forest}
            for tree={grow'=0,folder}
            [/, draw
              [home/
                [pi/]]]
            \end{forest}
        \end{minipage}
        \begin{minipage}{0.75\hsize}
            \footnotesize
            \begin{tabular}{ll}
                絶対パス & /からの相対パス \\ \hline
                / & . \\ \hline
                /home\quad/home/ & \begin{tabular}{l} home\quad./home  \\ home/\quad./home/ \end{tabular} \\ \hline 
                /home/pi\quad/home/pi/ & \begin{tabular}{l} home/pi\quad./home/pi  \\ home/pi/\quad./home/pi/ \end{tabular} \\ \hline
            \end{tabular}
        \end{minipage}
    \end{figure}
      
    \begin{figure}[H]
        \begin{minipage}{0.2\hsize}
            \begin{forest}
                for tree={grow'=0,folder}
                [/
                    [home/ , draw
                        [pi/]]]
            \end{forest}
        \end{minipage}
        \begin{minipage}{0.75\hsize}
        \footnotesize
            \begin{tabular}{ll}
                絶対パス & /home/からの相対パス \\ \hline
                / & \begin{tabular}{l} ..\quad../ \\\quad./..\quad./../ \end{tabular} \\ \hline
                /home\quad/home/ & \begin{tabular}{l} . \\./  \end{tabular} \\ \hline 
                /home/pi\quad/home/pi/ & \begin{tabular}{l} pi\quad./pi \\ pi/\quad./pi/ \end{tabular} \\ \hline
            \end{tabular}
        \end{minipage}
    \end{figure}
\end{frame}

\begin{frame}
    \frametitle{パスを指定しよう}   
    \begin{figure}[H]
        \begin{minipage}{0.2\hsize}
            \begin{forest}
                for tree={grow'=0,folder}
                [/
                    [home/
                        [pi/ , draw ]]]
            \end{forest}
        \end{minipage}
        \begin{minipage}{0.75\hsize}
        \footnotesize
            \begin{tabular}{ll}
                絶対パス & /home/pi/からの相対パス \\ \hline
                / & \begin{tabular}{l} ../..\quad../../ \\ ./../..\quad./../../ \end{tabular} \\ \hline
                /home\quad/home/ & \begin{tabular}{l} ..\quad../ \\ ./..\quad./../ \end{tabular} \\ \hline 
                /home/pi\quad/home/pi/ & \begin{tabular}{l} . \\ ./ \end{tabular} \\ \hline
            \end{tabular}
        \end{minipage}
    \end{figure}
\end{frame}

\begin{frame}
    \frametitle{問題をといてみよう!}
    \begin{itemize}
        \item 教科書9\textasciitilde10ページ 問題3-3
    \end{itemize}
\end{frame}

\begin{frame}
    \frametitle{ターミナルでファイルを操作してみよう}
    \begin{itemize}
        \item cd␣ディレクトリ
        \begin{itemize}
            \item カレントディレクトリを指定したディレクトリに移動するよ
        \end{itemize}
        \item cd
        \begin{itemize}
            \item ホームディレクトリに移動するよ
        \end{itemize}
        \item cat␣ファイル
        \begin{itemize}
            \item ファイルに書かれている文字を表示するよ
        \end{itemize}
        \item cp␣ファイル1␣ファイル2
        \begin{itemize}
            \item ファイル1をファイル2という名前でコピーするよ
        \end{itemize}
        \item cp␣-r␣ディレクトリ1␣ディレクトリ2
        \begin{itemize}
            \item ディレクトリ1をディレクトリ2という名前でコピーするよ
        \end{itemize}
        \item mv␣移動したいファイルやディレクトリ␣移動先
        \begin{itemize}
            \item ファイルやディレクトリを移動するよ
        \end{itemize}
        \item mv␣ファイルやディレクトリの名前␣変えたい名前
        \begin{itemize}
            \item ファイルやディレクトリの名前を変えるよ
        \end{itemize}
    \end{itemize}
\end{frame}

\begin{frame}
    \frametitle{問題をといてみよう!}
    \begin{itemize}
        \item 教科書13ページ 問題3-4
        \item 教科書14ページ 問題3-5
    \end{itemize}
\end{frame}

\begin{frame}
    \frametitle{ターミナルでファイルを操作してみよう}
    \begin{itemize}
        \item less␣ファイル
        \begin{itemize}
            \item ファイルに書かれている文字を一画面ずつ見ることができるよ
            \item eを押すと一行進んで、yを押すと一行戻るよ
            \item qを押すと終わるよ
        \end{itemize}
        \item mousepad␣ファイル␣\&
        \begin{itemize}
            \item mousepadを使ってファイルを作成または編集するよ
        \end{itemize}
    \end{itemize}
\end{frame}

\begin{frame}
    \frametitle{ターミナルでファイルを操作してみよう}
    \begin{itemize}
        \item mkdir␣ディレクトリ
        \begin{itemize}
            \item 新しいディレクトリを作ることができるよ
        \end{itemize}
        \item rm␣ファイル
        \begin{itemize}
            \item ファイルを消したいときは
        \end{itemize}
        \item rm␣-r␣ディレクトリ
        \begin{itemize}
            \item ディレクトリを消したいときは
        \end{itemize}
    \end{itemize}
    \begin{figure}[h]
        \centering
        \includesvg[width=0.5\columnwidth]{images/chap03/command3.svg}
    \end{figure}
\end{frame}

\begin{frame}
    \frametitle{問題をといてみよう!}
    \begin{itemize}
        \item 教科書17ページ 問題3-6
    \end{itemize}
\end{frame}

\begin{frame}
    \frametitle{コマンドの入出力}
    コマンドを実行すると3つのデータの通り道(チャネル)が準備されるよ
    \begin{itemize}
        \item 標準入力
        \item 標準出力
        \item 標準エラー出力
    \end{itemize}
    \begin{figure}[h]
        \centering
        \includesvg[width=0.8\columnwidth]{images/chap03/std_in_out_err.svg}
    \end{figure}
\end{frame}

\begin{frame}[fragile]
    \frametitle{catコマンド}
    \begin{itemize}
        \item cat␣ファイル
        \begin{itemize}
            \item ファイルの中身を標準出力(ディスプレイ)に表示する
        \end{itemize}
        \item cat
        \begin{itemize}
            \item ファイルを指定しないと標準入力(キーボード)からデータを受け取るよ
        \end{itemize}
    \end{itemize}
    \begin{lstlisting}[title=catの標準入力・標準出力, label=stdioCat]
    <#green#pi@raspberrypi#>:<#blue#~ $#> cat 
    sansu <Enter> <- 標準入力(キーボード)からの入力
    sansu         <- 標準出力(ディスプレイ)への出力
    <Ctrl+D>      <- 標準入力からEOFを入力し、入力が終了したことを伝える
    <#green#pi@raspberrypi#>:<#blue#~ $#>
    \end{lstlisting}
\end{frame}

\begin{frame}
    \frametitle{リダイレクトってなんだろう?}
    \begin{itemize}
        \item 標準入力、標準出力、標準エラー出力の出力先をファイルに変更することだよ
    \end{itemize}
    \begin{figure}
        \centering
        \includesvg[width=0.7\linewidth]{images/chap03/redirect.svg}
    \end{figure}
\end{frame}

\begin{frame}[fragile]
    \frametitle{リダイレクトの例}
    \begin{lstlisting}[title=lsの出力をリダイレクトする, label=redirectLs]
    <#green#pi@raspberrypi#>:<#blue#~ $#> ls 
    <#blue#01  03         Desktop    Downloads  Pictures  Templates
    02  Bookshelf  Documents  Music      Public    Videos#>
    <#green#pi@raspberrypi#>:<#blue#~ $#> ls > lsfile
    <#green#pi@raspberrypi#>:<#blue#~ $#> cat lsfile
    01
    02
    03
    Bookshelf
    Desktop
    Documents
    Downloads
    Music
    Pictures
    Public
    Templates
    Videos
    lsfile
    \end{lstlisting}
\end{frame}

\begin{frame}
    \frametitle{パイプライン}
    \begin{itemize}
        \item パイプ(|)を使って標準出力と標準入力をつなげることができるよ
    \end{itemize}
    \begin{figure}
        \centering
        \includesvg[width=1\linewidth]{images/chap03/pipe.svg}
    \end{figure}
\end{frame}

\begin{frame}[fragile]
    \frametitle{パイプラインの例}
    \begin{lstlisting}[title=lsコマンドの出力をパイプでlessコマンドに渡す, label=redirectCat]
    <#green#pi@raspberrypi#>:<#blue#~ $#> ls | less
    01
    02
    03
    Bookshelf
    Desktop
    Documents
    Downloads
    Music
    Pictures
    Public
    Templates
    Videos
    lsfile
    \end{lstlisting}
\end{frame}

\begin{frame}
    \frametitle{xargsコマンドってなんだろう?}
    \begin{itemize}
        \item 標準入力を受け取り、実行したいコマンドの引数として使えるよ
        \item xargsコマンドを使うと引数にいろいろなものを指定できるよ
    \end{itemize}
    \begin{figure}
        \centering
        \includesvg[width=0.55\linewidth]{images/chap03/xargs_command.svg}
    \end{figure}
\end{frame}

\begin{frame}[fragile]
    \frametitle{xargsコマンドを使う準備をしよう}
    \begin{itemize}
        \item xargsコマンドを使うためのディレクトリ\textasciitilde/03/rensyu/xargstestを作ろう
    \end{itemize}
    \begin{lstlisting}
    <#green#pi@raspberrypi#>:<#blue#~ $#> cd 03/rensyu
    <#green#pi@raspberrypi#>:<#blue#~/03/rensyu $#> mkdir xargstest
    <#green#pi@raspberrypi#>:<#blue#~/03/rensyu $#> cp ~/lsfile ./xargstest
    <#green#pi@raspberrypi#>:<#blue#~/03/rensyu $#> cp ./kokugo/syousetu.txt ./xargstest
    <#green#pi@raspberrypi#>:<#blue#~/03/rensyu $#> cd xargstest
    <#green#pi@raspberrypi#>:<#blue#~/03/rensyu/xargstest $#> ls
    <#magenta#lsfile  syousetu.txt#>
    \end{lstlisting}
\end{frame}

\begin{frame}[fragile]
    \frametitle{xargsコマンドを実際に使ってみよう}
    \begin{lstlisting}[title=xargsコマンドを使ってcatコマンドを使う]
    <#green#pi@raspberrypi#>:<#blue#~/03/rensyu/xargstest $#> ls | xargs cat
    01
    02
    03
    Bookshelf
    Desktop
                                                   ...
    このファイルは、インターネットの図書館、青空文庫(http://www.aozora.gr.jp/)で作られました。
    入力、校正、制作にあたったのは、ボランティアの皆さんです。
        
        
        
    https://www.aozora.gr.jp/cards/000121/files/628_14895.html
    <#green#pi@raspberrypi#>:<#blue#~/03/rensyu/xargs $#>
    \end{lstlisting}
\end{frame}

\begin{frame}[fragile]
    \frametitle{出力を作るコマンド}
    \begin{itemize}
        \item seq␣数字1␣数字2
        \begin{itemize}
            \item 数字1から数字2までの数字を順番に出力するよ
            \begin{lstlisting}
            <#green#pi@raspberrypi#>:<#blue#~/03/rensyu/xargs $#> seq 1 5
            1
            2
            3
            4
            5
            <#green#pi@raspberrypi#>:<#blue#~/03/rensyu/xargs $#>
            \end{lstlisting}
        \end{itemize}
        \item echo␣文字
        \begin{itemize}
            \item 文字をそのまま出力するよ
            \begin{lstlisting}
            <#green#pi@raspberrypi#>:<#blue#~/03/rensyu/xargs $#> echo hello
            hello
            <#green#pi@raspberrypi#>:<#blue#~/03/rensyu/xargs $#>
            \end{lstlisting}
        \end{itemize}
    \end{itemize}
\end{frame}

\begin{frame}
    \frametitle{xargsコマンドのオプション}
    \begin{itemize}
        \item xargs␣-p␣コマンド
        \begin{itemize}
            \item どのようなコマンドが実行されるかを表示するよ
        \end{itemize}
        \item xargs␣-i␣コマンド␣\{\}
        \begin{itemize}
            \item 標準入力を1つずつ受け取って{}の中に当てはめ、コマンドを実行するよ
        \end{itemize}
        \item xargs␣-L␣数字␣コマンド
        \begin{itemize}
            \item xargsで一度にコマンドを渡す引数の最大数を指定するよ
            \item 標準出力から渡されたデータ数が-Lオプションの指定した数より大きい場合は、すべての入力が終わるまでコマンドが繰り返されるよ
            \item -iオプションと-Lオプションは一緒に使えないよ
        \end{itemize}
    \end{itemize}
\end{frame}

\begin{frame}
    \frametitle{出力を作るコマンド}
    \begin{tabular}{ll}
        コマンド & 動作                                         \\ \hline
        ls       & ファイルやディレクトリを出力する             \\
        du       & ディレクトリの中のファイルの大きさを報告する \\
        wc       & 入力の文字数・単語数・行数を出力する         \\
        echo     & 文字をそのまま出力する                       \\ \hline
    \end{tabular}
\end{frame}

\begin{frame}
    \frametitle{duコマンド}
    du␣大きさを調べたいディレクトリやファイルのパス
    \begin{itemize}
        \item ディレクトリやファイルの大きさを報告するコマンドだよ
    \end{itemize}
    duコマンドのオプション
    \begin{itemize}
        \item  -hオプション
        \begin{itemize}
            \item ディレクトリの大きさの数字に単位を付けるよ
        \end{itemize}
        \item -aオプション
        \begin{itemize}
            \item ディレクトリ内のファイルの大きさも表示するよ
        \end{itemize}
        \item -sオプション
        \begin{itemize}
            \item ディレクトリの大きさの合計のみを表示するよ
        \end{itemize}
    \end{itemize}
\end{frame}

\begin{frame}[fragile]
    \frametitle{wcコマンド}
    wc␣対象のファイルのパス
    \begin{itemize}
        \item 入力の文字数・単語数・行数を出力するコマンドだよ
    \end{itemize}
    \begin{lstlisting}[title=wcコマンドの実行例, label=wc_example]
    <#green#pi@raspberrypi#>:<#blue#~ $#> wc ~/lsfile
    13 13 93 /home/pi/lsfile
    \end{lstlisting}
\end{frame}

\begin{frame}
    \frametitle{フィルタコマンド}
    \begin{tabular}{ll}
        コマンド & 動作                               \\ \hline
        cat      & 入力をなにもせずに出力する         \\
        tac      & 行を逆順に出力する                 \\
        shuf     & 行をランダムに入れ替えて出力する   \\
        head     & 先頭のいくつかの行を表示する       \\
        tail     & 末尾のいくつかの行を表示する       \\
        sort     & 行を順番にならべかえる             \\
        grep     & 検索パターンに一致する行を出力する \\ \hline
    \end{tabular}
\end{frame}

\begin{frame}[fragile]
    \frametitle{tacコマンド}
    tac␣表示したいファイルのパス
    \begin{itemize}
        \item 行を逆順に出力するコマンドだよ
    \end{itemize}
    \begin{lstlisting}[title=tacコマンドの実行例, label=tac_example]
    <#green#pi@raspberrypi#>:<#blue#~ $#> tac ~/lsfile
    lsfile
    Videos
    Templates
    Public
    Pictures
    Music
    Downloads
    Documents
    Desktop
    Bookshelf
    03
    02
    01 
    \end{lstlisting}
\end{frame}

\begin{frame}[fragile]
    \frametitle{shufコマンド}
    shuf␣表示したいファイルのパス
    \begin{itemize}
        \item 行をランダムに入れ替えて出力するコマンドだよ
    \end{itemize}
    \begin{lstlisting}[title=shufコマンドの実行例, label=shuf_example]
    <#green#pi@raspberrypi#>:<#blue#~ $#> shuf ~/lsfile
    03
    Desktop
    Documents
    02
    Music
    01
    Videos
    Pictures
    Bookshelf
    Templates
    Public
    lsfile
    Downloads
    \end{lstlisting}
\end{frame}

\begin{frame}[fragile]
    \frametitle{headコマンド}
    head␣表示したいファイルのパス␣-n␣行数
    \begin{itemize}
        \item 先頭のいくつかの行を表示するコマンドだよ
        \item 先頭から指定した行数分だけ表示されるよ
    \end{itemize}
    \begin{lstlisting}[title=headコマンドの実行例, label=shuf_example]
    <#green#pi@raspberrypi#>:<#blue#~ $#> head ~/lsfile -n 2
    01
    02
    \end{lstlisting}
\end{frame}

\begin{frame}[fragile]
    \frametitle{tailコマンド}
    tail␣表示したいファイルのパス␣-n␣行数
    \begin{itemize}
        \item 末尾のいくつかの行を表示するコマンドだよ
        \item 末尾から指定した行数分だけ表示されるよ
    \end{itemize}
    \begin{lstlisting}[title=tailコマンドの実行例, label=shuf_example]
    <#green#pi@raspberrypi#>:<#blue#~ $#> tail ~/lsfile -n 2
    Videos
    lsfile
    \end{lstlisting}
\end{frame}

\begin{frame}[fragile]
    \frametitle{sortコマンド}
    sort␣表示したいファイルのパス
    \begin{itemize}
        \item 行を順番に並べ替えるコマンドだよ
        \item ファイルの名前が数字のものが先に表示されるよ
    \end{itemize}
    \begin{lstlisting}[title=sortコマンドの実行例, label=sort_example]
    <#green#pi@raspberrypi#>:<#blue#~ $#> sort ~/lsfile
    01
    02
    03
    Bookshelf
    Desktop
    Documents
    Downloads
    lsfile
    Music
    Pictures
    Public
    Templates
    Videos
    \end{lstlisting}
\end{frame}

\begin{frame}[fragile]
    \frametitle{grepコマンド}
    grep␣検索パターン␣表示したいファイルのパス
    \begin{itemize}
        \item 検索パターンに一致する行を出力するコマンドだよ
        \item 一致する文字列は赤字で表示されるよ
        \item 一致する行がある場合はその行のみが表示されるよ
    \end{itemize}
    \begin{lstlisting}[title=grepコマンドの実行例, label=grep_example]
    <#green#pi@raspberrypi#>:<#blue#~ $#> grep 03 ~/lsfile
    <#red#03#>
    \end{lstlisting}
\end{frame}

\begin{frame}[fragile]
    \frametitle{パイプラインとフィルタコマンドを組み合わせよう}
    標準出力を出すコマンド | フィルタコマンド
    \begin{itemize}
        \item 標準出力を出すコマンドの結果をフィルタコマンドに渡すよ
    \end{itemize}
    \begin{lstlisting}[title=パイプラインを用いたsortコマンドの実行例, label=sort_example]
    <#green#pi@raspberrypi#>:<#blue#~ $#> ls ~/03 | sort
    01
    02
    03
    Bookshelf
    Desktop
    Documents
    Downloads
    lsfile
    Music
    Pictures
    Public
    Templates
    Videos
    \end{lstlisting}
\end{frame}

\begin{frame}
    \frametitle{置き換えをするコマンド}
    \begin{tabular}{ll}
        コマンド & 動作                                                       \\ \hline
        tr       & 入力された文字を指定する方法で置き換えて出力する           \\
        sed      & 入力から指定するパターンを見つけ、それを置き換えて出力する \\ \hline
    \end{tabular}
\end{frame}

\begin{frame}[fragile]
    \frametitle{trコマンドで文字を置き換えよう}
    tr␣置き換えたい文字␣置き換える文字
    \begin{itemize}
        \item 文字列中の特定の文字を別の文字に置き換えるよ
        \item 1文字ごとの置き換えだよ
        \item 「-」は置き換えられないよ
        \item 0-9 や A-Z, a-z のように範囲を指定できるよ
    \end{itemize}
    \begin{lstlisting}[title=範囲指定を使った置き換え, label=tr_range]
    <#green#pi@raspberrypi#>:<#blue#~ $#> echo "HELLO, WORLD!" | tr A-Z a-z
    hello, world!
    \end{lstlisting}
\end{frame}

\begin{frame}[fragile]
    \frametitle{sedコマンドで文字を置き換えよう}
    sed␣'s/置き換え対象の文字列/置き換え後の文字列/g'
    \begin{itemize}
        \item 文字列中の特定の文字列を別の文字列に置き換えるよ
        \item 1文字ごとでなく、文字列で置き換えられるよ
    \end{itemize}
    \begin{lstlisting}[title=sed sedでの文字の置き換え, label=sed_app]
    <#green#pi@raspberrypi#>:<#blue#~ $#> echo "Hello, World!" | sed 's/Hello/Hi/g'
    Hi, World!
    \end{lstlisting}
\end{frame}

\begin{frame}[fragile]
    \frametitle{Bashで計算しよう}
    復習
    \begin{itemize}
        \item echo␣文字 : 文字をそのまま表示する
    \end{itemize}
    \&()
    \begin{itemize}
        \item コマンドの置き換えを行うよ
        \item echoコマンドと組み合わせると計算結果を表示できるよ
    \end{itemize}
    \begin{lstlisting}[title=echo コマンドでの計算, label=cmdsbs:calc]
    <#green#pi@raspberrypi#>:<#blue#~ $#> echo $((138 + 395))
    533
    \end{lstlisting}
        
\end{frame}

\begin{frame}[fragile]
    \frametitle{コマンドに別名を付けよう}
    alias␣名前='コマンド'
    \begin{itemize}
        \item コマンドの別名として名前を付けられるよ
    \end{itemize}
    \begin{itemize}
        \item ターミナルでコマンドを実行するとターミナルが閉じるまで名前の設定は保存されるよ
        \item 別名を保存するには.bash\_aliasesというファイルを作ってコマンドを書いて、sourceコマンドを実行する必要があるよ
    \end{itemize}
    \begin{lstlisting}[title=\textasciitilde/.bash\_aliasesの書き方, label=bashAliasesGrammar1]
    alias 名前='コマンド'
    alias name='command'
                :
                :
    \end{lstlisting}
\end{frame}
