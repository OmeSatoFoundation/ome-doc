\subsection{コマンドを使ってみよう}
\subsubsection{自分がどのディレクトリにいるかしろう}

\begin{description}
\item[pwd]\mbox{}\\
 カレントディレクトリ(自分が今いるディレクトリ)が出てきます。
\end{description}

\begin{lstlisting}[caption=pwdコマンドの例,label=pwdtest]
<#green#pi@raspberrypi#>:<#blue#~ $#> pwd
/home/pi   <-- カレントディレクトリが表示されます
<#green#pi@raspberrypi#>:<#blue#~ $#>
\end{lstlisting}

\begin{tcolorbox}[title=\useOmetoi]

\begin{enumerate}
 \item 実際にpwdを使って、カレントディレクトリが出ることを確かめましょう。出てきたカレントディレクトリを書いてください。\\
\underline{答え.\hspace{0.8\linewidth}}
\end{enumerate}
\end{tcolorbox}

\subsubsection{ディレクトリの中を見てみよう}

\begin{description}
\item[ls■-F■ディレクトリ]\mbox{}\\
ディレクトリの中のファイルやディレクトリが出てきます。ファイルはピンクの文字、ディレクトリは青い文字になっています。
\end{description}

%//terminal[lsF-test][ls -F コマンドの例]{
\begin{lstlisting}[caption=ls -F コマンドの例,label=lsFtest]
<#green#pi@raspberrypi#>:<#blue#~ $#> ls -F
<#blue#MagPi/  ダウンロード/ デスクトップ/  ビデオ/ 画像/
ome/   テンプレート/  ドキュメント/  音楽/   公開/#> <--カレントディレクトリが表示されます
<#green#pi@raspberrypi#>:<#blue#~ $#>
\end{lstlisting}

\begin{itemize}
\item[<例>] ls■-Fだけだとカレントディレクトリの中を見ることができます。 
\item[<例>] Picturesというディレクトリの中を見る場合は ls■-F■Pictures/と打ちます。 
\end{itemize}

ディレクトリの中にあるファイルは人によってちがいます。
\begin{lstlisting}[caption=ls -F Pictures/コマンドの例,label=lsFPicttest]
<#green#pi@raspberrypi#>:<#blue#~ $#> ls -F Pictures
<#magenta#2019-07-08-145604_1366X768_scrot.png  2019-07-08-150326_1366X768_scrot.png  
2019-07-08-150313_1366X768_scrot.png  2019-07-08-150348_1366X768_scrot.png  
2019-07-08-150323_1366X768_scrot.png  2019-07-08-150356_1366X768_scrot.png  #>
<#green#pi@raspberrypi#>:<#blue#~ $#> 
\end{lstlisting}

\begin{tcolorbox}[caption=\useOmetoi]
\begin{enumerate}
\item ls -Fと入力して出てきたファイルとディレクトリの名前を1つずつ書きましょう。\\
\underline{答え.\hspace{0.8\linewidth}}
\item ls -F 画像/ と入力して出てきたファイルかディレクトリの名前を1つ書きましょう。\\
\underline{答え.\hspace{0.8\linewidth}}
\end{enumerate}

%\end{minipage}
\end{tcolorbox}

\subsubsection{ディレクトリへ移動してみよう}
\begin{description}
\item[cd■ディレクトリ]\mbox{}\\
指定したディレクトリへ移動することができます。
\end{description}
\begin{itemize}
\item[<例>] 画像ディレクトリに移動するときはcd 画像/と入力します
\end{itemize}

\begin{lstlisting}[caption=cd directoryの例, label=cdDir]
<#green#pi@raspberrypi#>:<#blue#~ $#> cd 画像/
<#green#pi@raspberrypi#>:<#blue#~/画像 $#>
\end{lstlisting}
文字の左側が変わっています。ここにはカレントディレクトリが書かれています。

\begin{description}
\item[cd]\mbox{}\\
ホームディレクトリに移動できます。
\end{description}
\begin{itemize}
\item[<例>] 画像ディレクトリに移動したあとに、cdとだけ入力するとホームディレクトリに移動できます。
\end{itemize}
\begin{lstlisting}[caption=cdの例, label=cd]
<#green#pi@raspberrypi#>:<#blue#~/画像 $#> cd
<#green#pi@raspberrypi#>:<#blue#~ $#> 
\end{lstlisting}
ディレクトリを入力しなくてもホームディレクトリに移動できました。

\begin{tcolorbox}[title=\useOmetoi]
%\begin{minipage}{0.94\hsize}
\begin{enumerate}
\item cd■Pictures/と入力してPicturesディレクトリに移動してみましょう。\\
\item cdと入力してホームディレクトリに移動してみましょう。\\
\underline{答え.\hspace{0.8\linewidth}}
\end{enumerate}
%\end{minipage}
\end{tcolorbox}

\subsubsection{便利な Tab(タブ)キーを使ってみよう}
入力したい文字の途中まで入力してから、Tab キーを押すと、コンピュータが残りの文字を推測して
くれます。似たような文字があるときは、候補を出してくれます。]
\begin{itemize}
\item[<例>]コマンドを入力する途中で Tab キーを使ってみます。
\end{itemize}
\begin{lstlisting}[caption=Tabの例1, label=Tab1]
<#green#pi@raspberrypi#>:<#blue#~ $#> pw
pwck	pwconv	pwd		pwdx		pwunconv	<--Tabを打つと出てくる
<#green#pi@raspberrypi#>:<#blue#~ $#> pw
\end{lstlisting}
どのコマンドのことかわからないので、コマンドの候補を出してくれました。
\begin{itemize}
\item[<例>]ファイルやディレクトリの名前の途中で Tab キーを使ってみます。
\end{itemize}
\begin{lstlisting}[caption=Tabの例2, label=Tab2]
<#green#pi@raspberrypi#>:<#blue#~ $#> cat /home/pi/o
<#green#pi@raspberrypi#>:<#blue#~ $#> cat /home/pi/ome/	<--Tabを打つと出てくる
\end{lstlisting}
コンピュータが残りの文字を入力してくれました。
\begin{itemize}
\item[<例>]ファイルやディレクトリの名前の途中で Tab キーを使ってみます。
\end{itemize}
\begin{lstlisting}[caption=Tabの例3, label=Tab3]
<#green#pi@raspberrypi#>:<#blue#~ $#> cat /home/pi/ome/0
01/	02/	03/	04/	05/	06/	07/	08/	<--Tabを打つと出てくる
<#green#pi@raspberrypi#>:<#blue#~ $#> cat /home/pi/ome/0
\end{lstlisting}
どのファイルやディレクトリかわからないので、ディレクトリやファイルの候補を出してくれました。

\subsubsection{ファイルの中身を見てみよう}
\begin{description}
\item[cat■ファイル]\mbox{}\\
ファイルに書かれている文字を表示することができます。
\end{description}
\begin{itemize}
\item[<例>]syousetu.txt に書かれている文字を表示します。
\end{itemize}
\begin{lstlisting}[caption=catの例, label=cat]
<#green#pi@raspberrypi#>:<#blue#~ $#> cat /home/pi/ome/03/rensyu/kokugo/syousetu.txt

した。




底本:「新美南吉童話集」岩波文庫、岩波書店
   1996(平成8)年7月16日発行第1刷
   1997(平成9)年7月15日発行第2刷
初出:「赤い鳥 復刊第三巻第一号」
   1932(昭和7)年1月号
※入力時に使われた底本が不明とのことなので、表記は岩波文庫版に合わせた。
入力:林裕司
校正:浜野智
1998年10月23日公開
2012年5月8日修正
青空文庫作成ファイル:
このファイルは、インターネットの図書館、青空文庫(http://www.aozora.gr.jp/)で作られ
ました。入力、校正、制作にあたったのは、ボランティアの皆さんです。<#green#pi@raspberrypi#>:<#blue#~ $#>
\end{lstlisting}

\begin{tcolorbox}[title=\useOmetoi]
%\begin{minipage}{0.94\hsize}
\begin{enumerate}
\item cat■/home/pi/ome/03/rensyu/kokugo/syousetu.txt と入力してみましょう。\\表示された小説の題名はなんでしょうか。\\
\underline{答え.\hspace{0.8\linewidth}}
\end{enumerate}
%\end{minipage}
\end{tcolorbox}

\subsubsection{コピーしてみよう}
\begin{description}
\item[cp■ファイル1■ファイル2]\mbox{}\\
ファイルをコピーすることができます。
\end{description}
\begin{itemize}
\item[<例>]/home/pi/ome/03/rensyu/rika/rika.png をrika2.png という名前でホームディレクトリにコピー
します。ls■-F すると rika2.png がコピーされていることがわかります。
\end{itemize}
\begin{lstlisting}[caption=cpの例, label=cp]
<#green#pi@raspberrypi#>:<#blue#~ $#> cp /home/pi/ome/03/rensyu/rika/rika.png /home/pi/rika2.png
<#green#pi@raspberrypi#>:<#blue#~ $#> ls -F
<#blue#MagPi/	#>	<#magenta#rika2.png#>	<#blue#...#>
\end{lstlisting}
\begin{description}
\item[cp■-r■ディレクトリ1■ディレクトリ2]\mbox{}\\
ディレクトリをコピーすることができます。
\end{description}
\begin{itemize}
\item[<例>]/home/pi/ome/03/rensyu/rika/rika をrikaという名前でホームディレクトリにコピー
します。ls■-F すると rikaディレクトリ がコピーされていることがわかります。
\end{itemize}
\begin{lstlisting}[caption=cp -rの例, label=cp-R]
<#green#pi@raspberrypi#>:<#blue#~ $#> cp -r /home/pi/ome/03/rensyu/rika /home/pi/rika
<#green#pi@raspberrypi#>:<#blue#~ $#> ls -F
<#blue#MagPi/	#>	<#magenta#rika2.png#>	<#blue#...#>
\end{lstlisting}
\begin{tcolorbox}[title=\useOmetoi]
%\begin{minipage}{0.94\hsize}
\begin{enumerate}
\item cp■/home/pi/ome/03/rensyu/rika/rika.png■/home/pi/rika2.png と入力してみましょう。\\ls■-Fと入力してrika2.pngがコピーされているか見てみましょう。\\
\item cp■-r■/home/pi/ome/03/rensyu/rika■/home/pi/rikaと入力してみましょう。\\ls■-Fと入力してrikaディレクトリがコピーされているか見てみましょう。\\
\end{enumerate}
%\end{minipage}
\end{tcolorbox}

\subsubsection{ファイルやディレクトリを移動してみよう}
\begin{description}
\item[mv■ファイルやディレクトリ■移動先]\mbox{}\\
ファイルやディレクトリを移動することができます。
\end{description}
\begin{itemize}
\item[<例>]rika2.pngをrika ディレクトリに移動します。ls■-F■rikaとコマンドを打つとrika2.pngがrikaディレクトリの中に移動されていることがわかります。
\end{itemize}
\begin{lstlisting}[caption=mvの例, label=mv]
<#green#pi@raspberrypi#>:<#blue#~ $#> cp /home/pi/ome/03/rensyu/rika/rika.png /home/pi/rika2.png
<#green#pi@raspberrypi#>:<#blue#~ $#> ls -F
<#blue#MagPi/	#>	<#magenta#rika2.png#>	<#blue#...#>
\end{lstlisting}
\begin{tcolorbox}[title=\useOmetoi]
%\begin{minipage}{0.94\hsize}
\begin{enumerate}
\item cdと入力してホームディレクトリに移動しましょう。\\ mv■rika2.png■rika/ と入力してみましょう。\\ls■-F■rikaと入力してrika2.pngが rika ディレクトリの下に移動しているか見てみましょう。\\
\end{enumerate}
%\end{minipage}
\end{tcolorbox}

\subsubsection{名前を変えてみよう}
\begin{description}
\item[mv■名前■変えたい名前]\mbox{}\\
ファイルやディレクトリの名前を変えることができます。
\end{description}
\begin{itemize}
\item[<例>]rikaディレクトリにあるrika2.pngをbika.pngに名前を変えます。ls■-F■rikaで、名前が変わっているとわかります。
\end{itemize}
\begin{lstlisting}[caption=mvNameの例, label=mvName]
<#green#pi@raspberrypi#>:<#blue#~ $#> cd /home/pi/rika
<#green#pi@raspberrypi#>:<#blue#~ $#> mv rika2.png bika.png
<#green#pi@raspberrypi#>:<#blue#~ $#> ls -F
<#magenta#bika.png	mokei.png		rika.png.#>
\end{lstlisting}
\begin{tcolorbox}[title=\useOmetoi]
%\begin{minipage}{0.94\hsize}
\begin{enumerate}
\item cd■rikaと入力してrikaディレクトリに移動しましょう。\\ mv■rika2.png■bika.png と入力してみましょう。\\ls■-Fと入力してrika2.pngがbika.pngに変わっているか見てみましょう。\\
\end{enumerate}
%\end{minipage}
\end{tcolorbox}

\subsubsection{テキストファイルを作ってみよう}
\begin{description}
\item[leafpad■ファイル]\mbox{}\\
leafpad を使ってファイルを作ることができます。
\end{description}
\begin{itemize}
\item[<例>]my.txt を作り、文字を書いて保存します。
\end{itemize}
\begin{lstlisting}[caption=leafpadの例, label=leafpad]
<#green#pi@raspberrypi#>:<#blue#~ $#> leafpad my.txt &
\end{lstlisting}
\begin{tcolorbox}[title=\useOmetoi]
%\begin{minipage}{0.94\hsize}
\begin{enumerate}
\item leafpad■my.txt■\&と入力してみましょう。リーフパッドが開きます。自分の名前を書いて保存しましょう。\\ターミナルにls■-Fと入力してmy.txtがあるか見てみましょう。\\
\end{enumerate}
%\end{minipage}
\end{tcolorbox}

\subsubsection{ディレクトリを作ってみよう}
\begin{description}
\item[mkdir■ディレクトリ]\mbox{}\\
ディレクトリを作ることができます。
\end{description}
\begin{itemize}
\item[<例>]my ディレクトリを作ります。
\end{itemize}
\begin{lstlisting}[caption=mkdirの例, label=mkdir]
<#green#pi@raspberrypi#>:<#blue#~ $#> mkdir my
<#green#pi@raspberrypi#>:<#blue#~ $#> ls -F
\end{lstlisting}
\begin{tcolorbox}[title=\useOmetoi]
%\begin{minipage}{0.94\hsize}
\begin{enumerate}
\item mkdir■myと入力してみましょう。\\ls■-Fと入力してmyディレクトリがあるか見てみましょう。\\
\end{enumerate}
%\end{minipage}
\end{tcolorbox}

\subsubsection{消してみよう}
ファイルやディレクトリは一度消すと\textbf{元には戻りません}。気を付けましょう。
\begin{description}
\item[rm■ファイル]\mbox{}\\
ファイルを消すことができます。
\end{description}
\begin{itemize}
\item[<例>]my.txt を消します。
\end{itemize}
\begin{lstlisting}[caption=cpの例, label=cp]
<#green#pi@raspberrypi#>:<#blue#~ $#> rm my.txt
<#green#pi@raspberrypi#>:<#blue#~ $#> ls -F
<#blue#MagPi/	#>	<#magenta#rika2.png#>	<#blue#...#>
\end{lstlisting}
\begin{description}
\item[rm■-r■ディレクトリ]\mbox{}\\
ディレクトリを消すことができます。
\end{description}
\begin{itemize}
\item[<例>]my ディレクトリを消します。
\end{itemize}
\begin{lstlisting}[caption=cp -rの例, label=cp-R]
<#green#pi@raspberrypi#>:<#blue#~ $#> rm -r my
<#green#pi@raspberrypi#>:<#blue#~ $#> ls -F
<#blue#MagPi/	#>	<#magenta#rika2.png#>	<#blue#...#>
\end{lstlisting}
\begin{tcolorbox}[title=\useOmetoi]
%\begin{minipage}{0.94\hsize}
\begin{enumerate}
\item rm■my.txtと入力してみましょう。\\ls■-Fと入力してmyディレクトリが消えているか見てみましょう。\\
\item rm■-r■myと入力してみましょう。\\ls■-Fと入力してmyディレクトリが消えているか見てみましょう。\\
\end{enumerate}
%\end{minipage}
\end{tcolorbox}

\subsubsection{ファイルの中身を見てみよう(2)}
\begin{description}
\item[less■ファイル]\mbox{}\\
ファイルに書かれている文字を一画面ずつ見ることができます。
\end{description}
\begin{itemize}
\item[<例>]syousetu.txt の中を見てみます。
\end{itemize}
\begin{lstlisting}[caption=lessの例, label=less]
<#green#pi@raspberrypi#>:<#blue#~ $#> less /home/pi/ome/03/rensyu/kokugo/syousetu.txt
ごん狐
新美南吉


+目次

一

 これは、私わたしが小さいときに、村の茂平もへいというおじいさんからきいたお話です。
 むかしは、私たちの村のちかくの、中山なかやまというところに小さなお城があって、中山
さまというおとのさまが、おられたそうです。 その中山から、少しはなれた山の中に、
「ごん狐ぎつね」という狐がいました。ごんは、一人ひとりぼっちの小狐で、しだの一ぱいしげった
森の中に穴をほって住んでいました。そして、夜でも昼でも、あたりの村へ出てきて、
いたずらばかりしました。はたけへ入って芋をほりちらしたり、菜種なたねがらの、ほしてあるのへ
火をつけたり、百姓家ひゃくしょうやの裏手につるしてあるとんがらしをむしりとって、いったり、
いろんなことをしました。
 或ある秋あきのことでした。二、三日雨がふりつづいたその間あいだ、ごんは、
外へも出られなくて穴の中にしゃがんでいました。
 雨があがると、ごんは、ほっとして穴からはい出ました。空はからっと晴れていて、
百舌鳥もずの声がきんきん、ひびいていました。
 ごんは、村の小川おがわの堤つつみまで出て来ました。あたりの、すすきの穂には、
まだ雨のしずくが光っていました。川は、いつもは水が少すくないのですが、三日もの雨で、
水が、どっとましていました。ただのときは水につかることのない、川べりのすすきや、
萩はぎの株が、黄いろくにごった水に横だおしになって、もまれています。
ごんは川下かわしもの方へと、ぬかるみみちを歩いていきました。
:
\end{lstlisting}
e を押すと一行進みます。\\
y を押すと一行戻ります。\\
q を押すと終わります。\\
\begin{tcolorbox}[title=\useOmetoi]
%\begin{minipage}{0.94\hsize}
\begin{enumerate}
\item less■/home/pi/ome/03/rensyu/kokugo/syousetu.txt と入力してみましょう。\\
\item qを押して終わってみよう。\\
\end{enumerate}
%\end{minipage}
\end{tcolorbox}

\subsubsection{応用:GIF アニメーションの作成}
\label{GIF}
gif アニメーションを使ってスライドショー作ってみましょう。gif とは画像ファイルのフォーマットの 1 つで、アニメーションを表現することもできます。\\
\begin{enumerate}
\item ディレクトリ slideshow を/home/pi に作りましょう。\\
\item スライドショーの素材(パラパラ漫画の一枚一枚)を集めます。インターネットで画像を検索したり、Gimp で作ったりしましょう。画像は今作った slideshow というディレクトリに保存しましょう。\\
\item 画像の名前が連番になるように名前を変更します。00.jpg 01.jpg 02.jpg … のようにスライドショーで表示したい順番の番号を名前にしましょう。\\
\item Gimp を開きます。「ファイル」→「開く/インポート」で「画像ファイルを開く」ウィンドウを出し、最初の一枚目を選択して開きます。Gimp で「ファイル」→「レイヤーとして開く」で、2 番め以降の全てのファイルを選択して開きましょう。Shift キーを押しながらクリックすることで複数選択できます。選択するときは、「名前」をクリックして名前の順に並び替えておきましょう。\\
\item Gimp で「ファイル」→「名前をつけてエクスポート」をクリックします。名前をslideshow.gif とし、エクスポートボタンを押します。
\item 「画像をエクスポート: GIF 形式」というウィンドウが出るので、「アニメーションとしてエクスポート」にチェック、「指定しない場合のディレイ」をお好みに(2000 ミリ秒くらいがオススメ)、「指定しない場合のフレーム処理」を「レイヤーごとに 1 フレーム(置換)」に、「全フレームのディレイにこの値を使用」にチェック、「全フレームのフレーム処理にこの値を使用」にチェックをしてエクスポートボタンを押します。\\
\item 出来上がった画像をみてみましょう。下記どちらかのコマンドでみることができます。\\
chromium-browser slideshow.gif\\
gpicview slideshow.gif\\
\end{enumerate}

\begin{tcolorbox}[title=\useOmetoi]
%\begin{minipage}{0.94\hsize}
\begin{enumerate}
\item \ref{GIF}を見ながら、オリジナルのアニメーションを作成しましょう。
\end{enumerate}
%\end{minipage}
\end{tcolorbox}


\subsubsection{おまけ:こんなメッセージが出たときは}
\begin{lstlisting}[caption=コマンドがちがうときの例, label=cmdMiss]
<#green#pi@raspberrypi#>:<#blue#~ $#> pwb
bash: pwb: コマンドが見つかりません
<#green#pi@raspberrypi#>:<#blue#~ $#> 
\end{lstlisting}
コマンドがちがいます。スペル (アルファベット)があっているか見てみましょう。\\\\

\begin{lstlisting}[caption=ディレクトリやファイルの名前がちがうときの例, label=nameMiss]
<#green#pi@raspberrypi#>:<#blue#~ $#> cd AAA
bash: cd: AAA: そのようなファイルやディレクトリはありません
<#green#pi@raspberrypi#>:<#blue#~ $#> 
\end{lstlisting}
ディレクトリやディレクトリの名前が違います。スペルがあっているか見てみましょう。\\

\subsubsection{豆知識}
コマンドは英文や英単語の省略になっています。\\
pwd: print working directory → ワーキングディレクトリを表示する\\
cd: change directory → ディレクトリを変える\\
cat: concatenate → 結合する\\
mv: move → 移動する\\
mkdir: make directory → ディレクトリを作る\\
rm: remove → 消す\\

\subsubsection{rensyu ディレクトリを最初の状態に戻すには}
/home/pi/ome/03/rensyu ディレクトリはファイルそうさの練習用のディレクトリです。練習することで、内容が変わってしまいます。授業が始まる前の状態に戻すには次のコマンドを使います。rensyu ディレククトリの下の自分で作ったファイルやディレクトリは消えるので注意してください。\\
\begin{enumerate}
\item  cd■/home/pi/ome/03/
\item  tar■--recursive-unlink■-zxvf rensyu.tar.gz
% --表示できない
\end{enumerate}

%まとめ問題


















