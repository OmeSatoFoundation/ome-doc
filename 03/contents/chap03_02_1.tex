\section{ターミナルでファイルとディレクトリを使ってみよう}
\subsection{コマンドについて知ろう}

ターミナルではコマンドを使ってコンピュータとやりとりをします。
コマンドとはコンピュータにあたえる命令のことです。
コマンドは下のようなかたちで書きます。

\begin{description}
\item[コマンド■オプション■\ruby{引数}{ひき|すう}1■\ruby{引数}{ひき|すう}2]\mbox{}\\
 コマンドは動作、引数はそうさのたいしょうです。
 コマンドの最後にEnterキー(エンターキー)を押して、
 コンピュータにコマンドを送ります。
 EnterキーはReturn(リターン)キーと呼ぶこともあります。
\end{description}

使うときは次のことに気を付けましょう。
\begin{itemize}
\item \emph{半角英数字でかくこと}
\item \emph{間にスペース(空白)をいれること}
\end{itemize}

今回はわかりやすいように、スペースが■のように黒くなっています。
次回からは黒くしないのでしっかり覚えましょう。

みなさんが使うラズパイでは英語が使われています。最初から入っているフォルダの名前も英語になっています。読み方を知りたい人は\pageref{英語と日本語の対応表}ページの英語と日本語の対応表を見てみましょう。