
\section{ターミナルを使ってみよう}
\subsection{コマンドについて知ろう}

ターミナルではコマンドを使ってコンピュータとやりとりをします。
コマンドとはコンピュータに\ruby{与}{あた}える命令のことです。
コマンドは下のような\ruby{形}{かたち}で書きます。

\begin{description}
\item[コマンド\textvisiblespace オプション\textvisiblespace \ruby{引数}{ひき|すう}1\textvisiblespace 
\ruby{引数}{ひき|すう}2]\mbox{}\\
コマンドは動作、引数は操作の\ruby{対象}{たい|しょう}です。
 コマンドの最後にEnter(エンター)キーを\ruby{押}{お}して、
 コンピュータにコマンドを送ります。
 EnterキーはReturn(リターン)キーと\ruby{呼}{よ}ぶこともあります。
\end{description}

コマンドは、{\bf 半角英数字}で入力します。また、コマンド、オプション、\ruby{引数}{ひき|すう}の\ruby{間}{あいだ}には{\bf 半角のスペース(空白)}をいれます。教科書ではわかりやすいように、スペースを「\textvisiblespace 」のように示すこともあります。
必ずそうなっているわけではありませんので、どこにスペースを入れるべきかをしっかり自分で覚えましょう。
また、みなさんが使うラズパイでは英語が使われています。最初から入っているフォルダの名前も英語になっています。
読み方を知りたい人は\pageref{英語と日本語の対応表}ページの英語と日本語の\ruby{対応表}{たい|おう|ひょう}を見てみましょう。

\subsection{カレントディレクトリを知ろう}
ディレクトリとは、第1回で習ったフォルダと\ruby{基本的}{き|ほん|てき}に同じものです。
マウスで操作する場合はフォルダ、コマンドで操作する場合はディレクトリと呼ぶことが多いです。
ターミナルで作業をするディレクトリを{\bf カレントディレクトリ}と呼びます。
カレントディレクトリを調べるには以下のコマンドを使用します。

\begin{description}
\item[● \texttt{pwd}]\mbox{}\\
 カレントディレクトリ(ターミナルでの作業ディレクトリ)を表示します。
\end{description}

\begin{lstlisting}[caption=pwdコマンドの例,label=pwdtest]
<#green#pi@raspberrypi#>:<#blue#~ $#> pwd
/home/pi   <-- カレントディレクトリを表示します
<#green#pi@raspberrypi#>:<#blue#~ $#>
\end{lstlisting}

ターミナルがコマンドの入力を待っているときは、\$ (ドル、ダラー)マークが表示されます。
\$ マークの左側には、ユーザー名やコンピュータの名前、カレントディレクトリが表示されています。
\$ マークやその左側の表示は、{\bf プロンプト}と呼ばれます。

\subsection{ディレクトリの中を見てみよう}
\begin{description}
\item[● \texttt{ls}\textvisiblespace \texttt{-F}\textvisiblespace \underline{ファイル}$\ldots$]\mbox{}\\
\underline{ファイル}$\ldots$の中のファイルやディレクトリが表示されます。\underline{ファイル}$\ldots$を指定しないときは、カレントディレクトリのファイルやディレクトリが表示されます。ファイルはピンクの文字、ディレクトリは青い文字で表示されます。Fは大文字なので注意してください。
\end{description}

%//terminal[lsF-test][ls -F コマンドの例]{
\begin{minipage}{\linewidth}
\begin{lstlisting}[caption=ls -F コマンドの例。ファイルやディレクトリが表示されます,label=lsFtest]
<#green#pi@raspberrypi#>:<#blue#~ $#> ls -F
<#blue#01/  03/         Desktop/    Downloads/  Pictures/  Templates/
02/  Bookshelf/  Documents/  Music/      Public/    Videos/#>
<#green#pi@raspberrypi#>:<#blue#~ $#>
\end{lstlisting}
\end{minipage}

\begin{itemize}
\item[<例>] ls\textvisiblespace -Fだけを入力するとカレントディレクトリの中を見ることができます。 
\item[<例>] Picturesというディレクトリの中を見たい場合は ls\textvisiblespace -F\textvisiblespace Pictures/と入力します。 
\end{itemize}

ディレクトリの中にあるファイルは人によってちがいます。
\begin{lstlisting}[caption=ls -F Pictures/コマンドの例,label=lsFPicttest]
<#green#pi@raspberrypi#>:<#blue#~ $#> ls -F Pictures/
<#magenta#2019-07-08-145604_1366X768_scrot.png  2019-07-08-150326_1366X768_scrot.png  
2019-07-08-150313_1366X768_scrot.png  2019-07-08-150348_1366X768_scrot.png  
2019-07-08-150323_1366X768_scrot.png  2019-07-08-150356_1366X768_scrot.png  #>
<#green#pi@raspberrypi#>:<#blue#~ $#> 
\end{lstlisting}

\subsection{便利な Tab(タブ)キーを使ってみよう}
入力したい文字の\ruby{途中}{と|ちゅう}まで入力してから、Tab キーを\ruby{押}{お}すと、コンピュータが残りの文字を\ruby{推測}{すい|そく}して
くれます。似たような文字があるときは、\ruby{候補}{こう|ほ}を出してくれます。
\begin{itemize}
\item[<例>]コマンドを入力する途中で Tab キーを使ってみます。
\end{itemize}
\begin{lstlisting}[caption=Tabの例1, label=Tab1]
<#green#pi@raspberrypi#>:<#blue#~ $#> pw
pwck  pwconv  pwd  pwdx  pwunconv <--Tabを打つと出てくる
<#green#pi@raspberrypi#>:<#blue#~ $#> pw
\end{lstlisting}
どのコマンドのことかわからないので、コマンドの候補を出してくれました。
\begin{itemize}
\item[<例>]ファイルやディレクトリの名前の途中で Tab キーを使ってみます。
\end{itemize}
\begin{lstlisting}[caption=Tabの例2, label=Tab2]
<#green#pi@raspberrypi#>:<#blue#~ $#> ls -F Pi
<#green#pi@raspberrypi#>:<#blue#~ $#> ls -F Pictures	<--Tabを打つと出てくる
\end{lstlisting}
コンピュータが残りの文字を入力してくれました。
\begin{itemize}
\item[<例>]ファイルやディレクトリの名前の途中で Tab キーを使ってみます。
\end{itemize}
\begin{lstlisting}[caption=Tabの例3, label=Tab3]
<#green#pi@raspberrypi#>:<#blue#~ $#> ls -F 0
01/	02/	03/	<--Tabを打つと出てくる
<#green#pi@raspberrypi#>:<#blue#~ $#> ls -F 0
\end{lstlisting}
どのファイルやディレクトリかわからないので、ディレクトリやファイルの候補を出してくれました。

\begin{figure}[h]
\begin{tcolorbox}[title=\useOmetoi]
\begin{enumerate}
  \addquiz{\ruby{実際}{じっ|さい}にpwdを使って、カレントディレクトリが\ruby{表示}{ひょう|じ}されることを確かめましょう。
  表示されたカレントディレクトリを書いてください。}
\addquiz{\texttt{ls}\textvisiblespace \texttt{-F}と入力しましょう。実行して\ruby{表示}{ひょう|じ}されたファイルとディレクトリの名前を1つずつ書きましょう。Fは大文字です。}
\addquiz{Tabキーを使いながら \texttt{ls}\textvisiblespace \texttt{-F}\textvisiblespace Pictures/ と入力しましょう。実行して\ruby{表示}{ひょう|じ}されたファイルかディレクトリの名前を1つ書きましょう。}
\end{enumerate}
\end{tcolorbox}
\end{figure}
