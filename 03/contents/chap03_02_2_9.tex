\subsection{テキストファイルを作ってみよう}
\begin{description}
\item[\texttt{mousepad}\textvisiblespace ファイル\textvisiblespace \&]\mbox{}\\
mousepad を使ってファイルを作ることができます。
\end{description}
\begin{itemize}
\item[<例>]my.txt を作り、文字を書いて\ruby{保存}{ほ|ぞん}します。
\end{itemize}
\begin{lstlisting}[caption=mousepadの例, label=mousepad]
<#green#pi@raspberrypi#>:<#blue#~ $#> mousepad my.txt &
\end{lstlisting}

\begin{itembox}[c]{\& (アンパサンド)の意味について}
    ラズベリーパイのターミナルではコマンドの最後に「\&」を付けると、そのコマンドをターミナルから見えない場所で処理を実行してくれます。このようにターミナルから見えない場所で実行している処理を「バックグラウンドジョブ」と呼びます。一方で、今入力して実行しているコマンドは、「フォアグラウンドジョブ」と呼びます。
\end{itembox}

\begin{tcolorbox}[title=\useOmetoi]
\begin{enumerate}
\addex{ホームディレクトリに移動しましょう。その後、mousepad\textvisiblespace my.txt\textvisiblespace \&と入力してみましょう。マウスパッドが開きます。自分の名前を書いて保存しましょう。\\
ターミナルにls\textvisiblespace -Fと入力してmy.txtがあるか見てみましょう。}
\end{enumerate}
%\end{minipage}
\end{tcolorbox}
