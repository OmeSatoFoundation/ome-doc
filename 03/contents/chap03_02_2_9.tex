\subsection{テキストファイルを作ってみよう}
\begin{description}
\item[leafpad\textvisiblespace ファイル]\mbox{}\\
leafpad を使ってファイルを作ることができます。
\end{description}
\begin{itemize}
\item[<例>]my.txt を作り、文字を書いて保存します。
\end{itemize}
\begin{lstlisting}[caption=leafpadの例, label=leafpad]
<#green#pi@raspberrypi#>:<#blue#~ $#> leafpad my.txt &
\end{lstlisting}
\begin{tcolorbox}[title=\useOmetoi]
\begin{enumerate}
\addex{leafpad\textvisiblespace my.txt\textvisiblespace \&と入力してみましょう。リーフパッドが開きます。自分の名前を書いて保存しましょう。\\
ターミナルにls\textvisiblespace -Fと入力してmy.txtがあるか見てみましょう。}
\end{enumerate}
%\end{minipage}
\end{tcolorbox}
