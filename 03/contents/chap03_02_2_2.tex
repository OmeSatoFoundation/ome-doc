\subsubsection{ディレクトリの中を見てみよう}

\begin{description}
\item[ls■-F■ディレクトリ]\mbox{}\\
ディレクトリの中のファイルやディレクトリが出てきます。ファイルはピンクの文字、ディレクトリは青い文字になっています。
\end{description}

%//terminal[lsF-test][ls -F コマンドの例]{
\begin{lstlisting}[caption=ls -F コマンドの例,label=lsFtest]
<#green#pi@raspberrypi#>:<#blue#~ $#> ls -F
<#blue#MagPi/  ダウンロード/ デスクトップ/  ビデオ/ 画像/
ome/   テンプレート/  ドキュメント/  音楽/   公開/#> <--カレントディレクトリが表示されます
<#green#pi@raspberrypi#>:<#blue#~ $#>
\end{lstlisting}

\begin{itemize}
\item[<例>] ls■-Fだけだとカレントディレクトリの中を見ることができます。 
\item[<例>] Picturesというディレクトリの中を見る場合は ls■-F■Pictures/と打ちます。 
\end{itemize}

ディレクトリの中にあるファイルは人によってちがいます。
\begin{lstlisting}[caption=ls -F Pictures/コマンドの例,label=lsFPicttest]
<#green#pi@raspberrypi#>:<#blue#~ $#> ls -F Pictures
<#magenta#2019-07-08-145604_1366X768_scrot.png  2019-07-08-150326_1366X768_scrot.png  
2019-07-08-150313_1366X768_scrot.png  2019-07-08-150348_1366X768_scrot.png  
2019-07-08-150323_1366X768_scrot.png  2019-07-08-150356_1366X768_scrot.png  #>
<#green#pi@raspberrypi#>:<#blue#~ $#> 
\end{lstlisting}

\begin{tcolorbox}[title=\useOmetoi]
\begin{enumerate}
\item ls■-Fと入力して出てきたファイルとディレクトリの名前を1つずつ書きましょう。\\
\underline{答え.\hspace{0.8\linewidth}}
\item ls■-F■Pictures/ と入力して出てきたファイルかディレクトリの名前を1つ書きましょう。\\
\underline{答え.\hspace{0.8\linewidth}}
\end{enumerate}
\end{tcolorbox}

