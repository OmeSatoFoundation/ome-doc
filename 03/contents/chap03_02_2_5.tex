\newpage
\subsection{ファイルの中身を見てみよう}
\begin{description}
\item[\texttt{cat}\textvisiblespace ファイル]\mbox{}\\
ファイルに書かれている文字を\ruby{表示}{ひょう|じ}することができます。
\end{description}
\begin{itemize}
\item[<例>]syousetu.txt に書かれている文字を表示します。
\end{itemize}
\begin{lstlisting}[caption=catの例, label=cat]
<#green#pi@raspberrypi#>:<#blue#~ $#> cat ~/03/rensyu/kokugo/syousetu.txt

した。




底本:「新美南吉童話集」岩波文庫、岩波書店
   1996(平成8)年7月16日発行第1刷
   1997(平成9)年7月15日発行第2刷
初出:「赤い鳥 復刊第三巻第一号」
   1932(昭和7)年1月号
※入力時に使われた底本が不明とのことなので、表記は岩波文庫版に合わせた。
入力:林裕司
校正:浜野智
1998年10月23日公開
2012年5月8日修正
青空文庫作成ファイル:
このファイルは、インターネットの図書館、青空文庫(http://www.aozora.gr.jp/)で作られ
ました。入力、校正、制作にあたったのは、ボランティアの皆さんです。<#green#pi@raspberrypi#>:<#blue#~ $#>
\end{lstlisting}

\begin{tcolorbox}[title=\useOmetoi]
%\begin{minipage}{0.94\hsize}
\begin{enumerate}
\addquiz{cat\textvisiblespace \textasciitilde /03/rensyu/kokugo/syousetu.txt と入力してみましょう。\\表示された小説の題名はなんでしょうか。}
\end{enumerate}
%\end{minipage}
\end{tcolorbox}
