\newpage
\section{コマンドをつなげて便利に使う}
ラズパイにはコマンドをつなげて便利に使うための方法が用意されています。
コマンドをつなげると、いくつかの処理を連続して行うことができるようになります。
まずは、コマンドが入力・出力をどのように行うのかを理解しましょう。
次に入出力を変更して、コマンドの入力と出力がどのように変わるのかを理解しましょう。
最後に、コマンドをいくつかつなげて、複雑な処理をやってみましょう。

\subsection{コマンドの\ruby{入出力}{にゅう|しゅつ|りょく}}
 コマンドを実行すると、{\bf 標準入力}、{\bf 標準出力}、{\bf 標準エラー出力}の3つのデータの通り道が準備されます。このデータの通り道をチャネルといいます。

 チャネルに何も指定せずにコマンドを実行すると、標準入力はキーボードに設定されます。標準出力と標準エラー出力はディスプレイに設定されます。

コマンドによっては、標準の入力をコマンドの引数として与えることができるものがあります。例えば cat などのコマンドは cat ファイル名 で指定したファイルを標準出力に出力します。cat にファイル名を指定しないと、標準入力、つまりからデータを受け取ります。

cat 
aaaa <enter> <- 標準入力(キーボード)からの入力
aaaa         <- 標準出力(ディスプレイ)への出力


\subsection{リダイレクト}
標準入力、標準出力、標準エラー出力の出力先は、ファイルに変更することができます。
この機能を使うと、ファイルからデータを読み込んだり、結果をファイルに保存したりすることが簡単にできるようになります。

標準出力を出力先は、ファイルに変更することができます。

ls 
ほげ ほげほげ げげげ
ls $\gt$ lsfile
cat lsfile
ほげ
ほげほげ
げげげ

「$\gt$」の後にファイル名を指定することで、
ファイルを作成し、そのファイルにコマンドの結果を保存します。
いつも画面で見ている結果を、リダイレクトでファイルに保存するように変更しました。


標準入力の入力元を、ファイルに変更することができます。
cat < lsfile
ほげ
ほげほげ
げげげ

今回はリダイレクトの練習のため、
catコマンドでファイル名を指定するのではなく、リダイレクトと同じことをしました。

\subsection{パイプライン}

パイプ記号「|」を使うことで、パイプ記号の前のコマンドの標準出力と、
パイプ記号の後のコマンドの標準入力をつなぐことができます。

ls | less

lsした結果が一画面に収まらない場合は、1画面ずつ 表示する。

この方法では1ファイルが一行になるが、それは正しい結果である。


↓はフィルタ。
ラズパイのコマンドは、入力を受け取って、それを加工してから、出力をする。



\subsection{xargsによるコマンドの組み立て}


\section{フィルタコマンド}

\subsection{フィルタコマンド}
\begin{tabular}{ll}
    コマンド & 動作 \\ \hline
    cat & 入力をなにもせずに出力する \\
    tac & 行を逆順に出力する\\
    shuf & 行をランダムに入れ替えて出力する \\
    head & 先頭のいくつかの行を表示する \\
    tail & 末尾のいくつかの行を表示する\\
    sort & 行を順番にならべかえる\\
    grep & 検索パターンに一致する行を出力する\\
\end{tabular}

\subsection{}
\begin{tabular}{ll}
    コマンド & 動作 \\ \hline
    tr & 入力された文字を指定する方法で置き換えて出力する \\
    sed & 入力から指定するパターンを見つけ、それを置き換えて出力する \\
    grep & 検索パターンに一致する行を出力する\\
    du & \\
    echo & \\
    wc & 行数などを出力する\\ \hline
\end{tabular}

\begin{tabular}{ll}
    コマンド & 動作 \\ \hline
    tr & 入力された文字を指定する方法で置き換えて出力する \\
    sed & 入力から指定するパターンを見つけ、それを置き換えて出力する \\ \hline
\end{tabular}


\section{その他の便利なコマンド}

\subsection{文字を置き換える}

trコマンドの説明

実行例の説明

問題

\subsection{単語を置き換える}

sed s/xxx/yyy/g の説明

実行例の説明

問題


\subsection{計算をする}

コマンドの説明: echo \$(式)
\$()で式を囲むと、計算ができる。echoコマンドは引数で与えられた文字列をそのまま標準出力に出力する。計算は整数で行われる。


実行例の説明

問題

\section{コマンドに別名を付ける}

aliasの説明

実行例の説明

.bashrcの簡単な説明。記述を間違えるとラズパイが正しく動作しなくなるので注意。

問題
