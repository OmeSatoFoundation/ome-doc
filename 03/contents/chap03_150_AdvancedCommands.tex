\newpage
\section{コマンドをつなげて\ruby{便利}{べん|り}に使う}
ラズパイにはコマンドをつなげて便利に使うための方法が\ruby{用意}{よう|い}されています。
コマンドをつなげると、いくつかの\ruby{処理}{しょ|り}を連続して行うことができるようになります。
まずは、コマンドが入力・出力をどのように行うのかを\ruby{理解}{り|かい}しましょう。
次に入出力を変更して、コマンドの入力と出力がどのように変わるのかを\ruby{理解}{り|かい}しましょう。
最後に、コマンドをいくつかつなげて、\ruby{複雑}{ふく|ざつ}な\ruby{処理}{しょ|り}をやってみましょう。

\subsection{コマンドの\ruby{入出力}{にゅう|しゅつ|りょく}}
 コマンドを\ruby{実行}{じっ|こう}すると、{\bf \ruby{標準入力}{ひょう|じゅん|にゅう|りょく}}、{\bf \ruby{標準出力}{ひょう|じゅん|しゅつ|りょく}}、{\bf \ruby{標準}{ひょう|じゅん}エラー出力}の3つのデータの\ruby{通}{とお}り道が\ruby{準備}{じゅん|び}されます。このデータの\ruby{通}{とお}り道をチャネルといいます。

 チャネルに何も指定せずにコマンドを\ruby{実行}{じっ|こう}すると、\ruby{標準入力}{ひょう|じゅん|にゅう|りょく}はキーボードに\ruby{設定}{せっ|てい}されます。\ruby{標準出力}{ひょう|じゅん|しゅつ|りょく}と\ruby{標準}{ひょう|じゅん}エラー出力はディスプレイに\ruby{設定}{せっ|てい}されます。

\begin{figure}
    \centering
    \includesvg[width=0.8\linewidth]{images/chap03/std_in_out_err.svg}
    \caption{標準入力・標準出力・標準エラー出力の図}
    \label{ch03:stdioerr}
\end{figure}

コマンドによっては、\ruby{標準}{ひょう|じゅん}の入力をコマンドの引数として与えることができるものがあります。例えば cat などのコマンドは cat ファイル名 で\ruby{指定}{し|てい}したファイルを\ruby{標準出力}{ひょう|じゅん|しゅつ|りょく}に出力します。cat にファイル名を指定しないと、\ruby{標準入力}{ひょう|じゅん|にゅう|りょく}、つまりからデータを受け取ります。

\begin{lstlisting}[caption=catの標準入力・標準出力, label=stdioCat]
<#green#pi@raspberrypi#>:<#blue#~ $#> cat 
sansu <Enter> <- 標準入力(キーボード)からの入力
sansu         <- 標準出力(ディスプレイ)への出力
<Ctrl+D>      <- 標準入力からEOFを入力し、入力が終了したことを伝える
<#green#pi@raspberrypi#>:<#blue#~ $#>
\end{lstlisting}
EOFとはEnd Of File(エンドオブファイル)の\ruby{略称}{りゃく|しょう}でどこがファイルの最後なのかを教えるために使います。


\subsection{リダイレクト}
\ruby{標準入力}{ひょう|じゅん|にゅう|りょく}、\ruby{標準出力}{ひょう|じゅん|しゅつ|りょく}、\ruby{標準}{ひょう|じゅん}エラー出力の出力先は、ファイルに\ruby{変更}{へん|こう}することができます。
この\ruby{機能}{き|のう}を使うと、ファイルからデータを読み込んだり、結果をファイルに保存したりすることが\ruby{簡単}{かん|たん}にできるようになります。

\begin{figure}
    \centering
    \includesvg[width=0.8\linewidth]{images/chap03/redirect.svg}
    \caption{リダイレクトを表した図}
    \label{ch03:redirect}
\end{figure}


\begin{lstlisting}[caption=lsの出力をリダイレクトする, label=redirectLs]
<#green#pi@raspberrypi#>:<#blue#~ $#> ls 
<#blue#01  03         Desktop    Downloads  Pictures  Templates
02  Bookshelf  Documents  Music      Public    Videos#>
<#green#pi@raspberrypi#>:<#blue#~ $#> ls > lsfile
<#green#pi@raspberrypi#>:<#blue#~ $#> cat lsfile
01
02
03
Bookshelf
Desktop
Documents
Downloads
Music
Pictures
Public
Templates
Videos
lsfile
\end{lstlisting}

「>」の後にファイル名を指定することで、
ファイルを作成し、そのファイルにコマンドの結果を保存します。
いつも画面で見ている結果を、リダイレクトでファイルに保存するように\ruby{変更}{へん|こう}しました。


\ruby{標準入力}{ひょう|じゅん|にゅう|りょく}の入力元を、ファイルに\ruby{変更}{へん|こう}することができます。
\begin{lstlisting}[caption=catコマンドにリダイレクトでファイルを入力する, label=redirectCat]
<#green#pi@raspberrypi#>:<#blue#~ $#> cat < lsfile
01
02
03
Bookshelf
Desktop
Documents
Downloads
Music
Pictures
Public
Templates
Videos
lsfile
\end{lstlisting}

今回はリダイレクトの\ruby{練習}{れん|しゅう}のため、
catコマンドでファイル名を\ruby{指定}{し|てい}するのではなく、リダイレクトと同じことをしました。

\subsection{パイプライン}

パイプ\ruby{記号}{き|ごう}「|」を使うことで、パイプ記号の前のコマンドの\ruby{標準出力}{ひょう|じゅん|しゅつ|りょく}と、
パイプ記号の後のコマンドの\ruby{標準入力}{ひょう|じゅん|にゅう|りょく}をつなぐことができます。

\begin{figure}
    \centering
    \includesvg[width=0.8\linewidth]{images/chap03/pipe.svg}
    \caption{パイプを表した図}
    \label{ch03:pipe}
\end{figure}


\begin{lstlisting}[caption=lsコマンドの出力をパイプでlessコマンドに渡す, label=redirectCat]
<#green#pi@raspberrypi#>:<#blue#~ $#> ls | less
01
02
03
Bookshelf
Desktop
Documents
Downloads
Music
Pictures
Public
Templates
Videos
lsfile
\end{lstlisting}

lsコマンドはリダイレクトやパイプを使うと1ファイルが一行ずつ出力されるようになっています。
lessコマンドにlsコマンドの\ruby{標準出力}{ひょう|じゅん|しゅつ|りょく}を渡すことによってファイルやディレクトリを一画面に表示することができました。

\subsection{xargsによるコマンドの組み立て}

\begin{figure}
    \centering
    \includesvg[width=0.8\linewidth]{images/chap03/xargs_command.svg}
    \caption{xargsコマンドを表した図}
    \label{ch03:xargs_command}
\end{figure}

\ruby{標準入力}{ひょう|じゅん|にゅう|りょく}を受け取って、\ruby{標準出力}{ひょう|じゅん|しゅつ|りょく}を出力に出すコマンドをフィルタコマンドといいます。
xargsコマンドもフィルタコマンドのうちの一つです。
xargsコマンドは\ruby{標準入力}{ひょう|じゅん|にゅう|りょく}を受け取って実行したいコマンドの引数として渡すことができます。

\begin{lstlisting}[caption=xargsコマンドを使う準備をする]
<#green#pi@raspberrypi#>:<#blue#~ $#> cd 03/rensyu
<#green#pi@raspberrypi#>:<#blue#~/03/rensyu $#> mkdir xargs
<#green#pi@raspberrypi#>:<#blue#~/03/rensyu $#> cp ~/lsfile ./xargs
<#green#pi@raspberrypi#>:<#blue#~/03/rensyu $#> cp ./kokugo/syousetu.txt ./xargs
<#green#pi@raspberrypi#>:<#blue#~/03/rensyu $#> cd xargs 
<#green#pi@raspberrypi#>:<#blue#~/03/rensyu/xargs $#> ls
<#magenta#lsfile  syousetu.txt#>
\end{lstlisting}
xargsコマンドを使うためのディレクトリを\textasciitilde /03/rensyu/xargsに作りました。

\begin{lstlisting}[caption=xargsコマンドを使ってcatコマンドを使う]
<#green#pi@raspberrypi#>:<#blue#~/03/rensyu/xargs $#> ls | xargs cat
01
02
03
Bookshelf
Desktop
                                           ...
このファイルは、インターネットの図書館、青空文庫(http://www.aozora.gr.jp/)で作られました。
入力、校正、制作にあたったのは、ボランティアの皆さんです。



https://www.aozora.gr.jp/cards/000121/files/628_14895.html
<#green#pi@raspberrypi#>:<#blue#~/03/rensyu/xargs $#>
\end{lstlisting}
lsコマンドで取り出したファイル名をxargsコマンドに渡して、2つのファイルの中身を同時に見ることができました。

\subsection{xargsのオプション}
xargsのオプションの紹介の前にこの章で使うコマンドの説明をします。
\begin{description}
\item[\texttt{seq}\textvisiblespace 数字1\textvisiblespace 数字2]\mbox{}\\
数字1から数字2までの数字を出力します。
\end{description}

\begin{lstlisting}[caption=seqコマンド]
<#green#pi@raspberrypi#>:<#blue#~/03/rensyu/xargs $#> seq 1 5
1
2
3
4
5
<#green#pi@raspberrypi#>:<#blue#~/03/rensyu/xargs $#>
\end{lstlisting}
seqコマンドを使うことで、数字を\ruby{順番}{じゅん|ばん}に出力することができました。

\begin{description}
\item[\texttt{echo}\textvisiblespace 文字]\mbox{}\\
文字をそのまま出力します。
\end{description}

\begin{lstlisting}[caption=seqコマンド]
<#green#pi@raspberrypi#>:<#blue#~/03/rensyu/xargs $#> echo hello
hello
<#green#pi@raspberrypi#>:<#blue#~/03/rensyu/xargs $#>
\end{lstlisting}
echoコマンドを使うことで好きな文字を出力できました。

xargsにもオプションがいくつかあります。そのうち、よくつかわれるオプション3つを\ruby{紹介}{しょう|かい}します。

\begin{description}
\item[\texttt{xargs}\textvisiblespace \texttt{-p}\textvisiblespace コマンド]\mbox{}\\
どのようなコマンドが実行されるかを表示してくれます。
<Enter>を押すと何もせずに終了。<y>を押すとコマンドを実行します。
\end{description}

\begin{lstlisting}[caption=xargsコマンドのオプションp]
<#green#pi@raspberrypi#>:<#blue#~/03/rensyu/xargs $#> ls | xargs -p echo
echo lsfile syousetu.txt?...<Enter>
<#green#pi@raspberrypi#>:<#blue#~/03/rensyu/xargs $#> ls | xargs -p echo
echo lsfile syousetu.txt?...<Y><Enter>
lsfile syousetu.txt
<#green#pi@raspberrypi#>:<#blue#~/03/rensyu/xargs $#>
\end{lstlisting}

\begin{description}
\item[\texttt{xargs}\textvisiblespace \texttt{-i}\textvisiblespace コマンド\textvisiblespace \{\}]\mbox{}\\
\ruby{標準入力}{ひょう|じゅん|にゅう|りょく}を受け取って\{\}の中に1つづつ引数に当てはめます。
\end{description}

\begin{lstlisting}[caption=xargsコマンドのオプションi]
<#green#pi@raspberrypi#>:<#blue#~/03/rensyu/xargs $#> ls | xargs -i echo {}
lsfile
syousetu.txt
<#green#pi@raspberrypi#>:<#blue#~/03/rensyu/xargs $#>
\end{lstlisting}

\begin{description}
\item[\texttt{xargs}\textvisiblespace \texttt{-L}\textvisiblespace 数字\textvisiblespace コマンド]\mbox{}\\
Lオプションは、xarg引数コマンドに一度に\ruby{展開}{てん|かい}されるデータの最大数を指定します。
\ruby{標準出力}{ひょう|じゅん|しゅつ|りょく}から渡されたデータ数が、-Lで指定された数より大きい場合、全てのデータが\ruby{展開}{てん|かい}し終わるまで xargs引数コマンドが繰り返し実行されます。
-iオプションと-Lオプションを一緒に使うことはできません。
\end{description}

\begin{lstlisting}[caption=xargsコマンドのオプションL]
<#green#pi@raspberrypi#>:<#blue#~/03/rensyu/xargs $#> seq 1 5 | xargs -L 1 echo
1
2
3
4
5
<#green#pi@raspberrypi#>:<#blue#~/03/rensyu/xargs $#> seq 1 5 | xargs -L 2 echo
1 2
3 4
5
\end{lstlisting}

\begin{tcolorbox}[title=\useOmetoi]
\begin{enumerate}
    \addex{catと入力して好きな文字を打ち込みましょう。}
    \addex{echo\textvisiblespace 好きな文字\textvisiblespace >\textvisiblespace rensyufile と入力して好きな文字をrensyufileに保存しよう。}
    \addex{リダイレクトの記号「<」を使ってrensyufileの中身を表示してみよう。}
    \addex{xargsを使って、rensyufileとlsfileを合体して表示してみよう。}
\end{enumerate}
\end{tcolorbox}

\section{パイプラインと文字列処理}

\subsection{出力をつくるコマンド}
\begin{tabular}{ll}
    コマンド & 動作 \\ \hline
    ls & ファイルやディレクトリを出力する \\
    du & ディレクトリの中のファイルの大きさを報告する\\
    wc & 入力の文字数・単語数・行数を出力する\\
    echo & 文字をそのまま出力する\\ \hline
\end{tabular}


\subsection{フィルタコマンド}
\begin{tabular}{ll}
    コマンド & 動作 \\ \hline
    cat & 入力をなにもせずに出力する \\
    tac & 行を逆順に出力する\\
    shuf & 行をランダムに入れ替えて出力する \\
    head & 先頭のいくつかの行を表示する \\
    tail & 末尾のいくつかの行を表示する\\
    sort & 行を順番にならべかえる\\
    grep & 検索パターンに一致する行を出力する\\ \hline
\end{tabular}


\subsection{置き換えをするコマンド}
\begin{tabular}{ll}
    コマンド & 動作 \\ \hline
    tr & 入力された文字を指定する方法で置き換えて出力する \\
    sed & 入力から指定するパターンを見つけ、それを置き換えて出力する \\ \hline
\end{tabular}


\section{その他の便利なコマンド}

\subsection{文字の置き換え}
% trコマンドの説明
文字を置き換えるためのコマンドとして、trコマンドを紹介します。
trコマンドを使うと、入力された文字列の中の指定した文字を、別の文字に置き換えて出力することができます。
trコマンドの\ruby{基本的}{き|ほん|てき}な使い方は、次の通りです。

\begin{lstlisting}[caption=trコマンドの基本的な使い方, label=tr_basic_usage]
tr 置き換え対象の文字 置き換え後の文字
\end{lstlisting}

ここで、「置き換え対象の文字」には、置き換えたい文字を指定します。「置き換え後の文字」には、置き換え後の文字を指定します。
例えば、「置き換え対象の文字」に$O_1O_2O_3$、「置き換え後の文字」に$N_1N_2N_3$を指定すると、$O_1$を$N_1$に、$O_2$を$N_2$に、$O_3$を$N_3$に置き換えます。

% 実行例の説明
それでは、trコマンドの使い方を段階的に見ていきましょう。

まず、3文字程度の置き換えから始めてみます。
「HELLO, WORLD!」という文字列があるとします。
この文字列の"H"、"L"、"W"を小文字の"h"、"l"、"w"に置き変えてみましょう。

\begin{lstlisting}[caption=3文字の置き換え, label=tr_3_chars]
<#green#pi@raspberrypi#>:<#blue#~ $#> echo "HELLO, WORLD!" | tr HLW hlw
hEllO, wORlD!
\end{lstlisting}

文字列中の"H"、"L"、"W"が"h"、"l"、"w"に置き換わっていることが分かります。

次に、アルファベット全体を大文字から小文字に置き換えてみましょう。

\begin{lstlisting}[caption=アルファベット全体の置き換え, label=tr_all_chars]
<#green#pi@raspberrypi#>:<#blue#~ $#> echo "HELLO, WORLD!" \
| tr ABCDEFGHIJKLMNOPQRSTUVWXYZ abcdefghijklmnopqrstuvwxyz
hello, world!
\end{lstlisting}

この例では、trコマンドにアルファベットの大文字全てと対応する小文字全てを指定しています。
また、この例で使用している「\textbackslash」は、コマンドを複数行に分けて書くための記号です。
コマンドの実行後、アルファベット全てが小文字に置き換わっていることが分かります。

しかし、アルファベット全てを指定するのは面倒ですよね。
そこで、trコマンドでは、連続するアルファベットや数字などに関して「A-Z」や「a-z」、「0-9」のように範囲指定を使うことで簡潔に表現することができます。

\begin{lstlisting}[caption=範囲指定を使った置き換え, label=tr_range]
<#green#pi@raspberrypi#>:<#blue#~ $#> echo "HELLO, WORLD!" | tr A-Z a-z
hello, world!
\end{lstlisting}

このコマンドでも先程と同じ結果が表示されます。
ちなみに、範囲指定を使うときは以下のような注意点があります。
\begin{itemize}
    \item 「-」は置き換えられない。
    \item 範囲指定は、「A」から「Z」、「a」から「z」のように、前の方の文字から後ろの方の文字へと順番に指定する必要がある。(例: 「Z-A」はエラーになる) 
\end{itemize}

このように、trコマンドを使うと、文字列の中の文字を簡単に置き換えることができます。
また、trコマンドを使うことで文字の削除なども行うことができます。
興味がある人は、文字の置き換え以外にもどのような使い方ができるか、調べてみましょう。

% 問題
それでは、次の問題に挑戦してみましょう。
\begin{enumerate}
    \addex{
        次の文字列の大文字を小文字に変換してみましょう。\\
        "I LOVE PROGRAMMING!"
    }
    \addex{
        次の文字列の小文字を大文字に変換してみましょう。\\
        "i love programming!"
    }
\end{enumerate}

\subsection{単語を置き換える}
tr は一文字ごとの置き換えを処理できました。
それでは、単語など文字列、つまり複数の文字を置き換えたいときはどうしたらよいでしょう。
このために、また別なコマンドがあります。
sed コマンドです。

\begin{lstlisting}[caption=sed コマンドの基本的な使い方, label=sed_usage]
sed 's/置き換え対象の文字列/置き換え後の文字列/g'
\end{lstlisting}

tr コマンドと似たように、「置き換え対象の文字列」には、置き換えたい文字列を指定します。
「置き換え後の文字列」には、置き換え後の文字列を指定します。
tr コマンドと異なり、置き換えは一文字ずつではなく文字列を指定することができます。

再度「Hello, World!」という文章で例を示します。
この文章の「Hello」を「Hi」に置き換えたいときは、次のようにコマンドを入力します。

\begin{lstlisting}[caption=sed コマンドを使った例, label=sed_app]
<#green#pi@raspberrypi#>:<#blue#~ $#> echo "Hello, World!" | sed 's/Hello/Hi/g'
\end{lstlisting}

すると、次のような結果が表示されます。

\begin{lstlisting}[caption=sed コマンドで変換した結果, label=sed_result]
Hi, Wowld!
\end{lstlisting}

それでは、次の問題に取り組んでみましょう。

\begin{enumerate}
    \addex{%
次の文章の「like」を「love」に置き換えましょう。\\%
``I like programming!"%
}
    \addex{%
/home/pi/03/rensyu/kokugo/syosetsu.txt には小説「ごんぎつね」が掲載されています。%
「ごんぎつね」に出てくる「ごんぎつね」の名前を「Fox」に変えて標準出力に出力しましょう。\\%
ヒント: cat /home/pi/03/rensyu/kokugo/syosetsu.txt | sed ???%
}
\end{enumerate}

ここまで単語を置き換える使い方を説明しました。
実は sed というコマンド名は 「stream editor」の略で、まどろっこしい名前をしています
(単語を置き換える機能だけなら、「replace」や「substutite」などの簡単な英単語をコマンド名にすればいいはずです)。
その名前に相応しく、実は sed は非常に多くの機能を持ちます。
興味がある人は

\begin{lstlisting}
<#green#pi@raspberrypi#>:<#blue#~ $#> man sed
\end{lstlisting}

と実行してオンラインマニュアルを閲覧したり、ブラウザで「sed 使い方」などと検索してみたりしましょう。

\subsection{計算をする}

コマンドの説明: echo \$((式))
\$(())で式を囲むと、計算ができます。echoコマンドは引数で与えられた文字列をそのまま標準出力に出力します。
$138 + 395$のように電卓を使って計算したいものを目の前のコンピュータに計算してもらえたら便利ですね。

\begin{lstlisting}[caption=echo コマンドを使った例, label=echo]
echo $((138 + 395))
\end{lstlisting}

実行すると次のような出力が得られます。

\begin{lstlisting}[caption=echo コマンドを実行した結果, label=echo_result]
533
\end{lstlisting}

それでは、次の問題に取り組んでみましょう。

\begin{enumerate}
   \addquiz{$2024 - 727$}
   \addquiz{$117 \times 13$}
   \addquiz{$(84 + 67 + 49 + 96) \div 4$}
\end{enumerate}

\subsection{コマンドに別名を付ける}

コマンドを打っていて、よく使うのに覚えにくいものがあったりします。
そんな時はコマンドに別名を付けてしまいましょう。これを英語ではalias(エイリアス)と言います。

例えば、

\begin{lstlisting}[caption=lsコマンド, label=lsAlias]
    ls -l
\end{lstlisting}

に ll という別名を付けたいとします。

コマンドに別名を付けるためにはにはホームディレクトリに .bash{\_}aliases というファイルを作成し編集しなければなりません。
mousepad を使って .bash{\_}aliases というファイルを作成し、編集します。
\begin{lstlisting}[caption=.bashaliasesを開く, label=openBashAliases]
    <#green#pi@raspberrypi#>:<#blue#~ $#> mousepad ~/.bash_aliases &
\end{lstlisting}

.bash{\_}aliases には、次のように記述し、保存します。
\begin{lstlisting}[caption=.bashaliasesの中身, label=bashAliasesContents]
    alias ll='ls -l'
\end{lstlisting}

保存が終わったら、.bash{\_}aliases を実行して、別名を適用しなければなりません。
.bash{\_}aliases を実行します。
\begin{lstlisting}[caption=.bashaliasesを実行, label=executeBashAliases]
    <#green#pi@raspberrypi#>:<#blue#~ $#> ~/.bash_aliases
\end{lstlisting}

別名を付けたコマンドを打ってみましょう。
\begin{lstlisting}[caption=別名の確認, label=confirmAlias]
    <#green#pi@raspberrypi#>:<#blue#~ $#> ll
\end{lstlisting}

別名を付けた元のコマンドも打ってみましょう。
\begin{lstlisting}[caption=元のコマンドの確認, label=confirmCommand]
    <#green#pi@raspberrypi#>:<#blue#~ $#> ls -l
\end{lstlisting}

同じ出力が出るはずです。

.bash{\_}aliases は正しい文法で記述しないとラズベリーパイが正しく動作しなくなるので注意して編集してください。

問題
