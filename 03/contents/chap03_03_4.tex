\section{ボタンを上手に使ってみよう}
\subsection{ボタンを使って LED をつけてみよう}
赤いボタンを押している間 LED が光る HSP プログラムを考えてみましょう。HSP スクリプトエディタで\textasciitilde /03/button\_led.hsp を読み込み実行してください。センサーボードのボタンのどちらかを押すと LED が 1 つ点灯します。\\

\begin{lstlisting}[caption=button\_led.hsp,label=button_led.hsp]
#include "hsp3dish.as"		<#blue#;スクリプトの設定を読み込む#>
#include "rpz-gpio.as"		<#blue#;スクリプトの設定を読み込む#>

	gpio 17, 0
	gpio 18, 0
	gpio 22, 0
	gpio 27, 0
   
	redraw 0		<#blue#;画面更新(仮想画面に描画)#>
	font "",30		<#blue#;フォントサイズを決める#>
	pos 20,20		<#blue#;文字の位置を決める#>
	mes "ボタンを押している間 LED が光るよ"		<#blue#;表示する文字を決める#>
	redraw 1		<#blue#;画面更新(実際の画面に描画)#>

*led
	btn1 = gpioin(5) 	<#blue#;ボタンの状態を調べる#>
	if btn1==0 : gpio 18, 1 : else : gpio 18, 0 	<#blue#;ボタンによってLEDを消灯/点灯#>
	await 10 		<#blue#;0.01 秒待つ#>
	goto *led 		<#blue#;*led に戻る#>
\end{lstlisting}

ここで出てくる新しい命令は \code{gpioin} 命令です。\code{gpioin()}は\code{()}の中にボタンの番号(赤いボタンは
GPIO5 なので5、黒いボタンは GPIO 6なので6)を書きます。\\

\code{gpioin(5)}で赤いボタンの状態(押されているか、押されていないか)を調べることができます。\\
\code{btn1 = gpioin(5)}で、赤いボタンの状態を \code{btn1} という変数に代入しています。赤いボタンは押されている間、変数の数字が 0、押されていない間は 1 になります。\\

赤いボタンが押されている間、LED をつけたい。そこで、もし赤いボタンが押されていたら(変数の
数字が 0 のとき)LED を光らせる、それ以外のときは消すという命令を書きます。もし~は \code{if} 命令
を書きます。

\code{if btn1==0} で \code{btn1} 変数の数字が 0 のとき(赤いボタンが押されているとき)は\code{ }: の次に書かれている命令を実行します。\\
\code{: gpio 18, 1} で GPIO18 の LED を光らせます。\\
\code{: else : gpio 18, 0} では、else(それ以外)のとき、GPIO18 の LED を消しています。\\

\begin{tcolorbox}[title=\useOmetoi]
\begin{enumerate}
\item ボタンが押されているか調べる命令を書きましょう。\\
\underline{答え.\hspace{0.8\linewidth}}
\item ターミナルを使って button\_led.hsp のコピーを、331.hsp という名前で作りましょう。\\
\fbox{\phantom{白}} $\leftarrow$できたらチェックしましょう。
\item 黒いボタン(GPIO6)を使って LED を光らせるように、331.hsp のプログラムを変えてみましょう。\\
\fbox{\phantom{白}} $\leftarrow$できたらチェックしましょう。
\item  赤いボタン(GPIO5)で LED1(GPIO17)、黒いボタン(GPIO6)で LED3(GPIO22)が光るように、331.hsp のプログラムを変えてみましょう。\\
\fbox{\phantom{白}} $\leftarrow$できたらチェックしましょう。
\end{enumerate}
\end{tcolorbox}