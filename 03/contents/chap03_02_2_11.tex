\subsubsection{消してみよう}
ファイルやディレクトリは一度消すと\textbf{元には戻りません}。気を付けましょう。
\begin{description}
\item[rm■ファイル]\mbox{}\\
ファイルを消すことができます。
\end{description}
\begin{itemize}
\item[<例>]my.txt を消します。
\end{itemize}
\begin{lstlisting}[caption=cpの例, label=cp]
<#green#pi@raspberrypi#>:<#blue#~ $#> ls -F
<#blue#Desktop/	MagPi/	my/	 #>	<#magenta#my.txt#>	<#blue#...#>
<#green#pi@raspberrypi#>:<#blue#~ $#> rm my.txt
<#green#pi@raspberrypi#>:<#blue#~ $#> ls -F
<#blue#Desktop/	MagPi/	my/	 ...#>
\end{lstlisting}
\begin{description}
\item[rm■-r■ディレクトリ]\mbox{}\\
ディレクトリを消すことができます。
\end{description}
\begin{itemize}
\item[<例>]my ディレクトリを消します。
\end{itemize}
\begin{lstlisting}[caption=cp -rの例, label=cp-R]
<#green#pi@raspberrypi#>:<#blue#~ $#> ls -F
<#blue#Desktop/	MagPi/	my/	 ...#>
<#green#pi@raspberrypi#>:<#blue#~ $#> rm -r my
<#green#pi@raspberrypi#>:<#blue#~ $#> ls -F
<#blue#Desktop/	MagPi/	Public/	 ...#>
\end{lstlisting}
\begin{tcolorbox}[title=\useOmetoi]
%\begin{minipage}{0.94\hsize}
\begin{enumerate}
\item rm■my.txtと入力してみましょう。\\ls■-Fと入力してmyディレクトリが消えているか見てみましょう。\\
\item rm■-r■myと入力してみましょう。\\ls■-Fと入力してmyディレクトリが消えているか見てみましょう。\\
\end{enumerate}
%\end{minipage}
\end{tcolorbox}