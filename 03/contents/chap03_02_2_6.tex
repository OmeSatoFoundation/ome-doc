\subsubsection{コピーしてみよう}
\begin{description}
\item[cp■ファイル1■ファイル2]\mbox{}\\
ファイルをコピーすることができます。
\end{description}
\begin{itemize}
\item[<例>]/home/pi/ome/03/rensyu/rika/rika.png をrika2.png という名前でホームディレクトリにコピー
します。ls■-F すると rika2.png がコピーされていることがわかります。
\end{itemize}
\begin{lstlisting}[caption=cpの例, label=cp]
<#green#pi@raspberrypi#>:<#blue#~ $#> cp /home/pi/ome/03/rensyu/rika/rika.png /home/pi/rika2.png
<#green#pi@raspberrypi#>:<#blue#~ $#> ls -F
<#blue#MagPi/	#>	<#magenta#rika2.png#>	<#blue#...#>
\end{lstlisting}
\begin{description}
\item[cp■-r■ディレクトリ1■ディレクトリ2]\mbox{}\\
ディレクトリをコピーすることができます。
\end{description}
\begin{itemize}
\item[<例>]/home/pi/ome/03/rensyu/rika/rika をrikaという名前でホームディレクトリにコピー
します。ls■-F すると rikaディレクトリ がコピーされていることがわかります。
\end{itemize}
\begin{lstlisting}[caption=cp -rの例, label=cp-R]
<#green#pi@raspberrypi#>:<#blue#~ $#> cp -r /home/pi/ome/03/rensyu/rika /home/pi/rika
<#green#pi@raspberrypi#>:<#blue#~ $#> ls -F
<#blue#MagPi/	#>	<#blue#rika#>	<#magenta#rika2.png#>	<#blue#...#>
\end{lstlisting}
\begin{tcolorbox}[title=\useOmetoi]
%\begin{minipage}{0.94\hsize}
\begin{enumerate}
\item cp■/home/pi/ome/03/rensyu/rika/rika.png■/home/pi/rika2.png と入力してみましょう。\\ls■-Fと入力してrika2.pngがコピーされているか見てみましょう。\\□ ← コピーできたらチェックしましょう。
\item cp■-r■/home/pi/ome/03/rensyu/rika■/home/pi/rikaと入力してみましょう。\\ls■-Fと入力してrikaディレクトリがコピーされているか見てみましょう。\\□ ← コピーできたらチェックしましょう。
\end{enumerate}
%\end{minipage}
\end{tcolorbox}