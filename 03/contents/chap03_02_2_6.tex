\subsection{コピーしてみよう}
\begin{description}
\item[\texttt{cp}\textvisiblespace ファイル1\textvisiblespace ファイル2]\mbox{}\\
ファイルをコピーすることができます。
\end{description}
\begin{itemize}
\item[<例>]\textasciitilde /03/rensyu/rika/rika.png をrika2.png という名前でホームディレクトリにコピー
します。ls\textvisiblespace -F を実行すると rika2.png がコピーされていることを\ruby{確認}{かく|にん}できます。
\end{itemize}
\begin{lstlisting}[caption=cpの例, label=cp]
<#green#pi@raspberrypi#>:<#blue#~ $#> cp ~/03/rensyu/rika/rika.png ~/rika2.png
<#green#pi@raspberrypi#>:<#blue#~ $#> ls -F
<#blue#MagPi/	#>	<#magenta#rika2.png#>	<#blue#...#>
\end{lstlisting}
\begin{description}
\item[\texttt{cp}\textvisiblespace \texttt{-r}\textvisiblespace ディレクトリ1\textvisiblespace ディレクトリ2]\mbox{}\\
ディレクトリをコピーします。
\end{description}
\begin{itemize}
\item[<例>]\textasciitilde /03/rensyu/rika/rika をrikaという名前でホームディレクトリにコピー
します。ls\textvisiblespace -F を実行すると rikaディレクトリ がコピーされていることを確認できます。
\end{itemize}
\begin{lstlisting}[caption=cp -rの例, label=cp-R]
<#green#pi@raspberrypi#>:<#blue#~ $#> cp -r ~/03/rensyu/rika ~/rika
<#green#pi@raspberrypi#>:<#blue#~ $#> ls -F
<#blue#MagPi/	#>	<#blue#rika#>	<#magenta#rika2.png#>	<#blue#...#>
\end{lstlisting}
\begin{tcolorbox}[title=\useOmetoi]
%\begin{minipage}{0.94\hsize}
\begin{enumerate}
\addex{cp\textvisiblespace \textasciitilde /03/rensyu/rika/rika.png\textvisiblespace \textasciitilde /rika2.png と入力してみましょう。\\
ls\textvisiblespace -Fと入力してrika2.pngがコピーされているか見てみましょう。}
\addex{cp\textvisiblespace -r\textvisiblespace \textasciitilde /ome/03/rensyu/rika\textvisiblespace \textasciitilde /rikaと入力してみましょう。\\
ls\textvisiblespace -Fと入力してrikaディレクトリがコピーされているか見てみましょう。}
\end{enumerate}
%\end{minipage}
\end{tcolorbox}
