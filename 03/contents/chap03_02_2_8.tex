\subsection{ファイルやディレクトリの名前を変えてみよう}
\begin{description}
\item[\texttt{mv}\textvisiblespace 名前\textvisiblespace 変えたい名前]\mbox{}\\
ファイルやディレクトリの名前を変えることができます。
\end{description}
\begin{itemize}
\item[<例>]rikaディレクトリにあるrika2.pngをbika.pngに名前を変えます。ls\textvisiblespace -F\textvisiblespace rikaで、名前が変わっていることが\ruby{確認}{かく|にん}できます。
\end{itemize}
\begin{lstlisting}[caption=mvNameの例, label=mvName]
<#green#pi@raspberrypi#>:<#blue#~ $#> cd ~/rika
<#green#pi@raspberrypi#>:<#blue#~ $#> mv rika2.png bika.png
<#green#pi@raspberrypi#>:<#blue#~ $#> ls -F
<#magenta#bika.png	mokei.png		rika.png#>
\end{lstlisting}

\begin{tcolorbox}[title=\useOmetoi]
%\begin{minipage}{0.94\hsize}
\begin{enumerate}
\addex{cd\textvisiblespace\textasciitilde /rikaと入力してrikaディレクトリに\ruby{移動}{い|どう}しましょう。\\
mv\textvisiblespace rika2.png\textvisiblespace bika.png と入力してみましょう。\\
ls\textvisiblespace -Fと入力してrika2.pngがbika.pngに変わっているか見てみましょう。}
\end{enumerate}
%\end{minipage}
\end{tcolorbox}
