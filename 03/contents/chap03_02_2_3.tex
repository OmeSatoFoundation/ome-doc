\section{ターミナルでディレクトリやファイルを操作してみよう}

\subsection{ディレクトリへ移動してみよう}
\begin{description}
\item[cd\textvisiblespace ディレクトリ]\mbox{}\\
指定したディレクトリへ移動することができます。
\end{description}
\begin{itemize}
\item[<例>] Picturesディレクトリに移動するときはcd\textvisiblespace Pictures/と入力します
\end{itemize}

\begin{lstlisting}[caption=cd directoryの例, label=cdDir]
<#green#pi@raspberrypi#>:<#blue#~ $#> cd Pictures/
<#green#pi@raspberrypi#>:<#blue#~/Pictures $#>
\end{lstlisting}
文字の左側が変わっています。ここにはカレントディレクトリが書かれています。

\begin{description}
\item[cd]\mbox{}\\
ホームディレクトリに移動できます。
\end{description}
\begin{itemize}
\item[<例>] Picturesディレクトリに移動したあとに、cdとだけ入力するとホームディレクトリに移動できます。
\end{itemize}
\begin{lstlisting}[caption=cdの例, label=cd]
<#green#pi@raspberrypi#>:<#blue#~/Pictures $#> cd
<#green#pi@raspberrypi#>:<#blue#~ $#> 
\end{lstlisting}
ディレクトリを入力しなくてもホームディレクトリに移動できました。\\

\begin{tcolorbox}[title=\useOmetoi]
%\begin{minipage}{0.94\hsize}
\begin{enumerate}
\item cd\textvisiblespace Pictures/と入力してPicturesディレクトリに移動してみましょう。\\
\fbox{\phantom{白}} $\leftarrow$できたらチェックを入れましょう.
\item cdと入力してホームディレクトリに移動してみましょう。\\
\fbox{\phantom{白}} $\leftarrow$できたらチェックを入れましょう.
\end{enumerate}
%\end{minipage}
\end{tcolorbox}