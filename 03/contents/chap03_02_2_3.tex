\newpage
\section{ターミナルでディレクトリやファイルを\ruby{操作}{そう|さ}してみよう}

\subsection{ディレクトリへ\ruby{移動}{い|どう}してみよう}
\begin{description}
\item[\texttt{cd}\textvisiblespace ディレクトリ]\mbox{}\\
指定したディレクトリへ移動することができます。
\end{description}
\begin{itemize}
\item[<例>] Picturesディレクトリに移動するときはcd\textvisiblespace Pictures/と入力します
\end{itemize}

\begin{lstlisting}[caption=cd directoryの例, label=cdDir]
<#green#pi@raspberrypi#>:<#blue#~ $#> cd Pictures/
<#green#pi@raspberrypi#>:<#blue#~/Pictures $#>
\end{lstlisting}
文字の左側が変わっています。ここにはカレントディレクトリが書かれています。

\begin{description}
\item[\texttt{cd}]\mbox{}\\
ホームディレクトリに移動できます。
\end{description}
\begin{itemize}
\item[<例>] Picturesディレクトリに移動したあとに、cdとだけ入力するとホームディレクトリに移動できます。
\end{itemize}
\begin{lstlisting}[caption=cdの例, label=cd]
<#green#pi@raspberrypi#>:<#blue#~/Pictures $#> cd
<#green#pi@raspberrypi#>:<#blue#~ $#> 
\end{lstlisting}
ディレクトリを入力しなくてもホームディレクトリに移動できました。\\

\begin{tcolorbox}[title=\useOmetoi]
\begin{enumerate}
\addex{\code{cd\textvisiblespace Pictures}と入力してPicturesディレクトリに移動し、\code{pwd} コマンドでカレントディレクトリを確認してみましょう。}
\addex{\code{cd}と入力してホームディレクトリに移動し、\code{pwd} コマンドでカレントディレクトリを確認してみましょう。}
\addex{\code{cd ..} と入力して一つ上のディレクトリへ移動し、pwd コマンドでカレントディレクトリを確認してみましょう。}
\addex{\code{cd \textasciitilde} と入力して、pwd コマンドでカレントディレクトリを確認してみましょう。}
\end{enumerate}
\end{tcolorbox}
