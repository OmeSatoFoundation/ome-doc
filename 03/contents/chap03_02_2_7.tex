\subsection{ファイルやディレクトリを移動してみよう}
\begin{description}
\item[mv\textvisiblespace ファイルやディレクトリ\textvisiblespace 移動先]\mbox{}\\
ファイルやディレクトリを移動することができます。
\end{description}
\begin{itemize}
\item[<例>]rika2.pngをrika ディレクトリに移動します。ls\textvisiblespace -F\textvisiblespace rikaとコマンドを実行するとrika2.pngがrikaディレクトリの中に移動されていることを確認できます。
\end{itemize}
\begin{lstlisting}[caption=mvの例, label=mv]
<#green#pi@raspberrypi#>:<#blue#~ $#> mv ~/rika2.png ~/rika
<#green#pi@raspberrypi#>:<#blue#~ $#> ls -F rika
<#magenta#mokei.png#>	<#magenta#rika.png#>	<#magenta#rika2.png#>
\end{lstlisting}

\begin{tcolorbox}[title=\useOmetoi]
%\begin{minipage}{0.94\hsize}
\begin{enumerate}
\addex{\item cdと入力してホームディレクトリに移動しましょう。\\
mv\textvisiblespace rika2.png\textvisiblespace rika/ と入力してみましょう。\\
ls\textvisiblespace -F\textvisiblespace rikaと入力してrika2.pngが rika ディレクトリの下に移動しているか見てみましょう。}
\end{enumerate}
%\end{minipage}
\end{tcolorbox}
