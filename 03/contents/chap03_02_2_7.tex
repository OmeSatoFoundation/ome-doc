\subsubsection{ファイルやディレクトリを移動してみよう}
\begin{description}
\item[mv■ファイルやディレクトリ■移動先]\mbox{}\\
ファイルやディレクトリを移動することができます。
\end{description}
\begin{itemize}
\item[<例>]rika2.pngをrika ディレクトリに移動します。ls■-F■rikaとコマンドを打つとrika2.pngがrikaディレクトリの中に移動されていることがわかります。
\end{itemize}
\begin{lstlisting}[caption=mvの例, label=mv]
<#green#pi@raspberrypi#>:<#blue#~ $#> mv /home/pi/rika2.png /home/pi/rika
<#green#pi@raspberrypi#>:<#blue#~ $#> ls -F rika
<#magenta#mokei.png#>	<#magenta#rika.png#>	<#magenta#rika2.png#>
\end{lstlisting}

\begin{tcolorbox}[title=\useOmetoi]
%\begin{minipage}{0.94\hsize}
\begin{enumerate}
\item cdと入力してホームディレクトリに移動しましょう。\\ mv■rika2.png■rika/ と入力してみましょう。\\ls■-F■rikaと入力してrika2.pngが rika ディレクトリの下に移動しているか見てみましょう。\\□ ← できたらチェックしましょう。
\end{enumerate}
%\end{minipage}
\end{tcolorbox}