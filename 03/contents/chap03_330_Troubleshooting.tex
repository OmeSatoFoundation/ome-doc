\newpage
\section{ターミナルで\ruby{困}{こま}ったときの\ruby{対応}{たい|おう}}

\subsection{エラーメッセージが出たときは}
\begin{lstlisting}[caption=コマンドがちがうときの例, label=cmdMiss]
<#green#pi@raspberrypi#>:<#blue#~ $#> pwb
bash: pwb: コマンドが見つかりません
<#green#pi@raspberrypi#>:<#blue#~ $#> 
\end{lstlisting}
コマンドがちがいます。スペル (アルファベット)があっているか見てみましょう。\\\\

\begin{lstlisting}[caption=ディレクトリやファイルの名前がちがうときの例, label=nameMiss]
<#green#pi@raspberrypi#>:<#blue#~ $#> cd AAA
bash: cd: AAA: そのようなファイルやディレクトリはありません
<#green#pi@raspberrypi#>:<#blue#~ $#> 
\end{lstlisting}
ディレクトリやディレクトリの名前が\ruby{違}{ちが}います。スペルがあっているか見てみましょう。\\

\subsection{キーボードで文字が入力できない}
ターミナルは Ctrl+S を入力すると画面ロック機能が有効になります。画面のロック中は新しい表示がされないため、
キーボードの入力も受け付けていないように見えてしまいます。このような時は Ctrl+Q を入力して、画面のロックを解除してください。

\subsection{コマンドが終了しない}
一部のコマンドは、実行した後ずっと動作し続けるものがあります。このようなコマンドは、自分で\ruby{終了操作}{しゅう|りょう|そう|さ}をしなければなりません。実行中のコマンドを終了させるためには Ctrl+C を入力します。

\subsection{表示される文字がおかしい}
テキストファイルでないファイルを表示させると、そのあとの表示がおかしくなってしまうことがあります。
その際はまず、 Ctrl+L を入力するか、clearコマンドを実行して画面の内容を\ruby{消去}{しょう|きょ}してみてください。
それでもおかしい場合は、reset コマンドを実行しましょう。
resetコマンドでも直らない場合は、ターミナルのウィンドウをいったん閉じて、再度ターミナルを起動しましょう。

\subsection{コマンドをもっと知りたい}
コマンドの詳しい使い方についてい知りたい場合は、以下の方法があります。
\begin{description}
    \item[Web検索] "bash コマンド名"でWeb検索をすると、たいていのコマンドの日本語での説明が見つかります。
    \item[manで調べる] ターミナルに"man コマンド名"と入力すると、英語ですが詳しい使い方が表示されます。直接英語が読めると一番良いのですが、難しければWebのほんやくサービスなどを使って読むとよいでしょう。
\end{description}

\subsection{\ruby{豆知識}{まめ|ち|しき}}
\label{英語と日本語の対応表}
コマンドは英文や英単語の\ruby{省略}{しょう|りゃく}になっています。\\
\begin{figure}[h]
    \begin{tabular}{llp{0.5\columnwidth}}\hline
     コマンド & 対応する英語             & 意味 \\ \hline
     pwd     & print working directory & ワーキングディレクトリを表示する \\
     ls      & list segments           & セグメント(Linuxの祖先でのファイル)の一覧を表示 \\
     cd      & change directory        & ディレクトリを変える\\
     cat     & concatenate             & 結合する\\
     cp      & copy                    & \ruby{複写}{ふく|しゃ}する\\
     mv      & move                    & \ruby{移動}{い|どう}する\\
     less    & more の反対の意味        & more:もっとみせて というコマンドの拡張版にlessという名前を付けた \\
     mkdir   & make directory          & ディレクトリを作る\\
     rm      &  remove                 & 消す\\
     du      & disk usage              & ディスク使用量\\ 
     wc      & word count              & 単語を数える\\ 
     seq     & sequence                & 順番  \\ 
     echo    & echo                    & やまびこ \\ 
     touch   & touch                   & 触る \\ 
     xargs   & extended argments       & 拡張した引数\\ 
     tac     & 対応なし                 & catと逆の処理なのでcatを逆にした\\ 
     shuf    & shuffle                 & ごちゃまぜにする \\ 
     head    & head                    & 頭、先頭\\ 
     tail    & tail                    & しっぽ、末尾\\ 
     sort    & sort                    & 並べ替え、仕分け\\ 
     sed     & stream editor           & ストリームエディタ、編集する決まりを与えてテキストを流すことに由来\\ 
     grep    & g/re/p                  & ラインエディタedのコマンドから。global/regular expression/print\\ 
     tr      & translate               & 変換する\\ 
     alias   & alias                   & 別名 \\ \hline
    \end{tabular}
\end{figure}