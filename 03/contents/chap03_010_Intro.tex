\chapter{温度や\ruby{湿度}{しつ|ど}、明るさを\ruby{測}{はか}ってみよう}
\section{はじめに}
\subsection{この章で学ぶこと}
この章では以下のことを学びます。
\begin{itemize}
    \item コマンドでファイルとディレクトリ扱う
    \item センサーボードをプログラムで動かす
    \item コマンドで高度な処理を行う
\end{itemize}

この章の前半では、コンピュータをコマンドで動かす方法を学びます。
マウスではなく、コマンドでコンピュータを\ruby{操作}{そう|さ}することは、とても重要です。
コマンドを使うと、コンピュータがするべき操作を文字で記録したり、
\ruby{離}{はな}れたコンピュータを\ruby{簡単}{かん|たん}に操作できるようになります。

次に、センサーボードをプログラムから動かす方法を学びます。温湿度気圧センサーの\ruby{情報}{じょう|ほう}を読み取ったり、センサーボードに取り付けてあるLEDを点灯・消灯させたり、ボタンを上手に使ったりして、センサーボードをプログラムから動かすことができるようになりましょう。

この章の後半では、ターミナルを利用して、高度な処理を行う方法を学びます。
ファイルブラウザではできない、テキストの操作や、ファイルの生成などができるようになりましょう。

