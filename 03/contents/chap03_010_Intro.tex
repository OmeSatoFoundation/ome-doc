\chapter{温度や\ruby{湿度}{しつ|ど}、明るさを\ruby{測}{はか}ってみよう}
\section{はじめに}
\subsection{この章で学ぶこと}
この章では以下のことを学びます。

\begin{itemize}
  \item コンピュータをコマンドを使って動かす
  \item ファイルとディレクトリを文字で指定する
  \item センサーボードをプログラムで動かす
  \begin{itemize}
    \item LEDの点灯・消灯
    \item 温湿度\ruby{気圧}{き|あつ}センサーの値を\ruby{読}{よ}む
    \item ボタンを上手に使う
  \end{itemize}
\end{itemize}

まずコンピュータをコマンドで動かす方法を学びます。
マウスではなく、コマンドでコンピュータを\ruby{操作}{そう|さ}することは、とても重要です。
コマンドを使うと、コンピュータがするべき操作を文字で記録したり、
\ruby{離}{はな}れたコンピュータを\ruby{簡単}{かん|たん}に操作できるようになります。
また、コマンドでファイルを\ruby{扱}{あつか}うために、ファイルが置いてある場所を\ruby{指示}{し|じ}する方法を学びます。
ターミナルを利用して、文字を使ってファイルの操作をできるようになりましょう。

次に、センサーボードをプログラムから動かす方法を学びます。
センサーボードに取り付けてあるLEDを点灯・消灯させたり、
温湿度気圧センサーの\ruby{情報}{じょう|ほう}を読み取ったり、ボタンを上手に使ったりして、
センサーボードをプログラムから動かすことができるようになりましょう。

\subsection{教材を自分のフォルダに置こう}
まずは、今回利用する教材をコピーしましょう。
第1回, 第2回と同じように、\nobreak/usr/local/share/ome という場所にあるフォルダ 03 をコピーして、/home/ユーザー名 に\ruby{貼}{は}り付けてください。

やり方を\ruby{忘}{わす}れてしまった人は、「第1回 4.1 例題1-18 教材をじぶんのフォルダに置こう」 を参考にしてみてください。

