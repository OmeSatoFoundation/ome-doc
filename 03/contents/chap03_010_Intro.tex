\chapter{温度や\ruby{湿度}{しつ|ど}、明るさを\ruby{測}{はか}ってみよう}
\section{はじめに}
\subsection{この章で学ぶこと}
この章では以下のことを学びます。

\begin{itemize}
  \item コンピュータをコマンドを使って動かす
  \begin{itemize}
    \item コマンドでファイルとディレクトリ扱う
    \item コマンドでいろいろな処理を行う
  \end{itemize}
  \item センサーボードをプログラムで動かす
  \begin{itemize}
    \item LEDの点灯・消灯
    \item 温湿度\ruby{気圧}{き|あつ}センサーの値を\ruby{読}{よ}む
    \item ボタンを上手に使う
  \end{itemize}
\end{itemize}

この章の前半では、コンピュータをコマンドで動かす方法を学びます。
マウスではなく、コマンドでコンピュータを\ruby{操作}{そう|さ}することは、とても重要です。
コマンドを使うと、コンピュータがするべき操作を文字で記録したり、
\ruby{離}{はな}れたコンピュータを\ruby{簡単}{かん|たん}に操作できるようになります。
まず、コマンドでファイルを\ruby{扱}{あつか}うために、ファイルが置いてある場所を\ruby{指示}{し|じ}する方法を学びます。
次に、ターミナルを利用して、ファイルの操作をする方法を学びます。最後に、コマンドをつなぎ合わせていろいろな処理をする方法を学びます。

この章の後半では、センサーボードをプログラムから動かす方法を学びます。
センサーボードに取り付けてあるLEDを点灯・消灯させたり、
温湿度気圧センサーの\ruby{情報}{じょう|ほう}を読み取ったり、ボタンを上手に使ったりして、
センサーボードをプログラムから動かすことができるようになりましょう。

