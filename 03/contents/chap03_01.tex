\chapter{温度や\ruby{湿度}{しつ|ど}、明るさを\ruby{測}{はか}ってみよう}
\section{はじめに}
\subsection{この章で学ぶこと}
この章では以下のことを学びます。

\begin{itemize}
  \item コンピュータをコマンドを使って動かす
  \item ファイルとディレクトリを文字で指定する
  \item センサーボードをプログラムで動かす
  \begin{itemize}
    \item LEDの点灯・消灯
    \item 温湿度\ruby{気圧}{き|あつ}センサーの値を\ruby{読}{よ}む
    \item ボタンを上手に使う
  \end{itemize}
\end{itemize}

まずコンピュータをコマンドで動かす方法を学びます。
マウスではなく、コマンドでコンピュータを\ruby{操作}{そう|さ}することは、とても重要です。
コマンドを使うと、コンピュータがするべき操作を文字で記録したり、
\ruby{離}{はな}れたコンピュータを\ruby{簡単}{かん|たん}に操作できるようになります。
また、コマンドでファイルを\ruby{扱}{あつか}うために、ファイルが置いてある場所を\ruby{指示}{し|じ}する方法を学びます。
ターミナルを利用して、文字を使ってファイルの操作をできるようになりましょう。

次に、センサーボードをプログラムから動かす方法を学びます。
センサーボードに取り付けてあるLEDを点灯・消灯させたり、
温湿度気圧センサーの\ruby{情報}{じょう|ほう}を読み取ったり、ボタンを上手に使ったりして、
センサーボードをプログラムから動かすことができるようになりましょう。

\subsection{教材を自分のフォルダに置こう}
まずは、今回利用する教材をコピーしましょう。
第1回, 第2回と同じように、\nobreak/usr/local/share/ome という場所にあるフォルダ 03 をコピーして、/home/ユーザー名 に\ruby{貼}{は}り付けてください。

やり方を\ruby{忘}{わす}れてしまった人は、「第1回 4.1 例題1-18 教材をじぶんのフォルダに置こう」 を参考にしてみてください。

\section{ターミナルを使ってみよう}
\subsection{コマンドについて知ろう}

ターミナルではコマンドを使ってコンピュータとやりとりをします。
コマンドとはコンピュータに\ruby{与}{あた}える命令のことです。
コマンドは下のような\ruby{形}{かたち}で書きます。

\begin{description}
\item[コマンド\textvisiblespace オプション\textvisiblespace \ruby{引数}{ひき|すう}1\textvisiblespace 
\ruby{引数}{ひき|すう}2]\mbox{}\\
コマンドは動作、引数は操作の\ruby{対象}{たい|しょう}です。
 コマンドの最後にEnter((エンター)キーを\ruby{押}{お}して、
 コンピュータにコマンドを送ります。
 EnterキーはReturn(リターン)キーと\ruby{呼}{よ}ぶこともあります。
\end{description}

使うときは次のことに気を付けましょう。
\begin{itemize}
\item \emph{半角英数字でかくこと}
\item \emph{\ruby{間}{あいだ}にスペース(空白)をいれること}
\end{itemize}


教科書ではわかりやすいように、スペースを「\textvisiblespace 」のように示すこともあります。
必ずそうなっているわけではありませんので、どこにスペースを入れるべきかをしっかり自分で覚えましょう。
また、みなさんが使うラズパイでは英語が使われています。最初から入っているフォルダの名前も英語になっています。
読み方を知りたい人は\pageref{英語と日本語の対応表}ページの英語と日本語の\ruby{対応表}{たい|おう|ひょう}を見てみましょう。

\subsection{カレントディレクトリを知ろう}
ディレクトリとは、第1回で習ったフォルダと\ruby{基本的}{き|ほん|てき}に同じものです。
マウスで操作する場合はフォルダ、コマンドで操作する場合はディレクトリと呼ぶことが多いです。
ターミナルで作業をするディレクトリを{\bf カレントディレクトリ}と呼びます。
カレントディレクトリを調べるには以下のコマンドを使用します。

\begin{description}
\item[\texttt{pwd}]\mbox{}\\
 カレントディレクトリ(ターミナルでの作業ディレクトリ)を表示します。
\end{description}

\begin{lstlisting}[caption=pwdコマンドの例,label=pwdtest]
<#green#pi@raspberrypi#>:<#blue#~ $#> pwd
/home/pi   <-- カレントディレクトリを表示します
<#green#pi@raspberrypi#>:<#blue#~ $#>
\end{lstlisting}

\subsection{ディレクトリの中を見てみよう}
\begin{description}
\item[\texttt{ls}\textvisiblespace \texttt{-F}\textvisiblespace ディレクトリ]\mbox{}\\
ディレクトリの中のファイルやディレクトリが表示されます。
ファイルはピンクの文字、ディレクトリは青い文字になっています。
Fは大文字なので注意してください。
\end{description}

%//terminal[lsF-test][ls -F コマンドの例]{
\begin{minipage}{\linewidth}
\begin{lstlisting}[caption=ls -F コマンドの例。ファイルやディレクトリが表示されます,label=lsFtest]
<#green#pi@raspberrypi#>:<#blue#~ $#> ls -F
<#blue#01/  03/         Desktop/    Downloads/  Pictures/  Templates/
02/  Bookshelf/  Documents/  Music/      Public/    Videos/#>
<#green#pi@raspberrypi#>:<#blue#~ $#>
\end{lstlisting}
\end{minipage}

\begin{itemize}
\item[<例>] ls\textvisiblespace -Fだけを入力するとカレントディレクトリの中を見ることができます。 
\item[<例>] Picturesというディレクトリの中を見たい場合は ls\textvisiblespace -F\textvisiblespace Pictures/と入力します。 
\end{itemize}

ディレクトリの中にあるファイルは人によってちがいます。
\begin{lstlisting}[caption=ls -F Pictures/コマンドの例,label=lsFPicttest]
<#green#pi@raspberrypi#>:<#blue#~ $#> ls -F Pictures/
<#magenta#2019-07-08-145604_1366X768_scrot.png  2019-07-08-150326_1366X768_scrot.png  
2019-07-08-150313_1366X768_scrot.png  2019-07-08-150348_1366X768_scrot.png  
2019-07-08-150323_1366X768_scrot.png  2019-07-08-150356_1366X768_scrot.png  #>
<#green#pi@raspberrypi#>:<#blue#~ $#> 
\end{lstlisting}

\subsection{便利な Tab(タブ)キーを使ってみよう}
入力したい文字の\ruby{途中}{と|ちゅう}まで入力してから、Tab キーを\ruby{押}{お}すと、コンピュータが残りの文字を\ruby{推測}{すい|そく}して
くれます。似たような文字があるときは、\ruby{候補}{こう|ほ}を出してくれます。
\begin{itemize}
\item[<例>]コマンドを入力する途中で Tab キーを使ってみます。
\end{itemize}
\begin{lstlisting}[caption=Tabの例1, label=Tab1]
<#green#pi@raspberrypi#>:<#blue#~ $#> pw
pwck  pwconv  pwd  pwdx  pwunconv <--Tabを打つと出てくる
<#green#pi@raspberrypi#>:<#blue#~ $#> pw
\end{lstlisting}
どのコマンドのことかわからないので、コマンドの候補を出してくれました。
\begin{itemize}
\item[<例>]ファイルやディレクトリの名前の途中で Tab キーを使ってみます。
\end{itemize}
\begin{lstlisting}[caption=Tabの例2, label=Tab2]
<#green#pi@raspberrypi#>:<#blue#~ $#> ls -F Pi
<#green#pi@raspberrypi#>:<#blue#~ $#> ls -F Pictures	<--Tabを打つと出てくる
\end{lstlisting}
コンピュータが残りの文字を入力してくれました。
\begin{itemize}
\item[<例>]ファイルやディレクトリの名前の途中で Tab キーを使ってみます。
\end{itemize}
\begin{lstlisting}[caption=Tabの例3, label=Tab3]
<#green#pi@raspberrypi#>:<#blue#~ $#> ls -F 0
01/	02/	03/	<--Tabを打つと出てくる
<#green#pi@raspberrypi#>:<#blue#~ $#> ls -F 0
\end{lstlisting}
どのファイルやディレクトリかわからないので、ディレクトリやファイルの候補を出してくれました。

\begin{figure}[b]
\begin{tcolorbox}[title=\useOmetoi]
\begin{enumerate}
  \addquiz{\ruby{実際}{じっ|さい}にpwdを使って、カレントディレクトリが\ruby{表示}{ひょう|じ}されることを確かめましょう。
  表示されたカレントディレクトリを書いてください。}
\addquiz{\texttt{ls}\textvisiblespace \texttt{-F}と入力しましょう。実行して\ruby{表示}{ひょう|じ}されたファイルとディレクトリの名前を1つずつ書きましょう。Fは大文字です。}
\addquiz{Tabキーを使いながら \texttt{ls}\textvisiblespace \texttt{-F}\textvisiblespace Pictures/ と入力しましょう。実行して\ruby{表示}{ひょう|じ}されたファイルかディレクトリの名前を1つ書きましょう。}
\end{enumerate}
\end{tcolorbox}
\end{figure}

\newpage
\section{ファイルとディレクトリ}
\subsection{ファイルってなんだろう?}
ラズパイで\ruby{扱}{あつか}う\ruby{情報}{じょう|ほう}は{\bf ファイル}として扱います。
ファイルはコンピュータに\ruby{保存}{ほ|ぞん}されている文章、\ruby{画像}{が|ぞう}、音楽などのデータです。
多くの場合、人間はコンピュータを使ってファイルを操作することで仕事をします。
ファイルには「\ruby{拡張子}{かく|ちょう|し}」がついており、
そのファイルがどんな種類なのかを表します。
たとえば、画像を表す.jpgや、動画を表す.mp4、
テキストを表す.txtなどがあります。

\begin{table}[H]
  \begin{center}
    \caption[tab:files]{ファイルの種類}
    \begin{tabular}{|c|c|c|} \hline
    \begin{minipage}{0.12\hsize}
      \begin{center}
        \includesvg[width=\linewidth]{images/chap03/oto.svg}
      \end{center}  
    \end{minipage} & 
    \begin{minipage}{0.12\hsize}
      \begin{center}
        \includesvg[width=\linewidth]{images/chap03/image.svg}
      \end{center}
    \end{minipage} &
    \begin{minipage}{0.12\hsize}
      \begin{center}
        \includesvg[width=\linewidth]{images/chap03/douga.svg}
      \end{center} 
    \end{minipage} \\ \hline
    oto.mp3 & gazou.jpg & douga.mp4 \\ \hline
  \end{tabular}
 \end{center}
\end{table}

\subsection{ディレクトリってなんだろう?}
\ruby{扱}{あつか}うファイルが\ruby{増}{ふ}えてくると、ファイルを\ruby{種類}{しゅ|るい}ごとに\ruby{分類}{ぶん|るい}して\ruby{保管}{ほ|かん}すると\ruby{便利}{べん|り}です。
ディレクトリとはファイルを\ruby{分類}{ぶん|るい}・\ruby{整理}{せい|り}するためのもので、
いくつかのファイルを入れておくためのファイルです。
ファイルは紙、ディレクトリはバインダー、だと考えるとわかりやすいでしょう。
以下に、ラズパイの主なディレクトリを示します。

\begin{figure}[H]
  \begin{minipage}{0.3\hsize}
    {\footnotesize 
    \begin{forest}
      for tree={grow'=0,folder}
      [/
        [bin/]
        [dev/]
        [etc/]
        [home/
          [あなたのユーザ名/]]
        [sbin/]
        [tmp/]
        [usr/
          [bin/]
          [lib/]
          [local/]]
        [var/]
      ]
    \end{forest}
    }
  \end{minipage}
  \begin{minipage}{0.6\hsize}
      \begin{description}
      \item [/bin] システムの動作に必要なコマンドの実行ファイルを置くためのディレクトリです。
      \item [/dev] ディスクやキーボードなどのハードウェアを操作するためのファイルを格納します。
      \item [/etc] ラズパイで動作するさまざまなアプリケーションの動作を設定するためのテキストファイルが置かれます。
      \item [/home] ホームディレクトリが配置されます。
      \item [/sbin] shutdownなどの管理者用のコマンドを置くためのディレクトリです。
      \item [/tmp] 一時的な作業ファイル(テンポラリファイル)を置くためのディレクトリです。
      \item [/usr] 追加でアプリケーションをインストールした場合にはこのディレクトリの下に配置されます。
      \item [/var] 変化するデータを置くためのディレクトリです。 
      \end{description}
  \end{minipage}
  \caption{ラズパイの主なディレクトリ}
  \label{fig:raspi-dir}
\end{figure}



\begin{tcolorbox}[title=\useOmetoi]
\begin{enumerate}
\addquiz{ファイルとはなんでしょうか。}
\addquiz{ディレクトリとはなんでしょうか。}
\addex{どんな拡張子があるかインターネットで調べてみましょう。
例を見て、拡張子とその拡張子が何を表すかを書いて表をうめてみましょう。
(ヒント: 教科書第1回 2.7 ブラウザでけんさくしよう を参考にしましょう。)
\begin{table}[H]
  \centering
  \begin{tabular}{p{2em}p{12em}p{12em}} \toprule
番号 & 拡張子 & 表すもの \\ \midrule
1.\vspace{16mm} & .jpg & 画像 \\
2.\vspace{16mm} & .mp4 & 動画 \\
3.\vspace{16mm} & .txt & \hspace{10\zw} \\
4.\vspace{16mm} & \hspace{10\zw} & \hspace{10\zw} \\
5.\vspace{16mm} & \hspace{10\zw} & \hspace{10\zw} \\ \bottomrule
\end{tabular}
\end{table}
%不具合のため全角スペースを挿入
 
}
\end{enumerate}
\end{tcolorbox}

\subsection{ディレクトリの関係}

\noindent {\bf ●ディレクトリの\ruby{上下関係}{じょう|げ|かん|けい}}

いま注目しているディレクトリの中に入っているディレクトリを{\bf 下のディレクトリ}または{\bf サブディレクトリ}と呼びます。
いま注目しているディレクトリを中に持っているディレクトリを{\bf 上のディレクトリ}または{\bf 親ディレクトリ}と呼びます。

\noindent {\bf ●ルートディレクトリ}

一番上のディレクトリは / (スラッシュ)で書きます。特別なディレクトリで、root(ルート)ディレクトリと呼びます。その下にあるbin boot dev etcなどのディレクトリはコンピュータの\ruby{設定}{せっ|てい}ファイルを持っているので、\ruby{間違}{ま|ちが}った使い方をすると\ruby{壊}{こわ}れます。気を付けましょう。

\noindent {\bf ●ホームディレクトリ}

ユーザが作業に使ってもよいディレクトリを{\bf ホームディレクトリ}と言います。
ユーザの作業はホームディレクトリの下で行います。
ホームディレクトリは、ラズベリーパイでは{\bf /home/あなたのユーザ名}です。
ホームディレクトリを示すために記号「\code{\textasciitilde}」(チルダ)を使うこともできます。「\code{\textasciitilde}」(チルダ)と「\code{-}」(ハイフン)は\ruby{慣}{な}れないと\ruby{区別}{く|べつ}が付きにくいので気を付けましょう。

\noindent {\bf ●カレントディレクトリ}

みなさんが今、ターミナルで作業に使っているディレクトリは{\bf カレントディレクトリ}と言います。
カレントディレクトリは「.」と書くことができます。
カレントディレクトリからひとつ上のディレクトリは「..」と書くことができます。

\subsection{\ruby{絶対}{ぜっ|たい}パスと\ruby{相対}{そう|たい}パス}
ディレクトリの位置はパスというディレクトリを / で区切ったもので指定します。
ディレクトリの位置の指定の方法は\ruby{絶対}{ぜっ|たい}パスと\ruby{相対}{そう|たい}パスの2種類があります。

\noindent {\bf ●\ruby{絶対}{ぜっ|たい}パス}

ディレクトリの位置を{\bf ルートディレクトリ(/)から}見てどこにあるかで指定します。
絶対パスの\ruby{特徴}{とく|ちょう}はパスの先頭に / がついています。
例えば、/usr/local/bin などとして指定します。

\noindent {\bf ●\ruby{相対}{そう|たい}パス}

ディレクトリの位置を{\bf カレントディレクトリから}見てどこにあるかで指定します。
相対パスの特徴はパスの先頭に / がついていません。
例えば、./usr/local/bin などとして指定します。

\noindent {\bf ●ホームディレクトリ}

\textasciitilde (チルダ)から始まるディレクトリの位置は、\textasciitilde を
「/home/あなたのユーザ名」と置き\ruby{換}{か}えたものとして扱われます。

\subsection{パスを\ruby{指定}{し|てい}する}
カレントディレクトリにあるディレクトリxxxの位置を指定する場合は、
xxx、xxx/、./xxx、./xxx/などと書きます。いろいろな指定の仕方があるのは面白いですね。

ホームディレクトリにあるファイルyyyの位置を指定する場合は、
\textasciitilde /yyy, または /home/あなたのユーザ名/yyy と書きます。ファイルの後ろにはスラッシュ記号(/)はつけません。

図\ref{fig:folder-path1}, \ref{fig:folder-path2}, \ref{fig:folder-path3} にユーザー名が「pi」のときのディレクトリの関係をパスで\ruby{表}{あらわ}してあります。
パスとフォルダ間の関係を考えながら図を読みましょう。

\begin{figure}[H]
  \begin{minipage}{0.2\hsize}
    \begin{forest}
      for tree={grow'=0,folder}
      [/, draw
        [home/
          [pi/]]]
    \end{forest}
  \end{minipage}
  \begin{minipage}{0.8\hsize}
    \begin{tabular}{ll}
      絶対パス & /からの相対パス \\ \hline
      / & . \\ \hline
      /home または /home/ & \begin{tabular}{l} home または ./home \\ または home/ または ./home/ \end{tabular} \\ \hline 
      /home/pi または /home/pi/ & \begin{tabular}{l} home/pi または ./home/pi \\ または home/pi/ または ./home/pi/ \end{tabular} \\ \hline
    \end{tabular}
  \end{minipage}
  \caption{フォルダの関係と/からのパスの指定}
  \label{fig:folder-path1}
\end{figure}

\begin{figure}[H]
  \begin{minipage}{0.2\hsize}
    \begin{forest}
      for tree={grow'=0,folder}
      [/
        [home/ , draw
          [pi/]]]
    \end{forest}
  \end{minipage}
  \begin{minipage}{0.8\hsize}
    \begin{tabular}{ll}
      絶対パス & /home/からの相対パス \\ \hline
      / & \begin{tabular}{l} .. または ../ \\ または ./.. または ./../ \end{tabular} \\ \hline
      /home または /home/ & \begin{tabular}{l} . \\または ./  \end{tabular} \\ \hline 
      /home/pi または /home/pi/ & \begin{tabular}{l} pi または ./pi \\ または pi/ または ./pi/ \end{tabular} \\ \hline
    \end{tabular}
  \end{minipage}
  \caption{フォルダの関係と/home/からのパスの指定}
  \label{fig:folder-path2}
\end{figure}

\begin{figure}[H]
  \begin{minipage}{0.2\hsize}
    \begin{forest}
      for tree={grow'=0,folder}
      [/
        [home/
          [pi/ , draw ]]]
    \end{forest}
  \end{minipage}
  \begin{minipage}{0.8\hsize}
    \begin{tabular}{ll}
      絶対パス & /home/pi/からの相対パス \\ \hline
      / & \begin{tabular}{l} ../.. または ../../ \\ または ./../.. または ./../../ \end{tabular} \\ \hline
      /home または /home/ & \begin{tabular}{l} .. または ../ \\ または ./.. または ./../ \end{tabular} \\ \hline 
      /home/pi または /home/pi/ & \begin{tabular}{l} . \\ または ./ \end{tabular} \\ \hline
    \end{tabular}
  \end{minipage}
  \caption{フォルダの関係と/home/pi/からのパスの指定}
  \label{fig:folder-path3}
\end{figure}

\begin{tcolorbox}[title=\useOmetoi,breakable]
\begin{figure}[H]
 \centering
%\includesvg[width=0.6\linewidth]{images/chap03/ex3-2.svg}
  \begin{forest}
    for tree={grow'=0,folder}
    [/
      [home/
        [pi/
          [Desktop/
            [Web/]]
          [Pictures/]
          [Video/
            [Eiga/]]
          [Music/]]]]
  \end{forest}
\end{figure}
ユーザー名を「pi」としたときに以下の問題に答えてみましょう。
\begin{enumerate}
\addquiz{ホームディレクトリはどれでしょうか}
\addquiz{カレントディレクトリが/home/pi/Videoのとき、ひとつ上のディレクトリはどれでしょうか。}
\addquiz{カレントディレクトリが/home/piのとき、ひとつ下のディレクトリはどれでしょうか。4つあります。}
\addquiz{カレントディレクトリが/home/pi/Desktopのとき、ひとつ上のディレクトリはどれでしょうか。}
\addquiz{カレントディレクトリが\textasciitilde /Desktopのとき、ひとつ下のディレクトリはどれでしょうか。}
\end{enumerate}
\end{tcolorbox}

