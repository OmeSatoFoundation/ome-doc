\section{チャレンジ問題(1):\ruby{闇夜}{やみ|よ}を照らすLED}
\label{Yamiyo}
これはチャレンジ問題です。レベルアップしたい人は\ruby{挑戦}{ちょう|せん}してみてください。

暗い時に全てのLEDが光るプログラムを書きましょう。\ref{sec:sensors}で明るさを調べる照度センサーを使いました。明るさを調べる命令(教科書\pageref{sec:sensors}ページから)を思い出し、これまでのプログラムを参考にして、新しいプログラムを書いてみましょう。\\
\begin{enumerate}
\renewcommand{\theenumii}{\arabic{enumii}}
\renewcommand{\labelenumii}{\theenumi.\theenumii}
\item HSPスクリプトエディタで\textasciitilde /03/に、light.hspという新しいファイルを作りましょう。
\textcolor{blue}{\item プログラムの\ruby{設定}{せっ|てい}をきめます。}
\begin{enumerate}
\item \#includeを使って”rpz-gpio.as”と”hsp3dish.as”を読み\ruby{込}{こ}みましょう。\\
\pageref{sensors.hsp}ページの1行目と2行目と同じように書きましょう。
\item ts12572に使うI2Cチャンネルを1に設定しましょう。\\
\pageref{sensors.hsp}ページの13行目と同じように書きましょう。
\end{enumerate}
\textcolor{blue}{\item ディスプレイに\ruby{表示}{ひょう|じ}するものをきめます。}
\begin{enumerate}
\item \ruby{仮想}{か|そう}画面に\ruby{描画}{びょう|が}するためにredrawを書きましょう。\\
redraw 0と書きましょう。
\item 文字をディスプレイに表示する命令を書きましょう。好きな文字を表示しましょう。\\
mes命令を使いましょう。
\item \ruby{実際}{じっ|さい}の画面に描画するためのredraw命令を書きましょう\\
redraw 1と書きましょう。
\item 一度、スクリプトを実行してみましょう。ディスプレイに先ほど命令に書いた文字が表示されればOKです。表示されなかったり、エラーが発生したときは、スクリプトを見直したり、先生に\ruby{質問}{しつ|もん}してみてください。
スクリプトを書くときは、少しずつ実行してエラーが出ないことを\ruby{確認}{かく|にん}することがとても大事です。
\end{enumerate}
\textcolor{blue}{\item 明るさを調べ、暗い時にLEDをすべて光らせます}
\begin{enumerate}
\item *mainという”はた”(目印)を使いましょう。\\
\pageref{sensors.hsp}ページの15行目と同じように書きましょう。
\item 明るさを調べる命令を書きましょう。\\
\pageref{sensors.hsp}ページの24行目と25行目と同じように書きましょう。
\item 暗い時を考えてみましょう。暗いときは、明るさが100より小さくなります。
明るさが100より小さい時、LED1,LED2,LED3,LED4が光るという命令を書きましょう。\ruby{条件}{じょう|けん}によって実行する命令を変える場合はifを使います。\\
明るさが100より小さい時という条件は\ 明るさが入っている変数 < 100と書きます。\\
if lux<100 \{\\
\ \ gpio 17,1\\
\ \ gpio 18,1\\
\ \ gpio 22,1\\
\ \ gpio 27,1\\
\}
\item 次に、部屋が暗くないときを考えてみましょう。条件に当てはまらない時に、実行する命令はelse :\{\}の\{\}の中に命令を書きます。\\
else : \{\\
\ \ gpio 17,0\\
\ \ gpio 18,0\\
\ \ gpio 22,0\\
\ \ gpio 27,0\\
\}
\item プログラムの一時停止のためのwait命令を書きましょう。\\
\pageref{sensors.hsp}ページの35行目と同じように書きましょう。
\item *mainまで\ruby{戻}{もど}る\ruby{繰}{く}り返しのgoto命令を書きましょう。\\
\pageref{sensors.hsp}ページの36行目と同じように書きましょう。
\end{enumerate}
\item  \{\}や”などの記号のつけ\ruby{忘}{わす}れがないか確認しましょう。
スクリプトを実行してみましょう。ディスプレイに「暗いとLEDが光るよ」と表示されればOKです。表示されなかったり、エラーが発生したときは、スクリプトを見直したり、TAや先生に質問してみてください。
\end{enumerate}

\begin{tcolorbox}[title=\useOmetoi]
    %\begin{minipage}{0.94\hsize}
    \begin{enumerate}
    \addex{\ref{Yamiyo}を見ながら、プログラムを完成させましょう。}
    \end{enumerate}
    %\end{minipage}
\end{tcolorbox}

