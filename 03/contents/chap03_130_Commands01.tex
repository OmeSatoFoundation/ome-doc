\newpage
\section{ターミナルでファイルを\ruby{操作}{そう|さ}してみよう(1)}

\subsection{カレントディレクトリを\ruby{変更}{へん|こう}してみよう}
\begin{description}
\item[● \texttt{cd}\textvisiblespace \underline{ディレクトリ}]\mbox{}\\
カレントディレクトリを\underline{ディレクトリ}に変更します。このことを「移動する」と言います。
\end{description}
\begin{itemize}
\item[<例>] Picturesディレクトリに移動するときはcd\textvisiblespace Pictures/と入力します
\end{itemize}

\begin{lstlisting}[caption=cd directoryの例, label=cdDir]
<#green#pi@raspberrypi#>:<#blue#~ $#> cd Pictures/
<#green#pi@raspberrypi#>:<#blue#~/Pictures $#>
\end{lstlisting}
プロンプトのカレントディレクトリ\ruby{表示}{ひょう|じ}が変わっています。

\begin{description}
\item[● \texttt{cd}]\mbox{}\\
ホームディレクトリに移動します。
\end{description}
\begin{itemize}
\item[<例>] Picturesディレクトリに移動したあとに、cdとだけ入力するとホームディレクトリに移動できます。
\end{itemize}
\begin{lstlisting}[caption=cdの例, label=cd]
<#green#pi@raspberrypi#>:<#blue#~/Pictures $#> cd
<#green#pi@raspberrypi#>:<#blue#~ $#> 
\end{lstlisting}
ディレクトリを入力しなくてもホームディレクトリに移動できました。\\


\subsection{ファイルの中身を見てみよう}
\begin{description}
\item[● \texttt{cat}\textvisiblespace \underline{ファイル}$\ldots$ ]\mbox{}\\
\underline{ファイル}$\ldots$に書かれている\ruby{内容}{ない|よう}を\ruby{表示}{ひょう|じ}します。
\end{description}
\begin{itemize}
\item[<例>]syousetu.txt に書かれている文字を表示します。
\end{itemize}
\begin{lstlisting}[caption=catの例, label=cat]
<#green#pi@raspberrypi#>:<#blue#~ $#> cat ~/03/rensyu/kokugo/syousetu.txt

した。




底本:「新美南吉童話集」岩波文庫、岩波書店
   1996(平成8)年7月16日発行第1刷
   1997(平成9)年7月15日発行第2刷
初出:「赤い鳥 復刊第三巻第一号」
   1932(昭和7)年1月号
※入力時に使われた底本が不明とのことなので、表記は岩波文庫版に合わせた。
入力:林裕司
校正:浜野智
1998年10月23日公開
2012年5月8日修正
青空文庫作成ファイル:
このファイルは、インターネットの図書館、青空文庫(http://www.aozora.gr.jp/)で作られ
ました。入力、校正、制作にあたったのは、ボランティアの皆さんです。<#green#pi@raspberrypi#>:<#blue#~ $#>
\end{lstlisting}



\subsection{コピーしてみよう}
\begin{description}
\item[● \texttt{cp}\textvisiblespace \underline{ファイル1}\textvisiblespace \underline{ファイル2}]\mbox{}\\
ファイル1をファイル2という名前でコピーします。
\end{description}
\begin{itemize}
\item[<例>]\textasciitilde /03/rensyu/rika/rika.png をrika2.png という名前でホームディレクトリにコピー
します。ls\textvisiblespace -F を実行すると rika2.png がコピーされていることを\ruby{確認}{かく|にん}できます。
\end{itemize}
\begin{lstlisting}[caption=cpの例, label=cp]
<#green#pi@raspberrypi#>:<#blue#~ $#> cp ~/03/rensyu/rika/rika.png ~/rika2.png
<#green#pi@raspberrypi#>:<#blue#~ $#> ls -F
<#blue#MagPi/	#>	<#magenta#rika2.png#>	<#blue#...#>
\end{lstlisting}
\begin{description}
\item[\texttt{cp}\textvisiblespace \texttt{-r}\textvisiblespace \underline{ディレクトリ1}\textvisiblespace \underline{ディレクトリ2}]\mbox{}\\
\underline{ディレクトリ1}を\underline{ディレクトリ2}という名前でコピーします。ディレクトリのなかみもすべてコピーします。
\end{description}
\begin{itemize}
\item[<例>]\textasciitilde /03/rensyu/rika をrikaという名前でホームディレクトリにコピーします。ls\textvisiblespace -F を実行すると rikaディレクトリ がコピーされていることを確認できます。
\end{itemize}
\begin{lstlisting}[caption=cp -rの例, label=cp-R]
<#green#pi@raspberrypi#>:<#blue#~ $#> cp -r ~/03/rensyu/rika ~/rika
<#green#pi@raspberrypi#>:<#blue#~ $#> ls -F
<#blue#MagPi/	#>	<#blue#rika/#>	<#magenta#rika2.png#>	<#blue#...#>
\end{lstlisting}

\subsection{ファイルやディレクトリを\ruby{移動}{い|どう}してみよう}
\begin{description}
\item[● \texttt{mv}\textvisiblespace \underline{ファイルやディレクトリ}$\ldots$\textvisiblespace \underline{移動先ディレクトリ}]\mbox{}\\
\underline{ファイルやディレクトリ}$\ldots$を\underline{移動先ディレクトリ}に移動します。
\end{description}
\begin{itemize}
\item[<例>]rika2.pngをrika ディレクトリに移動します。ls\textvisiblespace -F\textvisiblespace rikaとコマンドを実行するとrika2.pngがrikaディレクトリの中に移動されていることを\ruby{確認}{かく|にん}できます。
\end{itemize}
\begin{lstlisting}[caption=mvの例, label=mv]
<#green#pi@raspberrypi#>:<#blue#~ $#> mv ~/rika2.png ~/rika
<#green#pi@raspberrypi#>:<#blue#~ $#> ls -F rika
<#magenta#mokei.png#>	<#magenta#rika.png#>	<#magenta#rika2.png#>
\end{lstlisting}


\subsection{ファイルやディレクトリの名前を変えてみよう}
\begin{description}
\item[● \texttt{mv}\textvisiblespace \underline{名前}\textvisiblespace \underline{変えたい名前}]\mbox{}\\
ファイルやディレクトリの\underline{名前}を\underline{変えたい名前}に変更します。
\end{description}
\begin{itemize}
\item[<例>]rikaディレクトリにあるrika2.pngをbika.pngに名前を変えます。ls\textvisiblespace -F\textvisiblespace rikaで、名前が変わっていることが\ruby{確認}{かく|にん}できます。
\end{itemize}
\begin{lstlisting}[caption=mvNameの例, label=mvName]
<#green#pi@raspberrypi#>:<#blue#~ $#> cd ~/rika
<#green#pi@raspberrypi#>:<#blue#~ $#> mv rika2.png bika.png
<#green#pi@raspberrypi#>:<#blue#~ $#> ls -F
<#magenta#bika.png	mokei.png		rika.png#>
\end{lstlisting}

\begin{tcolorbox}[title=\useOmetoi]
    \begin{enumerate}
        \addex{\code{cd\textvisiblespace Pictures}と入力してPicturesディレクトリに移動し、\code{pwd} コマンドでカレントディレクトリを確認してみましょう。}
        \addex{\code{cd}と入力してホームディレクトリに移動し、\code{pwd} コマンドでカレントディレクトリを確認してみましょう。}
        \addex{\code{cd ..} と入力して一つ上のディレクトリへ移動し、pwd コマンドでカレントディレクトリを確認してみましょう。}
        \addex{\code{cd \textasciitilde} と入力して、pwd コマンドでカレントディレクトリを確認してみましょう。}\\ \\%調整用の空行に注意
        \addquiz{cat\textvisiblespace \textasciitilde /03/rensyu/kokugo/syousetu.txt と入力してみましょう。\\表示された小説の題名はなんでしょうか。}
    \end{enumerate}
\end{tcolorbox}

\begin{tcolorbox}[title=\useOmetoi]
    %\begin{minipage}{0.94\hsize}
    \begin{enumerate}
    \addex{ホームディレクトリに移動しましょう。}
    \addex{cp\textvisiblespace \textasciitilde /03/rensyu/rika/rika.png\textvisiblespace \textasciitilde /rika2.png と入力してみましょう。\\
    ls\textvisiblespace -Fと入力してrika2.pngがコピーされているか見てみましょう。}
    \addex{cp\textvisiblespace -r\textvisiblespace \textasciitilde /03/rensyu/rika\textvisiblespace \textasciitilde /rikaと入力してみましょう。\\
    ls\textvisiblespace -Fと入力してrikaディレクトリがコピーされているか見てみましょう。}
    \addex{cdと入力してホームディレクトリに移動しましょう。\\
    mv\textvisiblespace rika2.png\textvisiblespace rika/ と入力してみましょう。\\
    ls\textvisiblespace -F\textvisiblespace rikaと入力してrika2.pngが rika ディレクトリの下に移動しているか見てみましょう。}
    \addex{cd\textvisiblespace\textasciitilde /rikaと入力してrikaディレクトリに\ruby{移動}{い|どう}しましょう。\\
    mv\textvisiblespace rika2.png\textvisiblespace bika.png と入力してみましょう。\\
    ls\textvisiblespace -Fと入力してrika2.pngがbika.pngに変わっているか見てみましょう。}
    \end{enumerate}
    %\end{minipage}
\end{tcolorbox}
