\subsection{便利な Tab(タブ)キーを使ってみよう}
入力したい文字の\ruby{途中}{と|ちゅう}まで入力してから、Tab キーを\ruby{押}{お}すと、コンピュータが残りの文字を\ruby{推測}{すい|そく}して
くれます。似たような文字があるときは、\ruby{候補}{こう|ほ}を出してくれます。
\begin{itemize}
\item[<例>]コマンドを入力する途中で Tab キーを使ってみます。
\end{itemize}
\begin{lstlisting}[caption=Tabの例1, label=Tab1]
<#green#pi@raspberrypi#>:<#blue#~ $#> pw
pwck  pwconv  pwd  pwdx  pwunconv <--Tabを打つと出てくる
<#green#pi@raspberrypi#>:<#blue#~ $#> pw
\end{lstlisting}
どのコマンドのことかわからないので、コマンドの候補を出してくれました。
\begin{itemize}
\item[<例>]ファイルやディレクトリの名前の途中で Tab キーを使ってみます。
\end{itemize}
\begin{lstlisting}[caption=Tabの例2, label=Tab2]
<#green#pi@raspberrypi#>:<#blue#~ $#> cat ~/P
<#green#pi@raspberrypi#>:<#blue#~ $#> cat ~/Pictures	<--Tabを打つと出てくる
\end{lstlisting}
コンピュータが残りの文字を入力してくれました。
\begin{itemize}
\item[<例>]ファイルやディレクトリの名前の途中で Tab キーを使ってみます。
\end{itemize}
\begin{lstlisting}[caption=Tabの例3, label=Tab3]
<#green#pi@raspberrypi#>:<#blue#~ $#> cat /home/pi/0
01/	02/	03/	<--Tabを打つと出てくる
<#green#pi@raspberrypi#>:<#blue#~ $#> cat /home/pi/ome/0
\end{lstlisting}
どのファイルやディレクトリかわからないので、ディレクトリやファイルの候補を出してくれました。