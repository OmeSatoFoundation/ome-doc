\subsubsection{便利な Tab(タブ)キーを使ってみよう}
入力したい文字の途中まで入力してから、Tab キーを押すと、コンピュータが残りの文字を推測して
くれます。似たような文字があるときは、候補を出してくれます。]
\begin{itemize}
\item[<例>]コマンドを入力する途中で Tab キーを使ってみます。
\end{itemize}
\begin{lstlisting}[caption=Tabの例1, label=Tab1]
<#green#pi@raspberrypi#>:<#blue#~ $#> pw
pwck	pwconv	pwd		pwdx		pwunconv	<--Tabを打つと出てくる
<#green#pi@raspberrypi#>:<#blue#~ $#> pw
\end{lstlisting}
どのコマンドのことかわからないので、コマンドの候補を出してくれました。
\begin{itemize}
\item[<例>]ファイルやディレクトリの名前の途中で Tab キーを使ってみます。
\end{itemize}
\begin{lstlisting}[caption=Tabの例2, label=Tab2]
<#green#pi@raspberrypi#>:<#blue#~ $#> cat /home/pi/o
<#green#pi@raspberrypi#>:<#blue#~ $#> cat /home/pi/ome/	<--Tabを打つと出てくる
\end{lstlisting}
コンピュータが残りの文字を入力してくれました。
\begin{itemize}
\item[<例>]ファイルやディレクトリの名前の途中で Tab キーを使ってみます。
\end{itemize}
\begin{lstlisting}[caption=Tabの例3, label=Tab3]
<#green#pi@raspberrypi#>:<#blue#~ $#> cat /home/pi/ome/0
01/	02/	03/	04/	05/	06/	07/	08/	<--Tabを打つと出てくる
<#green#pi@raspberrypi#>:<#blue#~ $#> cat /home/pi/ome/0
\end{lstlisting}
どのファイルやディレクトリかわからないので、ディレクトリやファイルの候補を出してくれました。