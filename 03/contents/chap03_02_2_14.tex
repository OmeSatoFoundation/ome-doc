\section{ターミナルで\ruby{困}{こま}ったときの\ruby{対応}{たい|おう}とまとめの問題}

\subsection{エラーメッセージが出たときは}
\begin{lstlisting}[caption=コマンドがちがうときの例, label=cmdMiss]
<#green#pi@raspberrypi#>:<#blue#~ $#> pwb
bash: pwb: コマンドが見つかりません
<#green#pi@raspberrypi#>:<#blue#~ $#> 
\end{lstlisting}
コマンドがちがいます。スペル (アルファベット)があっているか見てみましょう。\\\\

\begin{lstlisting}[caption=ディレクトリやファイルの名前がちがうときの例, label=nameMiss]
<#green#pi@raspberrypi#>:<#blue#~ $#> cd AAA
bash: cd: AAA: そのようなファイルやディレクトリはありません
<#green#pi@raspberrypi#>:<#blue#~ $#> 
\end{lstlisting}
ディレクトリやディレクトリの名前が\ruby{違}{ちが}います。スペルがあっているか見てみましょう。\\


\subsection{rensyu ディレクトリを最初の\ruby{状態}{じょう|たい}に\ruby{戻}{もど}すには}
\textasciitilde /03/rensyu ディレクトリはファイルそうさの練習用のディレクトリです。練習することで、\ruby{内容}{ない|よう}が変わってしまいます。\ruby{授業}{じゅ|ぎょう}が始まる前の状態に戻すには次のコマンドを使います。rensyu ディレククトリの下の自分で作ったファイルやディレクトリは消えてしまうので注意してください。\\
\begin{enumerate}
\item  cd\textvisiblespace \textasciitilde /03
\item  tar\textvisiblespace \verb|--|recursive-unlink\textvisiblespace -zxvf\textvisiblespace rensyu.tar.gz
% --表示できない
\end{enumerate}

\subsection{\ruby{豆知識}{まめ|ち|しき}}
\label{英語と日本語の対応表}
コマンドは英文や英単語の\ruby{省略}{しょう|りゃく}になっています。\\
pwd: print working directory → ワーキングディレクトリを表示する\\
cd: change directory → ディレクトリを変える\\
cat: concatenate → 結合する\\
mv: move → \ruby{移動}{い|どう}する\\
mkdir: make directory → ディレクトリを作る\\
rm: remove → 消す\\
