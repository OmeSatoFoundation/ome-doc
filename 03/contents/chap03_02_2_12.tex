\subsection{ファイルの中身を見てみよう(2)}
\begin{description}
\item[less\textvisiblespace ファイル]\mbox{}\\
ファイルに書かれている文字を一画面ずつ見ることができます。
\end{description}
\begin{itemize}
\item[<例>]syousetu.txt の中を見てみます。
\end{itemize}
\begin{lstlisting}[caption=lessの例, label=less]
<#green#pi@raspberrypi#>:<#blue#~ $#> less ~/03/rensyu/kokugo/syousetu.txt
ごん狐
新美南吉


+目次

一

 これは、私わたしが小さいときに、村の茂平もへいというおじいさんからきいたお話です。
 むかしは、私たちの村のちかくの、中山なかやまというところに小さなお城があって、中山
さまというおとのさまが、おられたそうです。 その中山から、少しはなれた山の中に、
「ごん狐ぎつね」という狐がいました。ごんは、一人ひとりぼっちの小狐で、しだの一ぱいしげった
森の中に穴をほって住んでいました。そして、夜でも昼でも、あたりの村へ出てきて、
いたずらばかりしました。はたけへ入って芋をほりちらしたり、菜種なたねがらの、ほしてあるのへ
火をつけたり、百姓家ひゃくしょうやの裏手につるしてあるとんがらしをむしりとって、いったり、
いろんなことをしました。
 或ある秋あきのことでした。二、三日雨がふりつづいたその間あいだ、ごんは、
外へも出られなくて穴の中にしゃがんでいました。
 雨があがると、ごんは、ほっとして穴からはい出ました。空はからっと晴れていて、
百舌鳥もずの声がきんきん、ひびいていました。
 ごんは、村の小川おがわの堤つつみまで出て来ました。あたりの、すすきの穂には、
まだ雨のしずくが光っていました。川は、いつもは水が少すくないのですが、三日もの雨で、
水が、どっとましていました。ただのときは水につかることのない、川べりのすすきや、
萩はぎの株が、黄いろくにごった水に横だおしになって、もまれています。
ごんは川下かわしもの方へと、ぬかるみみちを歩いていきました。
:
\end{lstlisting}
e を押すと一行進みます。\\
y を押すと一行戻ります。\\
q を押すと終わります。\\
\begin{tcolorbox}[title=\useOmetoi]
%\begin{minipage}{0.94\hsize}
\begin{enumerate}
\item less\textvisiblespace \textasciitilde /03/rensyu/kokugo/syousetu.txt と入力してみましょう。\\
\item qを押して終わってみよう。\\
\fbox{\phantom{白}} $\leftarrow$できたらチェックしましょう。
\end{enumerate}
%\end{minipage}
\end{tcolorbox}