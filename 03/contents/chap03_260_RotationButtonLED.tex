\subsection{ボタンを\ruby{押}{お}して光るLEDを変えてみよう}
赤いボタンを押すと点くLEDが次々に変わるHSPプログラムを考えてみましょう。
HSPスクリプトエディタで\textasciitilde /03/button\_led3.hspを読み\ruby{込}{こ}み実行してください。

\begin{lstlisting}[escapechar=!,caption=button\_led3.hsp,label=button_led3.hsp]
#include "hsp3dish.as"		<#blue#;スクリプトの設定を読み込む#>
#include "rpz-gpio.as"		<#blue#;スクリプトの設定を読み込む#>

 gpio 17, 0
 gpio 18, 0
 gpio 22, 0
 gpio 27, 0
	
  redraw 0			<#blue#;画面更新(仮想画面に描画)#>
  font "",30			<#blue#;フォントサイズを決める#>
  pos 20,20			<#blue#;文字の位置を決める#>
  mes "赤いボタンで光るLEDを変えることができるよ"	<#blue#;表示する文字を決める#>
  redraw 1			<#blue#;画面更新(実際の画面に描画)#>

  ima_btn = 1
  ledidx = 0

*hata
  ima_btn = gpioin(5)
  if ima_btn = 0{
    ledidx = (ledidx+1) !\textbackslash ! 4	  <#blue#;ledidxで何番目(0~3)のLEDを光らせるか決めています。#>
    if ledidx=1 {		   <#blue#;l番目のときは、GPIO17のLEDを光らせます。#>
      gpio 17, 1
      gpio 18, 0
      gpio 22, 0
      gpio 27, 0
    }
    else : if ledidx=2 {	   <#blue#;2番目のときは、GPIO18のLEDを光らせます。#>
      gpio 17, 0
      gpio 18, 1
      gpio 22, 0
      gpio 27, 0
    }
    else : if ledidx=3 {	   <#blue#;3番目のときは、GPIO22のLEDを光らせます。#>
      gpio 17, 0
      gpio 18, 0
      gpio 22, 1
      gpio 27, 0
    }
    else : if ledidx=0 {	   <#blue#;0番目のときは、GPIO27のLEDを光らせます。#>
      gpio 17, 0
      gpio 18, 0
      gpio 22, 0
      gpio 27, 1
    }
  }

  wait 10
  goto *hata		<#blue#;*hataまで戻る#>
\end{lstlisting}

\noindent \textbf{\ruby{剰余算}{じょう|よ|ざん}}\\
今回、新たに\textbackslash (バックスラッシュ)という記号が登場しています。これは剰余算といって\ruby{割}{わ}り算をしたときの余りを計算できます。例えば、5 \textbackslash \ 4 は5を4で割った余りである1になります。

\code{ledidx = (ledidx+1) \textbackslash \ 4}では、何回\code{*hata}まで戻っているか数えています。LEDは4つあるので、4回動けば一周できます。一周したらまた0に戻りたいので、剰余算を使っています。今回は\code{ledidx}は0~3のどれかの数字になります。実際に計算すると、次のようになります。\\
\\
{\small 
\begin{tabular}{llll}
ledidxが0のとき & \code{ledidx=(0+1) \textbackslash \ 4} を計算すると& $1 \div 4=$0あまり1 なので& 次のledidxは1になる\\
ledidxが1のとき & \code{ledidx=(1+1) \textbackslash \ 4} を計算すると& $2 \div 4=$0あまり2 なので& 次のledidxは2になる\\
ledidxが2のとき & \code{ledidx=(2+1) \textbackslash \ 4} を計算すると& $3 \div 4=$0あまり3 なので& 次のledidxは3になる\\
ledidxが3のとき & \code{ledidx=(3+1) \textbackslash \ 4} を計算すると& $4 \div 4=$1あまり0 なので& 次のledidxは0になる
\end{tabular}
}
\\
このように剰余算をうまくつかうことで、4つのLEDの切り替えを実現しています。
\\

\noindent {\bf if \ruby{条件文}{じょう|けん|ぶん}1 \{実行文1\} else : if 条件文2\{実行文2\}}

この文は条件文1が成り立つときは実行文1を実行します。成り立たないときは、条件文2が成り立つかどうか調べます。条件文2が成り立つときは実行文2を実行します。 

しかし、このままだとボタンを押すたびにLEDが右どなりに動いているようには見えないですね?それはコンピュータが計算するのが速すぎるせいです。ボタンを一回押したら、ひとつ右どなりに動くようにするにはどうしたらよいでしょうか。\pageref{button_led2_toi}ページの問題の「変化を\ruby{検出}{けん|しゅつ}する」を参考に問題をといてみましょう。\\

\begin{tcolorbox}[title=\useOmetoi]
\begin{enumerate}
\addex{ターミナルを使ってbutton\_led3.hspのコピーを、333.hspという名前で作りましょう。}
\addex{赤いボタン(GPIO5)を一回押すとLEDが一\ruby{個}{こ}右となりに動くように、333.hspのプログラムを変えてみましょう。}
\addex{赤いボタン(GPIO5)を押すとLEDが左に\ruby{移動}{い|どう}するように、333.hspのプログラムを変えてみましょう。}
\end{enumerate}
\end{tcolorbox}
