\newpage
\section{ターミナルで\ruby{困}{こま}ったときの\ruby{対応}{たい|おう}}

\subsection{エラーメッセージが出たときは}
\begin{lstlisting}[caption=コマンドがちがうときの例, label=cmdMiss]
<#green#pi@raspberrypi#>:<#blue#~ $#> pwb
bash: pwb: コマンドが見つかりません
<#green#pi@raspberrypi#>:<#blue#~ $#> 
\end{lstlisting}
コマンドがちがいます。スペル (アルファベット)があっているか見てみましょう。\\\\

\begin{lstlisting}[caption=ディレクトリやファイルの名前がちがうときの例, label=nameMiss]
<#green#pi@raspberrypi#>:<#blue#~ $#> cd AAA
bash: cd: AAA: そのようなファイルやディレクトリはありません
<#green#pi@raspberrypi#>:<#blue#~ $#> 
\end{lstlisting}
ディレクトリやディレクトリの名前が\ruby{違}{ちが}います。スペルがあっているか見てみましょう。\\

\subsection{キーボードで文字が入力できない}
ターミナルは Ctrl+S を入力すると画面ロック機能が有効になります。画面のロック中は新しい表示がされないため、
キーボードの入力も受け付けていないように見えてしまいます。このような時は Ctrl+Q を入力して、画面のロックを解除してください。

\subsection{コマンドが終了しない}
一部のコマンドは、実行した後ずっと動作し続けるものがあります。このようなコマンドは、自分で\ruby{終了操作}{しゅう|りょう|そう|さ}をしなければなりません。実行中のコマンドを終了させるためには Ctrl+C を入力します。

\subsection{表示される文字がおかしい}
テキストファイルでないファイルを表示させると、そのあとの表示がおかしくなってしまうことがあります。
その際はまず、 Ctrl+L を入力するか、clearコマンドを実行して画面の内容を\ruby{消去}{しょう|きょ}してみてください。
それでもおかしい場合は、reset コマンドを実行しましょう。
resetコマンドでも直らない場合は、ターミナルのウィンドウをいったん閉じて、再度ターミナルを起動しましょう。


\subsection{\ruby{豆知識}{まめ|ち|しき}}
\label{英語と日本語の対応表}
コマンドは英文や英単語の\ruby{省略}{しょう|りゃく}になっています。\\
pwd: print working directory → ワーキングディレクトリを表示する\\
cd: change directory → ディレクトリを変える\\
cat: concatenate → 結合する\\
mv: move → \ruby{移動}{い|どう}する\\
mkdir: make directory → ディレクトリを作る\\
rm: remove → 消す\\
