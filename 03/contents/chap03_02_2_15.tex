% ※いい感じに分割してください※



\begin{tcolorbox}[title=\useOmetoi,breakable]
\begin{figure}[H]
    \centering
	\begin{forest}
		for tree={grow'=0,folder}
		[/
			[home/
				[あなたのユーザ名/
					[03/ 
						[rensyu/
							[kokugo/
								[syousetu.txt]]
							[sansu/
								[nihon.png]
								[sansu.txt]]
							[rika/
								[mokei.png]
								[rika.png]]
							[syakai/
								[sekai.png]]
						]
					]
				]
			]
		]
	\end{forest}
    %\includesvg[width=0.6\linewidth]{images/chap03/ex3-15.svg}
\end{figure}
この図はこの章で使うファイルの配置を\ruby{示}{しめ}した図です。
ターミナルを開いて、コマンドを使って問題をといてみましょう。\\
\begin{enumerate}
\addquiz{ホームディレクトリに移動しましょう。}
\addex{ホームディレクトリに置かれているファイルやディレクトリを見てみましょう。}
\addex{\textasciitilde /03/rensyuに\ruby{移動}{い|どう}してみましょう。}
\addquiz{rensyuディレクトリの中身を調べてみましょう。どんなディレクトリがありますか?}
\addex{一番上からのディレクトリからsyousetu.txtファイルの中身を見てみましょう。}
\addex{カレントディレクトリからsyousetu.txtファイルの中身を見てみましょう。}
\addex{kokugoディレクトリに移動してから、syousetu.txtファイルの中身を見てみましょう。}
\addquiz{\vspace{.5\zh} 問題5, 6, 7で見たファイルの\ruby{内容}{ない|よう}はぜんぶ同じでしたか?}
\addex{rikaディレクトリの下にあるmokei.pngをコピーしてrikaディレクトリの中にmokei2.pngを作りましょう。}
\addex{rikaディレクトリをコピーして、rensyuディレクトリの下にrika2ディレクトリを作りましょう。}
\addex{rensyuディレクトリの下にあるkanji.txtをkokugoディレクトリの下に移動しましょう。}
\addex{sansuディレクトリの下にあるnihon.pngをsyakaiディレクトリの下に移動しましょう。}
\addex{syakaiディレクトリの下に移動しましょう。その後、syakaiディレクトリの下にrekisiディレクトリを作りましょう。}
\addex{rensyuディレクトリの下にあるoda.jpgをsyakai/rekisiディレクトリに移動しましょう。}
\addex{syakaiディレクトリの下にあるrekisiディレクトリをrensyuディレクトリの下に移動しましょう。}
\addex{sansu/sansu.txtの名前をsansu/tasizan.txtに変えてみましょう。}
\item syousetu.txtを一画面ずつ表示してみましょう。\vspace{.5\zh}
\begin{enumerate}
	\addquiz{catと何が\ruby{違}{ちが}うでしょうか。}
	\item 一行進めてみましょう。
	\item 一行\ruby{戻}{もど}ってみましょう。
	\item 終わりましょう。
	\end{enumerate} \vspace{0.5\zh}
\fbox{\phantom{白}} $\leftarrow$できたらチェックしましょう。\\
\addex{rikaディレクトリの下にある、mokei2.pngを消してみましょう。}
\addex{rika2ディレクトリを消してみましょう。}
\addex{rensyuディレクトリの下にmy.txtファイルを作り、名前と好きな食べ物を書いてみましょう。書いたら\ruby{保存}{ほ|ぞん}しましょう。}
\addex{rensyuディレクトリの下にMyディレクトリを作り、my.txtをMyディレクトリの下に移動してみましょう。}
\end{enumerate}
\end{tcolorbox}
