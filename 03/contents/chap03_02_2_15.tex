% ※いい感じに分割してください※

\begin{tcolorbox}[title=\useOmetoi,breakable]
\begin{figure}[H]
    \centering
    \includesvg[width=0.6\linewidth]{images/chap03/ex3-15.svg}
\end{figure}
この図はユーザー名が「pi」のときのファイルの配置を\ruby{示}{しめ}した図です。
ターミナルを開いて、コマンドを使って問題をといてみましょう。\\
\begin{enumerate}
\addquiz{カレントディレクトリはどこか調べてみましょう。}
\addex{カレントディレクトリの中身を見てみましょう。}
\addex{\textasciitilde /03/rensyuに\ruby{移動}{い|どう}してみましょう。}
\addquiz{rensyuディレクトリの中身を調べてみましょう。どんなディレクトリがありますか?}
\addex{一番上からのディレクトリからsyousetu.txtファイルの中身を見てみましょう。}
\addex{カレントディレクトリからsyousetu.txtファイルの中身を見てみましょう。}
\addex{kokugoディレクトリに移動してから、syousetu.txtファイルの中身を見てみましょう。}
\addquiz{\vspace{.5\zh}\ruby{内容}{ない|よう}はぜんぶ同じでしたか?}
\addex{rikaディレクトリの下にあるmokei.pngをコピーしてrikaディレクトリの中にmokei2.pngを作りましょう。}
\addex{rikaディレクトリをコピーして、rensyuディレクトリの下にrika2ディレクトリを作りましょう。}
\addex{rensyuディレクトリの下にあるkanji.txtをkokugoディレクトリの下に移動しましょう。}
\addex{sansuディレクトリの下にあるnihon.pngをsyakaiディレクトリの下に移動しましょう。}
\addex{rekisiディレクトリを作り、syakaiディレクトリの下に移動しましょう。}
\addex{rensyuディレクトリの下にあるoda.jpgをsyakai/rekisiディレクトリに移動しましょう。}
\addex{syakaiディレクトリの下にあるrekisiディレクトリをrensyuディレクトリの下に移動しましょう。}
\addex{sansu/sansu.txtの名前をsansu/tasizan.txtに変えてみましょう。}
\item syousetu.txtを一画面ずつ表示してみましょう。\vspace{.5\zh}
\begin{enumerate}
	\addquiz{catと何が\ruby{違}{ちが}うでしょうか。}
	\item 一行進めてみましょう。
	\item 一行\ruby{戻}{もど}ってみましょう。
	\item 終わりましょう。
	\end{enumerate} \vspace{0.5\zh}
\fbox{\phantom{白}} $\leftarrow$できたらチェックしましょう。\\
\addex{rikaディレクトリの下にある、mokei2.pngを消してみましょう。}
\addex{rika2ディレクトリを消してみましょう。}
\addex{rensyuディレクトリの下にmy.txtファイルを作り、名前と好きな食べ物を書いてみましょう。書いたら\ruby{保存}{ほ|ぞん}しましょう。}
\addex{rensyuディレクトリの下にMyディレクトリを作り、my.txtをMyディレクトリの下に移動してみましょう。}
\end{enumerate}
\end{tcolorbox}
