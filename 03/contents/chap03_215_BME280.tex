\newpage
\section{温度、\ruby{湿度}{しつ|ど}、\ruby{気圧}{き|あつ}、明るさを調べてみよう}\label{sec:sensors}
センサーボードに\ruby{繋}{つな}がっている BME280(温湿度気圧センサー)と、TSL2572(照度センサー)を使ってみましょう。HSP スクリプトエディタで\textasciitilde /03/sensors.hsp を読み\ruby{込}{こ}み実行してください。\\

\begin{lstlisting}[caption=\textasciitilde/03/sensors.hsp,label=sensors.hsp]
#include "hsp3dish.as"		<#blue#;スクリプトの設定を読み込む#>
#include "rpz-gpio.as"		<#blue#;スクリプトの設定を読み込む#>

init_bme			<#blue#; 温湿度気圧センサ bme280 を初期化する#>
if stat { 			<#blue#; 初期化成功チェック#>
 redraw 0
 mes "failed to init bme: " + stat
 redraw 1
 stop
}

geti2c_lux_init 		<#blue#; 照度センサ tsl2572 を初期化する#>

*main
	redraw 0 		<#blue#; 仮想画面に描画#>
	pos 60, 60 		<#blue#; 表示位置を設定#>
	get_temp		<#blue#; rpz_temp に温度取得#>
	temp = rpz_temp
	get_humidity		<#blue#; rpz_hum に湿度取得#>
	hum = rpz_hum
	get_pressure		<#blue#; rpz_press に気圧取得#>
	press = rpz_press
	geti2c_lux_als		<#blue#; rpz_lux に照度取得#>
	lux = rpz_lux

	mes "温度: " + temp + " [ ]" ℃		<#blue#; ↓取得したデータの表示#>
	mes "湿度: " + hum + " [%]"
	mes "気圧: " + press+ " [hPa]"
	mes "照度: " + lux + " [lx]"
	redraw 1		<#blue#; 実際の画面に描画#>
	wait 100
	goto *main
\end{lstlisting}

まずは BME280 に関する行について説明します。\\
\code{init\_bme}\\
という命令で BME280 を初期化しています。この命令の\ruby{実行結果}{じっ|こう|けっ|か}が変数 stat に代入されています。初期化に失敗すると success に 0 以外の値が入るので、次の if 文で初期化が成功したかどうかを\ruby{確認}{かく|にん}しています。\\
\code{init\_bme} によって、\code{get\_temp}, \code{get\_humidity}, \code{get\_pressure} が使えるようになりました。\\
\code{get\_temp}	; rpz\_temp に温度取得\\
\code{get\_humidity}	; rpz\_hum に湿度取得\\
\code{get\_pressure}	; rpz\_press に気圧取得\\
のようにすることによって、\code{rpz\_temp} に気温 (temperature)、 \code{rpz\_hum} に湿度(humidity)、\code{rpz\_press} に気圧(pressure)が入ります。

次は TSL2572 に関係する行を説明します。\\
\code{geti2c\_lux\_init}
によって TSL2572 が初期化されます。BME280 のときと\ruby{違}{ちが}ってエラーチェックをする必要はありません。この初期化によって、\code{geti2c\_lux\_als} が使えるようになります。 \\
\code{geti2c\_lux\_als} ; rpz\_lux に照度取得
とすることで、rpz\_lux に照度が入ります。

TSL2572から得られた値を変数 \code{temp}, \code{hum}, \code{press}, \code{lux} にいれていきます。\\
\code{temp = rpz\_temp\\
hum = rpz\_hum\\
press = rpz\_press\\
lux = rpz\_lux\\}

そしてこれらの変数 \code{temp}, \code{hum}, \code{press}, \code{lux} を \code{mes} 命令を使って画面に表示しています。\\

\begin{tcolorbox}[title=\useOmetoi]
\begin{enumerate}
\addquiz{温湿度気圧センサ BME280 を初期化する命令を書きましょう。}
\addquiz{照度センサ TSL2572 を初期化する命令を書きましょう。}
\addquiz{温度を調べる命令を書きましょう。}
\addquiz{湿度(しつど)を調べる命令を書きましょう。}
\addquiz{気圧を調べる命令を書きましょう。}
\addex{sensors.hsp を起動します。センサーに\ruby{触}{ふ}れて温度をあげてみましょう。}
\addex{sensors.hsp を起動します。センサーに息をふきかけて湿度をあげてみましょう。}
\addex{sensors.hsp を起動します。センサーを手でかくして暗くしてみましょう。}
\end{enumerate}
\end{tcolorbox}
