\subsection{LED をチカチカさせよう}

次に LED をチカチカ(\ruby{点滅}{てん|めつ})させてみましょう。HSP スクリプトエディタで Ltika.hsp を開いて実
行してください。\textasciitilde /03 のディレクトリにあります。\\

\begin{lstlisting}[caption=Ltika.hsp,label=Ltika.hsp]
#include "hsp3dish.as"		<#blue#;スクリプトの設定を読み込む#>
#include "rpz-gpio.as"		<#blue#;スクリプトの設定を読み込む#>
	redraw 0		<#blue#;画面更新(仮想画面に描画)#>
	mes "0.5 秒毎に青色の LED がチカチカするよ"
	redraw 1		<#blue#;画面更新(実際の画面に描画)#>
*led
	gpio 22,0		<#blue#;gpio22 を消灯#>
	wait 50 		<#blue#;500 ミリ秒待つ#>
	gpio 22, 1 		<#blue#;gpio22 を点灯#>
	wait 50 		<#blue#;500 ミリ秒待つ#>
	goto *led 		<#blue#;*led にジャンプ(繰り返し)#>
\end{lstlisting}

プログラムを読んでみましょう。点滅させるためには LED の ON と OFF を\ruby{交互}{こう|ご}に繰り返すのでした。
OFF にする \code{gpio} 命令 (\code{gpio 22, 0}) と ON にする \code{gpio} 命令 (\code{gpio 22, 1}) が \code{wait} 命令と交互に
\ruby{現}{あらわ}れることを\ruby{確認}{かく|にん}しましょう。\\

\begin{tcolorbox}[title=\useOmetoi]
\begin{enumerate}
\addex{ターミナルを使って Ltika.hsp のコピーを、322.hsp という名前で作りましょう。}
\addex{ほかの色の LED がチカチカするように 322.hsp のプログラムを変えてみましょう(ヒント:
教科書第 2 回 2-9 LED が点滅するスクリプトを参考にしましょう)。}
\addex{チカチカする速さが速くなるように、322.hsp のプログラムを変えてみましょう(ヒント: 2-9 
LED が点滅するスクリプトを参考にしましょう)。}
\addex{チカチカする速さが\ruby{遅}{おそ}くなるように、322.hsp のプログラムを変えてみましょう。}
\end{enumerate}
\end{tcolorbox}
