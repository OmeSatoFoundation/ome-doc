\subsection{OpenJTalkが読み上げる文字列に変数を使う (1)}
これまではOpenJTalkに文字列を読ませていました。しかし読ませる文字列を変数にしたいことはよくあることです。まずは例として日付を読み上げさせましょう。HSPのgettime命令を使うと今日の日付を取得することができます。\\
\code{gettime(<取得タイプ>)}\\
<取得タイプ>を0から7で指定することで、日付時刻に関する様々な値が返ってきます。
\begin{itemize}
\item 0: 年
\item 1: 月
\item 2: 曜日
\begin{itemize}
 \item 0から6の値が返って来ます。0=日, 1=月, 2=火, 3=水, 4=木, 5=金, 6=土 を表します。
\end{itemize}
\item 3: 日
\item 4: 時
\item 5: 分
\item 6: 秒
\item 7: ミリ秒
\end{itemize}
さて、それではこの情報を使って今日の年、月、日を読み上げさせてみましょう。 \textasciitilde /06/readdate.hsp をHSPスクリプトエディタで開いてください。\\

\begin{lstlisting}[caption=readdate.hsp,label=readdate.hsp]
#include "hsp3dish.as"
#include "jtalk.as"

date = "" + gettime(0) + "年" + gettime(1) + "月" + gettime(3) + "日"

redraw 0
mes "日付を読み上げます"
mes date
redraw 1

jtload date, 0	<#blue#;date変数に代入されている文字を音声ファイルにし、読み込む#>
mmplay 0	<#blue#;音声ファイルを再生する#>
stop
\end{lstlisting}

4行目の代入によって、変数dateにたとえば2019年8月25日と、日付をあらわす文字列が代入されます。これが、11行目と12行目の\code{jtload}命令、\code{mmplay}命令によって読み上げられます。\\

\begin{tcolorbox}[title=\useOmetoi]
\begin{enumerate}
\item readdate.hspを実行してみましょう。\\□←できたらチェックしましょう。
\item 時間があったら、時刻を読み上げるようにプログラムを書きかえてみましょう。\\□←できたらチェックしましょう。
\end{enumerate}
\end{tcolorbox}
