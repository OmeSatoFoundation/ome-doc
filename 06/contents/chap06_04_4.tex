\newpage
\section{OpenJTalkが読み上げる文字列に変数を使う (2)}
 \textasciitilde /06/readcalc.hsp をhspスクリプトエディタで開き、見てみてください。

このプログラムは、足し算の結果を読み上げます。15行目の\code{wait}命令に注目してください。\code{mmplay}は音声を\ruby{再生}{さい|せい}した\ruby{瞬間}{しゅん|かん}に次の命令を実行し始めるので、この\code{wait}命令がないと直ちに「こたえはご」を読み上げ始めてしまい、2つの音声が重なって聞き取りづらくなってしまいます。これを\ruby{避}{さ}けるため、\code{wait}命令で1つめの読み上げが終わるまで待っています。\\

\begin{lstlisting}[caption=readcalc.hsp,label=readcalc.hsp]
#include "hsp3dish.as"
#include "jtalk.as"

a = 3
b = 2

<#blue#; 問題描画#>
redraw 0
mes "" + a + " + " + b
redraw 1

<#blue#; 問題読み上げ#>
jtload "" + a + "たす" + b, 0
mmplay 0
wait 100

<#blue#; 答え計算#>
kotae = a + b

<#blue#; 答え読み上げ#>
jtload "こたえは" + kotae, 1
mmplay 1

<#blue#; 答え描画#>
redraw 0
mes "" + a + " + " + b
mes "= " + kotae
redraw 1

wait 30
stop
\end{lstlisting}

\begin{tcolorbox}[title=\useOmetoi]
\begin{enumerate}
\addex{readcalc.hspをHSPスクリプトエディタで実行してみましょう。}
\addex{時間があったらやりましょう。\\引き算を計算し、答えを読み上げるようにプログラムを変えましょう。}
\end{enumerate}
\end{tcolorbox}
