\subsection{文法を設定する(発展)}
例えば数字を認識させたいときはどうすればよいでしょうか?実はこれを単語辞書のみで実現するとなると大変な労力が必要です。例えば123という数をするには、単語辞書のみを使う場合は「ひゃくにじゅうさん」という単語を登録しないといけません。そうしなければ「いちにさん」と読まれてしまいます。同様に、それぞれの数に対して読みを登録しておかないといけません。また、同じ数字でも読み方が違う場合もあります(例えば、7は「なな」や「しち」と読むことがあります)。つまり、1から1000までの数を認識させたいときは、1000個以上の単語を登録しなければなりません。これを解決するために、「文法」を定めることができます。ただ、これは難しいので、興味があれば以下のページなどを参考に挑戦してみてください。
\begin{itemize}
\item \url{https://atmarkit.itmedia.co.jp/fxml/ddd/ddd004/ddd004-bnf.html}
\item \url{http://julius.osdn.jp/index.php?q=doc/grammar.html}
\end{itemize}