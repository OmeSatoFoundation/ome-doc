\section{まとめ}
OpenJTalkとJuliusの使いかたをまとめました。

OpenJtalkの使いかたは、HSPスクリプトエディタで\\
\code{jtload “<読み上げたい文字列>”, <バッファ番号>}\\
で音声をロードした後、\\
\code{mmplay <バッファ番号>}\\
によって音声を再生します。\\

Juliusはターミナルで使うときは\\
\code{julius.sh <.dicファイル>}\\
のように使います。dic形式の辞書は、.yomiファイル (識別子(コンピュータが使うラベル)と読み方のペア) を作って、ターミナルに\\
\code{convert\_yomi.sh (今作った.yomiファイル) > (出力するファイルの名前(.dicファイル))}
とします。

HSPから使うときは\code{init\_julius}, \code{is\_received}, \code{get\_word\_list} を使って認識結果を取得します。\\
\code{init\_julius <dicファイル>}\\
で辞書ファイルを指定しつつJuliusを初期化します。\\
\code{is\_received (<ソケットID>)}\\
でJuliusが単語を受信したかチェックします。\\
\code{get\_word\_list <単語をあらわす文字列配列> <単語信\ruby{頼}{らい}度をあらわす小数配列>}\\
で<単語をあらわす文字列配列>に認識した単語、<単語信頼度をあらわす小数配列>に単語信頼度が代入されます。
