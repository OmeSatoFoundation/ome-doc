\section{OpenJTalk}
ここでは音声合成ソフトウェアOpenJTalkの紹介をします。まずは使ってみましょう。
\subsection{文を読み上げてもらう}
文を読み上げるHSPプログラムを紹介します。openjtalk.hsp をHSPスクリプトエディタで開いてください。openjtalk.hsp は /home/pi/ome/06/ の中にあります。\\

\begin{lstlisting}[caption=openjtalk.hsp,label=openjtalk.hsp]
#include "hsp3dish.as"
#include "jtalk.as"

redraw 0
mes "発話開始"
mes "「子どもIT未来塾」"
redraw 1

jtload "子どもIT未来塾", 0	<#blue#;音声ファイルを出力し、読み込む#>
mmplay 0			<#blue#;音声ファイルを再生する#>
stop
\end{lstlisting}

2行目\\
\code{\#include “jtalk.as”}\\
で、openjtalkをHSPで使うのに必要な設定が行われます。この行によって、後に説明する jtload 命令と setvoice 命令が使えるようになります。\\
4行目から7行目までは復習です。mes命令で文字を表示させています。\\
9行目からが新しい命令です。\\
\code{jtload "子どもIT未来塾", 0}\\
で、0番のバッファに “子どもIT未来塾” と読ませた音声ファイルがロードされます。 画像を配置するときや、I2Cを使うときと同じような仕組みです。より一般的に書くと、jtload命令は\\
jtload <読み上げ文字列>, <バッファ番号>\\
となります。\\
10行目\\
\code{mmplay 0}\\
で、0番のバッファ(今ロードしたバッファ)を再生します。より一般的に書くと、\\
\code{mmplay <再生するバッファ番号>}\\
となります。この命令は再生開始後すぐに次の命令を実行するので、複数の音声を順番に再生したいときはwait命令を入れるなどの工夫が必要になります。\\

\begin{tcolorbox}[title=\useOmetoi]
\begin{enumerate}
\item 「本日は青天なり」と読み上げるようにプログラムを書き変えて、sunny.hspという名前で/home/pi/ome/06/に保存し、実行しましょう。\\ヒント:jtload "子どもIT未来塾", 0をかきかえます。\\□←できたらチェックしましょう。
\item 時間があったらやりましょう。\\好きな言葉を読み上げさせましょう。\\□←できたらチェックしましょう。
\end{enumerate}
\end{tcolorbox}