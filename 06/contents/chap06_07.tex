\newpage
\section{宿題}
この中から好きな問題を1つ以上選んで、プログラムを作ってみましょう。
\begin{enumerate}
\item HSPで以下のようなプログラムを書きましょう。まず数字 (0から9まで) のキー入力を受け付けます。キーが\ruby{押}{お}されたらすぐにその数字をOpenJTalkに読み上げさせます。例えば、4キーが押されたら「よん」と読み上げさせます。
\item HSPで以下のようなプログラムを書きましょう。数字(0から9まで)のキーとエンターキーの入力を受け付けます。エンターキーが押されたら、プログラム開始後または直前にエンターキーが押されたあとに押された数字を順番に全て読み上げさせます。たとえば、 [1, 2, エンターキー, 5, 0, エンターキー]とキーが押されたら、最初にエンターキーが押されたときに「いち に」と、2番めのエンターキーがおされたら「ご ぜろ」と読み上げさせます。
\item HSPで以下のようなプログラムを書きましょう。\ruby{異}{こと}なる種類の画像を3\ruby{枚}{まい}用意します(例:車の画像、犬の画像、時計の画像)。これを画面に\ruby{並}{なら}べて表示します。どれかの画像がクリックされたら、直ちにその画像に写っているものを読み上げさせます。例えば、車の画像がクリックされたら「くるま」と読み上げさせます。
\item HSPで、文字の色を声で変えるプログラムを作りましょう。まず、画面に文字を表示します。例えば、「こんにちは」を表示しておきます。はじめの色は何色でも\ruby{構}{かま}いません。「あお」、「あか」、「きいろ」という音声を受け付けます。もし、「あお」という声が入力されたら、文字の色を青色にします。もし、「あか」という声が入力されたら、文字の色を赤色にします。もし、「きいろ」という声が入力されたら、文字の色を黄色にします。
\item HSPで以下のようなプログラムを書きましょう。異なる種類の画像を3枚用意します。初めは何も画面に表示しません。今用意した3種類の画像を表す声を受け付けます。声が入力されたら、その声が表す画像だけを表示させます。例えば、 車の画像、犬の画像、時計の画像を用意します。もし「くるま」という声を入力したら、画面に車の画像を表示します。他の画像は表示しません。以下同様にします。
\end{enumerate}
