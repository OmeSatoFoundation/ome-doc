\subsection{音声でLEDのON OFFをする}
HSPプログラムから、音声を使ってLEDをつけたり消したりしてみましょう。/home/pi/ome/06にonoff.yomiファイルがあります。これは音声でLEDのON/OFFをするための辞書ファイルです。\\

\begin{lstlisting}[caption=onoff.yomi,label=onoff.yomi]
ON おん
OFF おふ
\end{lstlisting}

リスト\ref{onoff.yomi}のように書いてあります。ONとOFFが表示する文字、おんとおふが読み方です。これを\code{convert\_yomi.sh}を使って変換したものがonoff.dicです。onoff.dicを使って、音声からLEDを制御してみましょう。/home/pi/ome/06/ledbyvoice.hsp を開いてください。\\

\begin{lstlisting}[caption=ledvoice.hsp,label=ledvoice.hsp]
#include "hsp3dish.as"
#include "julius.as"

redraw 0
pos 0, 0
mes "Loading Julius..."
redraw 1

init_julius "/home/pi/ome/06/onoff.dic"

sockidx = 0
domain = "127.0.0.1"
PORT = 10500

sockopen sockidx, domain, PORT

if stat != 0 {
  redraw 0
  mes "Error in opening socket"
  redraw 1
  wait 30
  stop
}

sdim wrods, 4096, 0
ddim cm, 0
repeat -1
  if is_recieved(sockidx) = 0 {
    get_word_list words, cm
    repeat length(words)
      if cm(cnt) > 0.7 {
        if words(cnt) = "ON" {			<#blue#;認識された文字がONのとき#>
          gpio 17, 1				<#blue#;GPIO17のLEDを光らせる#>
        }else : if words(cnt) = "OFF" {		<#blue#;認識された文字がOFFのとき#>
          gpio 17, 0				<#blue#;GPIO17を消す#>
        }
      }
    loop
  }
  redraw 0
  pos 0,0
  mes "話しかけてLEDのONとOFFをしよう"
  redraw 1
  await 30
loop
\end{lstlisting}

words(cnt)に認識された文字が代入されています。認識された文字がONなのかOFFかif文(条件分岐)を使って、LEDを制御しています。\\

\begin{tcolorbox}[title=\useOmetoi]
\begin{enumerate}
\item ledbyvoice.hspをHSPスクリプトエディタで実行しましょう。\\□←オン、オフと話しかけてLEDをつけたり消したりできたらチェックしましょう。
\item 点灯させるLEDを変更しましょう。\\□←できたらチェックしましょう。
\item 時間があったらやりましょう。\\点灯させるための言葉を「つける」と「けす」に変更しましょう。\\□←できたらチェックしましょう。
\item 時間があったらやりましょう。\\センサーボードには4色のLEDが点いています。色ONでその色のLEDがつき、色OFFでその色のLEDが消えるようなプログラムを書きましょう。(オレンジONでオレンジ色のLEDがつき、オレンジOFFでオレンジ色のLEDが消えるなど)\\□←できたらチェックしましょう。
\end{enumerate}
\end{tcolorbox}