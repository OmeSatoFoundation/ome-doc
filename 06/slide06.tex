% LuaLaTeX文書; 文字コードはUTF-8
\documentclass{beamer}% 'unicode'が必要
\usepackage{luatexja-fontspec}
\newjfontface{\nonproportional}{BIZ UDMincho}[YokoFeatures={JFM=ujis}]
\setmainjfont[Ligatures={Common,TeX},
  ItalicFont={BIZ UDPMincho},
  ItalicFeatures={FakeSlant=0.27},
  SlantedFont={BIZ UDPMincho},
  SlantedFeatures={FakeSlant=0.18},
  BoldFont={BIZ UDPGothic Bold},
  BoldSlantedFont={BIZ UDPGothic Bold},
  BoldSlantedFeatures={FakeSlant=0.18},
  BoldItalicFont={BIZ UDPGothic Bold},
  BoldItalicFeatures={FakeSlant=0.27},
  YokoFeatures={JFM=prop}]{BIZ UDPMincho}
%\newjfontface{\nonproportional}{BIZ UDGothic}[YokoFeatures={JFM=ujis}]
\setsansjfont[Ligatures={Common,TeX},
  ItalicFont={BIZ UDPGothic},
  ItalicFeatures={FakeSlant=0.23},
  SlantedFont={BIZ UDPGothic},
  SlantedFeatures={FakeSlant=0.23},
  BoldFont={BIZ UDPGothic Bold},
  BoldSlantedFont={BIZ UDPGothic Bold},
  BoldSlantedFeatures={FakeSlant=0.23},
  BoldItalicFont={BIZ UDPGothic Bold},
  BoldItalicFeatures={FakeSlant=0.23},
  YokoFeatures={JFM=prop}]{BIZ UDPGothic}


\setbeamertemplate{footline}[frame number]
\title{子どもIT未来塾 第6回}
\author{塾長 清水尚彦}

\begin{document}

\frame{
   \begin{center}
    \huge{子どもIT未来塾}\\

    \vspace{48pt}
	   \Large{第1回}\\
	   {\huge\bf ラズベリーパイの使い方・\\
	   \huge\bf 自己紹介ページを作ろう}\\
    \vspace{24pt}
    \large{塾長 しみずせんせい}\\
    \vspace{10pt}
    \large{\the\year 年 6月24日}
  \end{center}
}


\end {document}


\begin{frame}[fragile]
	%\frametitle{今回の授業:テキスト P.1~~~\raisebox{-3mm}{\includegraphics[width=0.1\textwidth]{raspberry}}}
	\frametitle{今回の授業:テキスト P.1~~~}
		\begin{description}
			\item[目標] ~\\
				\begin{itemize}
					\item ラズベリーパイになれよう
					\item 自分のホームページを作ろう
				\end{itemize}

			\item[授業の進め方]~\\
				\begin{itemize}
					\item 重要なことを先生が説明します
					\item スライドに示すページを開こう
					\item 例題・問題をすぐやろう\\
						やった問題にシールを貼ろう
				\end{itemize}
		\end{description}
		\vfill
		わからないことは、放っておかず、すぐにTAに聞きましょう
\end{frame}

\end{document}
