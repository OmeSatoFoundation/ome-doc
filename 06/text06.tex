\documentclass[a4paper,12pt]{ltjsbook}
\usepackage{omebook}
\usepackage{svg}
\svgsetup{
  inkscapelatex=false
}
\setcounter{chapter}{5}
\setcounter{secnumdepth}{4}
\begin{document}
  \input contents/chap06_01.tex
  \input contents/chap06_02.tex
  \input contents/chap06_03.tex
  \input contents/chap06_04_1.tex
  \input contents/chap06_04_2.tex
  \input contents/chap06_04_3.tex
  \input contents/chap06_04_4.tex
  \input contents/chap06_05.tex
  \input contents/chap06_05_1.tex
  \input contents/chap06_05_2.tex
  \input contents/chap06_05_3.tex
  \input contents/chap06_05_4.tex
  \input contents/chap06_05_5.tex
  \input contents/chap06_05_6.tex
  \input contents/chap06_06.tex
  \input contents/chap06_07.tex
  \begin{thebibliography}{99}
  \bibitem{hsp} おにたま, 悠黒喧史, うすあじ. HSPでつくる! はじめてのプログラミング HSP3.5+3Dish入門. 秀和システム. 2017
  \\ OpenJtalkについて
  \bibitem{jtalk1} Open JTalk - HMM-based Text-to-Speech System \\ \url{http://open-jtalk.sp.nitech.ac.jp/}
  \bibitem{jtalk2} Open JTalk (テキストを音声へ変換)をインストールし、使ってみる | レンタルサーバー・自宅サーバー設定・構築のヒント \\ \url{https://server-setting.info/centos/open-jtalk-install.html}
  \\ Juliusについて
  \bibitem{julius} 大語彙連続音声認識エンジン Julius \\ \url{http://julius.osdn.jp/}
  \end{thebibliography}
\end{document}
