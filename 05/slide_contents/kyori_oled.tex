\begin{frame}[plain]
    \begin{center}
        \vspace{48pt}
        {\huge\bf 距離センサーを使ってみよう}
    \end{center}
\end{frame}

\begin{frame}[fragile]
    \frametitle{距離センサーをピンにつけてみよう}
    \begin{columns}
        \begin{column}{0.48\textwidth}
            \includegraphics[width=\textwidth]{images/chap05/text05-img030.png} 
        \end{column}
        \begin{column}{0.48\textwidth}
            \includegraphics[width=\textwidth]{images/chap05/text05-img023.jpg} 
        \end{column}
    \end{columns}
    \begin{itemize}
        \item A0と距離センサーをつなげてみよう
        \item HSPでkyori.hspを動かしてみよう
    \end{itemize}
\end{frame}

\begin{frame}[fragile]
    \frametitle{距離センサーを使ったプログラム(kyori.hsp)}
\begin{lstlisting}
#include "hsp3dish.as"
#include "rpz-gpio.as"

spiopen 0

*main

	data = spiget(0,0)
	kyori = -1*(data*5000/1023)/36+845/9
	res = "距離 : "+kyori
	
	redraw 0
	font "",20
	pos 30,30
	mes res
	redraw 1

	wait 10
	goto *main

spiclose 0
\end{lstlisting}
\end{frame}

\begin{frame}[plain]
    \begin{center}
        \vspace{48pt}
        {\huge\bf 有機ELディスプレイを使ってみよう}
    \end{center}
\end{frame}

\begin{frame}
    \frametitle{有機ELディスプレイ}
    \begin{center}
        \includegraphics[width=0.4\textwidth]{images/chap05/text05-img025.png}
        \begin{itemize}
            \item 文字(記号や英数字)を画面に表示する
            \item ここでは日本語を表示できない
        \end{itemize}
    \end{center}
\end{frame}

\begin{frame}[fragile]
    \frametitle{有機ELディスプレイをピンにつけてみよう}
    \begin{columns}
        \begin{column}{0.48\textwidth}
            \includegraphics[width=\textwidth]{images/chap05/text05-img033.png} 
        \end{column}
        \begin{column}{0.48\textwidth}
            \includegraphics[width=\textwidth]{images/chap05/text05-img025.png} 
        \end{column}
    \end{columns}
    \begin{itemize}
        \item 有機ELディスプレイとI2Cをつなげてみよう
        \item HSPでoled.hspを動かしてみよう
    \end{itemize}
\end{frame}

\begin{frame}[fragile]
    \frametitle{有機ELディスプレイを使ったプログラム(oled.hsp)}
\begin{lstlisting}
#include "hsp3dish.as"
#include "rpz-gpio.as"

redraw 0
font "",20
pos 20,20
mes "有機ELディスプレイを見てね\nカンマ(,)で改行するよ"
redraw 1

oled "Good Morning,Good Bye,Good Afternoon"

wait 100
\end{lstlisting}
\end{frame}

\begin{frame}[fragile]
    \frametitle{問題を解いてみよう}
    \begin{description}
        \item a
        \item b
    \end{description}
\end{frame}