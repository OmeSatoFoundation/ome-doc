\begin{frame}[plain]
    \begin{center}
        \vspace{48pt}
        {\huge\bf 赤外線(せきがいせん)}
    \end{center}
\end{frame}

\begin{frame}[fragile]
    \frametitle{赤外線(IR)とは}
    \centering
    \includesvg[width=\linewidth]{images/chap05/electromagnetic_wave.svg}
\end{frame}

\begin{frame}
    \frametitle{赤外線で通信する}
    \begin{itemize}
        \item ついていたらA, 消えていたらB
        \item ex.
        \begin{itemize}
            \item 点いている: ロボットが前に進む
            \item 消えている: ロボットが止まる
        \end{itemize}
    \end{itemize}
    \note{画像を入れる}
\end{frame}

\begin{frame}
   \frametitle{2通りでは足りない} 
   \begin{itemize}
        \item 2通りでは
        \begin{itemize}
            \item 止まる
            \item 前に進む
        \end{itemize}
        \item だけ\dots
        \begin{itemize}
            \item 左に進む
            \item 右に進む
            \item 飛ぶ
            \item \dots
        \end{itemize}
   \end{itemize}
\end{frame}

\begin{frame}
    \frametitle{リモコンの原理} 
    \begin{itemize}
        \item 1秒点いて2秒消えてまた2秒点いたら歩く
        \item 2秒点いて3秒消えたらジャンプ
    \end{itemize}
    \centering
    \includesvg[width=0.8\linewidth]{images/chap05/text05-img037.svg}
\end{frame}

\begin{frame}
    \frametitle{赤外線をつかった家電}
    \begin{figure}[H]
    \begin{minipage}[t]{0.3\columnwidth}
        \centering
    \includesvg[scale=0.1]{images/chap05/tv.svg}
        \caption{テレビ}
    \end{minipage}
    %\hspace{0.01\columnwidth} % ここで隙間作成
    \begin{minipage}[t]{0.3\columnwidth}
        \centering
        \includesvg[scale=0.1]{images/chap05/aircondition.svg}
        \caption{エアコン}
    \end{minipage}
    \begin{minipage}[t]{0.3\columnwidth}
        \centering
        \includesvg[scale=0.1]{images/chap05/fan.svg}
        \caption{せんぷう機}
    \end{minipage}
    \end{figure}
\end{frame}

\begin{frame}
    \frametitle{Raspberry Piをリモコンにする}
    \begin{itemize}
        \item リモコンを1つにまとめられる
        \item プログラムで制御(せいぎょ)できる
        \begin{itemize}
            \item 外出先からエアコンをつける
            \item 音声でテレビのチャンネルを変える
        \end{itemize}
    \end{itemize}
\end{frame}

\begin{frame}
    \frametitle{Raspberry Piリモコンのしくみ}
    \centering
    \includesvg[width=0.8\linewidth]{images/chap05/IRproc.svg}
\end{frame}

\begin{frame}
    \frametitle{Raspberry Piをリモコンにする手順} 
    リモコンの動作をマネすることが必要
    \begin{enumerate}
        \item リモコンの信号を受信してファイルに記録する
        \item ファイルに記録された信号を赤外線LEDから出力する
    \end{enumerate}
\end{frame}

\begin{frame}
    \frametitle{班分け}
    \begin{itemize}
        \item チームごとに
        \begin{itemize}
            \item ロボットを使う班
            \item 扇風機(せんぷうき)を使う班
        \end{itemize}
        \item で別れて、ロボットか扇風機を準備しよう
    \end{itemize}
\end{frame}

\begin{frame}
    \frametitle{マネするだけの方法(1)} 
    \begin{enumerate}
        \item cd \~/05
        \item sudo service lircd stop
        \item mode2 -d /dev/lirc1 \textbar tee onoff.txt
        \item convert\_pattern onoff.txt \textgreater onoff.pattern
    \end{enumerate}
\end{frame}

\begin{frame}
    \frametitle{マネするだけの方法(2)} 
    onnoff.patternを使って設定ファイルを作る
    \begin{enumerate}
        \item cp template.lircd.conf 05.lircd.conf
        \item mousepad 05.lircd.conf
        \item mousepad onoff.pattern
    \end{enumerate}
\end{frame}

\begin{frame}[fragile]
    \frametitle{マネするだけの方法(3)}
    赤字の部分を書き換えよう
    \begin{lstlisting}
begin remote
    name <#red#fan#>
    flags RAW_CODES
    eps 30
    aeps 100

    gap 200000
    toggle_bit_mask 0x0
    
    begin raw_codes
        name <#red#onoff#>
        <#red#1207 588 447 838 447 823 475 812 1218 ...#>
    end raw_codes
end remote
    \end{lstlisting}
\end{frame}

\begin{frame}
    \frametitle{マネするだけの方法(4)}
    \begin{itemize}
        \item sudo cp 05.lircd.conf /etc/lirc/lircd.conf.d/
        \begin{itemize}
            \item 設定ファイルをコピー
        \end{itemize}
        \item sudo service lidrcd restart
        \begin{itemize}
            \item 設定ファイルが再読み込みされる
        \end{itemize}
        \item irsend SEND\_ONCE \color{red}fan onoff\color{black}
    \end{itemize}
\end{frame}

\begin{frame}
    \frametitle{問題をといてみよう}
    \begin{itemize}
        \item 教科書30ページ 例題5-12
        \begin{itemize}
            \item 手順を読みながら、赤外線を入力・出力してみよう
            \item 1問
        \end{itemize}
    \end{itemize}
\end{frame}

\begin{frame}
    \frametitle{HSPで赤外線を出力}
    \begin{itemize}
        \item exec命令を使ってirsendコマンドを呼び出す
        \begin{itemize}
            \item exec "irsend SEND\_ONCE fan onoff" 
        \end{itemize}
    \end{itemize}
\end{frame}

\begin{frame}[fragile]
    \frametitle{ボタンを押して扇風機を回す (ir.hsp)}
\begin{lstlisting}
#include "hsp3dish.as"
BUTTON_PIN = 24
prev = 0
status=0

*main
    redraw 0 
    font "",20
    pos 0,0
    mes "ボタンPUSHで赤外線照射!"
    redraw 1
    wait 1
    goto *edge

*edge
    current = gpioin(BUTTON_PIN)
    if (prev=0) & (current=1) {	 <#blue#;ボタンが押されたら#>
        exec "irsend SEND_ONCE fan onoff"<#blue#;irsendコマンドを実行する#>
    }
    prev = current
    wait 1
    goto *edge
\end{lstlisting}
\end{frame}

\begin{frame}
    \frametitle{問題を解いてみよう}
    \begin{itemize}
        \item 教科書31ページ 問題5-13
        \begin{itemize}
            \item 1問
        \end{itemize}
    \end{itemize}
\end{frame}

\begin{frame}[fragile]
    \frametitle{信号を二つ以上登録したいとき}
\begin{lstlisting}[caption=2つの信号を登録するときのtemplate.lircd.comf,label=2つの信号を登録するときのtemplate.lircd.comf]
begin remote
  name fan
  flags RAW_CODES
  eps 30
  aeps 100

  gap 200000
  toggle_bit_mask 0x0

  begin raw_codes

      name onoff
      1207 588 447 838 447 823 475 812 1218…

      <#red#name power
      1364 363 752 743 372 836 164 614 ...#>

  end raw_codes

end remote
\end{lstlisting}
\end{frame}

\begin{frame}[fragile]
    \frametitle{赤外線受信のまとめ}
\begin{lstlisting}
mode2 -d /dev/lirc1 | tee <filename1>
convert_pattern <filename1> > <filename2>
<filename3>.confを<filename2>を使って作成
sudo cp <filename3>.conf /etc/lirc/lircd.d/
sudo service lircd restart
\end{lstlisting}
\end{frame}

\begin{frame}
    \frametitle{赤外線を送るコマンド}
    irsend \textless 送り方\textgreater \textless リモコンの名前\textgreater \textless 信号の名前\textgreater

    \begin{itemize}
        \item \textless 送り方\textgreater
        \begin{itemize}
            \item SEND\_ONCE 一度だけ送る
            \item SEND\_START 送り続ける
            \item SEND\_STOP "SEND\_START"で送り続けていた信号を止める
        \end{itemize}
        \item \textless リモコンの名前\textgreater
        \begin{itemize}
            \item なんとか.lircd.confファイルの"begin remote"の直後の"name"
        \end{itemize}
        \item \textless 信号の名前\textgreater
        \begin{itemize}
            \item なんとか.lircd.confファイルの"begin codes"または"begin codes\_raw"の直後の"name"
        \end{itemize}
    \end{itemize}
\end{frame}

\begin{frame}
    \frametitle{センサー紹介}
    教科書34ページから46ページに詳しいセンサーの紹介があります。
\end{frame}

\begin{frame}
    \frametitle{宿題}
    教科書32ページ 問題 5-14
    \begin{itemize}
        \item 自分の家にある家電をRaspberry Piで制御してみましょう。
        \item エアコン、風船機、テレビなど赤外線を使って操作する家電を選びましょう。
        \item 赤外線で動かない家電もあります。宿題の結果が動かなかった、でもOKです。
        \item 1問
    \end{itemize}
\end{frame}