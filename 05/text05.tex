\documentclass[a4paper,12pt]{ltjsbook}
\usepackage{omebook}
\usepackage{svg}
\svgsetup{
  inkscapelatex=false
}
\setcounter{chapter}{4}
\setcounter{secnumdepth}{4}

% 表の行間を広げるための環境
% デフォルトは1.4倍
\newenvironment{widerrows}[1][1.4]
  {\renewcommand\arraystretch{#1}}
  {}

\begin{document}
  \input contents/chap05_01.tex
  \input contents/chap05_02_1.tex
  \input contents/chap05_02_2.tex
  \input contents/chap05_02_3.tex
  \input contents/chap05_03_1.tex
  \input contents/chap05_03_2.tex
  \input contents/chap05_03_3.tex
  \input contents/chap05_03_4.tex
  \input contents/chap05_03_5.tex
  \input contents/chap05_03_6.tex
  \input contents/chap05_03_7.tex
  \input contents/chap05_03_8.tex
  \input contents/chap05_04_1.tex
  \input contents/chap05_04_2.tex
  \input contents/chap05_04_3.tex
  \input contents/chap05_04_4.tex
  \input contents/chap05_04_5.tex \newpage
  \input contents/chap05_05_1.tex \newpage
  \input contents/chap05_05_2.tex \newpage
  \input contents/chap05_05_3.tex \newpage
  \input contents/chap05_05_4.tex \newpage
  \input contents/chap05_05_5.tex \newpage
  \input contents/chap05_05_6.tex \newpage
  \input contents/chap05_05_7.tex \newpage
  \input contents/chap05_05_8.tex \newpage
  \input contents/chap05_05_9.tex \newpage
  \input contents/chap05_05_10.tex \newpage
  \input contents/chap05_05_11.tex  
  \begin{thebibliography}{99}
  \bibitem{matlab} 樋口龍雄, 川又政征. MATLAB対応ディジタル信号処理. 森北出版. 2015.
  \bibitem{elecSym} 日本産業標準調査会. 電気用図記号. 2011. (規格番号 JISC0617)
  \bibitem{fabo} Fabo, Inc \\ \url{https://www.fabo.io/}
  \bibitem{curve} ナルガッキ. ボリューム、トーンに適したポットのカーブの特徴と違い. (図\ref{ボリュームの特性}) \\ \url{https://naru-gakki.com/pot-curve/}
  \end{thebibliography}
\end{document}
