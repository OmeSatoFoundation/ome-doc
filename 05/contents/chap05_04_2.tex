\subsection{赤外線をつかった家電}

\begin{figure}[H]
  \begin{minipage}[t]{0.3\columnwidth}
    \centering
 \includesvg[scale=0.1]{images/chap05/tv.svg}
    \caption{テレビ}
  \end{minipage}
  %\hspace{0.01\columnwidth} % ここで隙間作成
  \begin{minipage}[t]{0.3\columnwidth}
    \centering
    \includesvg[scale=0.1]{images/chap05/aircondition.svg}
    \caption{エアコン}
  \end{minipage}
  \begin{minipage}[t]{0.3\columnwidth}
    \centering
    \includesvg[scale=0.1]{images/chap05/fan.svg}
    \caption{せんぷう機}
  \end{minipage}
\end{figure}

テレビ、エアコン、\ruby{扇風機}{せん|ぷう|き}などはリモコンを使って動作を\ruby{制御}{せい|ぎょ}します。ボタンを\ruby{押}{お}せば\ruby{電源}{でん|げん}がついたり、風量や音量を調節できます。これは赤外線を使って家電を動作させているためです。電源を入れるときはリモコンから電源を入れるための赤外線信号が送られます。家電はそれを受け取り、信号の種類で決められた動作をします。
