\newpage
\section{\ruby{距離}{きょ|り}センサーを使ってみよう}
距離センサーをHSPで使ってみましょう。距離センサー(番号は書いてありません。\pageref{distance}ページを参考にしてください)をA0につなげましょう。HSPスクリプトエディタで、\textasciitilde /05/kyori.hspを開いて実行してください。
\begin{center}
  \begin{minipage}[t]{.3\columnwidth}
    \begin{figure}[H]
      \centering
      \includegraphics[width=.8\linewidth]{images/chap05/text05-img031.png}
      \caption{距離センサー}
    \end{figure}
  \end{minipage}
  %\hspace{0.01\columnwidth} % ここで隙間作成
  \begin{minipage}[t]{.5\columnwidth}
    \begin{figure}[H]
    \centering
    \includegraphics[width=.8\linewidth]{images/chap05/text05-img030.png}
    \caption{A0}
    \end{figure}
  \end{minipage}
\end{center}
\begin{lstlisting}[caption=kyori.hsp,label=kyori.hsp]
#include "hsp3dish.as"
#include "rpz-gpio.as"

spiopen 0	<#blue#;SPIチャンネルを開く#>

*main

	data = spiget(0,0)	<#blue#;SPIを使ってデータを受け取る#>
	kyori = -1*(data*5000/1023)/36+845/9	<#blue#;dataを距離に変換する#>
	res = "距離 : "+kyori
	
	redraw 0
	font "",20
	pos 30,30
	mes res
	redraw 1

	wait 10
	goto *main

spiclose 0	<#blue#;SPIチャンネルを閉じる#>
\end{lstlisting}

距離センサーのデータは\code{spiget}を使って受け取ることができます。\ruby{値}{あたい}は0~1023になります。これでは距離がわからないので、受け取った値をcm(センチメートル)に変換する必要があります。変換するための式は\pageref{distance}ページに書いてあります。今回は入力は\code{data}という変数にしているので、式のxをdataにします。\code{kyori = -1*(data*5000/1023)/36+845/9}で入力を距離に変換し、\code{kyori}変数に代入しています。単位はcmです。\\

\begin{tcolorbox}[title=\useOmetoi]
\begin{enumerate}
\addex{距離センサーをA0につなげてkyori.hspを実行してみましょう。\ruby{測定}{そく|てい}できる距離はだいたい10~80cmです。}
\end{enumerate}
\end{tcolorbox}
