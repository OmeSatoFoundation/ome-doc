\subsubsection{振動子}\label{vibrator}
\begin{table}[H]
	\begin{tabular}{|p{\colF}|p{\colG}|}	\hline
	名称 & 振動子(しんどうし)\\ \hline
	接続箇所 & デジタルコネクタ (3pin)\\ \hline
	機能概要 & 振動モーターでふるえる\\ \hline
  \end{tabular}
\end{table}

\begin{table}[H]
	\begin{tabular}{|p{\colF}|p{\colG}|}	\hline
	サンプルコードの場所 & 05/digout.hsp\\ \hline
	raspiへの入力 & なし\\ \hline
	raspiへの入力方法 & なし\\ \hline
	raspiからの出力 & 値が1のときふるえ、値が0のとき動かない\\ \hline
	raspiからの出力方法 & gpio GPIO番号, パラメータ\\ \hline
  \end{tabular}
\end{table}

\begin{table}[H]
	\begin{tabular}{|p{\colF}|p{\colG}|} \hline
	使い道 & 振動で何かが起きたことを伝える。目覚まし時計など。\\ \hline
	注意事項 & 振動している部分に触って\ruby{怪我}{け|が}をしたり、\ruby{壊}{こわ}したりしないように注意\\ \hline
	補足 & なし\\ \hline
  \end{tabular}
\end{table}

\begin{figure}[H]
	\begin{tabular}{|p{\colH}|p{\colI}|p{\colH}|p{\colI}|} \hline
	外観 & 
	\begin{minipage}[t]{\linewidth}
    \smallskip
      \centering
      \includegraphics[width=\linewidth]{images/chap05/text05-img017.png}
      \caption{振動子}
      \smallskip
    \end{minipage} &
    回路記号 & 
    \begin{minipage}[t]{\linewidth}
    \smallskip
      \centering
      \includegraphics[width=\linewidth]{images/chap05/text05-img044.png}
      \caption{振動子の回路図}
      \smallskip
    \end{minipage}\\ \hline
  \end{tabular}
\end{figure}
