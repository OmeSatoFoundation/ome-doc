\section{FaBoで使うことのできるブリック}
子どもIT\ruby{未来塾}{み|らい|じゅく}で使うブリックについて\ruby{紹介}{しょう|かい}します。ブリックには番号が書いてあるので、表で\ruby{確認}{かく|にん}しながら\ruby{間違}{ま|ちが}えて使わないように注意しましょう。\ref{sensor_intr}章の「\ruby{付録}{ふ|ろく}:センサー紹介」にはブリックについての\ruby{詳}{くわ}しい説明が書いてあります。問題を\ruby{解}{と}くときや自分で使ってみるときに参考にしましょう。\\

% 入出力
\newlength{\colA}
\setlength{\colA}{0.15\columnwidth}
% ブリック
\newlength{\colB}
\setlength{\colB}{0.15\columnwidth}
% 機能
\newlength{\colC}
\setlength{\colC}{0.3\columnwidth}
% 画像
\newlength{\colD}
\setlength{\colD}{0.2\columnwidth}
% 紹介ページ
\newlength{\colE}
\setlength{\colE}{0.07\columnwidth}

\begin{table}[H]
  \begin{widerrows}
		\begin{tabular}{|p{\colA}|p{\colB}|p{\colC}|p{\colD}|p{\colE}|}
			\hline
		入出力 &ブリック / \par ブリック番号 & \ruby{機能}{き|のう} & \ruby{画像}{が|ぞう} & 紹介 \par ページ\\ \hline
		デジタル出力\ruby{装置}{そう|ち} & LED / \#101 & 
		\begin{minipage}[t]{\linewidth}
		%\smallskip
		\begin{itemize}
		\item 順方向に\ruby{電圧}{でん|あつ}を加えると発光する
		\item プログラムでは\ruby{値}{あたい}が1のとき光り、値が0のとき消える
		\end{itemize}
		\smallskip
		\end{minipage} & 
			\begin{minipage}[t]{\linewidth}
			\smallskip
				\centering
				\includegraphics[width=\linewidth]{images/chap05/text05-img016.png}
				\smallskip
			\end{minipage} &
		\pageref{LED}\\ \cline{2-5}
		& \ruby{振動子}{しん|どう|し} / \#105 & 
		\begin{minipage}[t]{\linewidth}
		%\smallskip
		\begin{itemize}
		\item 電圧が加わると振動する
		\item プログラムでは値が1のとき振動し、値が0のとき止まる
		\end{itemize}
		\smallskip
		\end{minipage} & 
			\begin{minipage}[t]{\linewidth}
			\smallskip
				\centering
				\includegraphics[width=0.8\linewidth]{images/chap05/text05-img017.png}
				\smallskip
			\end{minipage} &
			\pageref{vibrator}\\ \hline
			
			デジタル入力装置 & ボタン / \#103 & 
		\begin{minipage}[t]{\linewidth}
		%\smallskip
		\begin{itemize}
		\item \ruby{押}{お}したり\ruby{離}{はな}したりしてONとOFFを切り\ruby{替}{か}える。写真はボタンが黄色だが青や黒などの色もある。
		\item プログラムではボタンが押されている間1、ボタンが押されていないと0が入力される
		\end{itemize}
		\smallskip
		\end{minipage} & 
			\begin{minipage}[t]{\linewidth}
			\smallskip
				\centering
				\includegraphics[width=0.8\linewidth]{images/chap05/text05-img028.png}
				\smallskip
			\end{minipage} &
		\pageref{button}\\ \cline{2-5}
		& \ruby{傾斜}{けい|しゃ}センサー / \#110 & 
		\begin{minipage}[t]{\linewidth}
		%\smallskip
		\begin{itemize}
		\item 傾いているかを調べることができる
		\item 傾くと1,傾いていないと0の値が入力される
		\end{itemize}
		\smallskip
		\end{minipage} & 
			\begin{minipage}[t]{\linewidth}
			\smallskip
			\smallskip
				\centering
				\includegraphics[width=0.8\linewidth]{images/chap05/text05-img018.png}
				\smallskip
			\end{minipage} &
			\pageref{tilt}\\ \hline   
		\end{tabular}
  \end{widerrows}	
\end{table}

\begin{table}[H]
  \begin{widerrows}
		\begin{tabular}{|p{\colA}|p{\colB}|p{\colC}|p{\colD}|p{\colE}|}
			\hline
			入出力 &ブリック/ \par ブリック番号 & 機能 & 画像 & 紹介ページ\\ \hline
			デジタル入力装置 & スイッチ / \#117 & 
			\begin{minipage}[t]{\linewidth}
				%\smallskip
				\begin{itemize}
					\item ボタンと\ruby{似}{に}ているが、ONのまま、OFFのままにできる
					\item ●にスライドさせるとON、○にスライドさせるとOFFになる
					\item ONのときは1,OFFのときは0が入力される
				\end{itemize}
				\smallskip
			\end{minipage} & 
			\begin{minipage}[t]{\linewidth}
				\smallskip
				\centering
				\includegraphics[width=0.8\linewidth]{images/chap05/text05-img019.jpg}
				\smallskip
			\end{minipage} &
			\pageref{switch}\\ \cline{2-5}
			& リミットスイッチ / \#107 & 
			\begin{minipage}[t]{\linewidth}
				%\smallskip
				\begin{itemize}
					\item ボタンに似ている。銀の部分が押されているとON、押されていないとOFFになる
					\item スイッチはONのときは1,OFFのときは0が入力される
				\end{itemize}
				\smallskip
			\end{minipage} & 
			\begin{minipage}[t]{\linewidth}
				\smallskip
				\centering
				\includegraphics[width=0.8\linewidth]{images/chap05/text05-img020.jpg}
				\smallskip
			\end{minipage} &
			\pageref{lmswitch}\\ \hline   
			アナログ入力装置 & 感圧センサー / \#106 & 
			\begin{minipage}[t]{\linewidth}
				%\smallskip
				\begin{itemize}
					\item 加えた力の大きさを調べることができる
					\item 0〜1023の値が入力される。圧力が小さい時は値が大きく、圧力が大きい時は値が小さくなる。
				\end{itemize}
				\smallskip
			\end{minipage} & 
			\begin{minipage}[t]{\linewidth}
				\smallskip
				\centering
				\includegraphics[width=0.8\linewidth]{images/chap05/text05-img021.jpg}
				\smallskip
			\end{minipage} &
			\pageref{touch}\\ \cline{2-5}
			& ボリューム / \#104 & 
			\begin{minipage}[t]{\linewidth}
				%\smallskip
				\begin{itemize}
					\item \ruby{軸}{じく}をひねると電圧の大きさを変えることができる
					\item 0〜1023の値が入力される。右に回すほど値が大きくなる。
				\end{itemize}
				\smallskip
			\end{minipage} & 
			\begin{minipage}[t]{\linewidth}
				\smallskip
				\centering
				\includegraphics[width=0.8\linewidth]{images/chap05/text05-img022.jpg}
				\smallskip
			\end{minipage} &
			\pageref{volume}\\ \hline 
		\end{tabular}
	\end{widerrows}	
\end{table}

\begin{table}[H]
  \begin{widerrows}
		\begin{tabular}{|p{\colA}|p{\colB}|p{\colC}|p{\colD}|p{\colE}|}
			\hline
			入出力 &ブリック/ \par ブリック番号 & 機能 & 画像 & 紹介ページ\\ \hline
			アナログ入力装置 & 距離センサー / \#116(番号は書いてありません) &
			\begin{minipage}[t]{\linewidth}
				%\smallskip
				\begin{itemize}
				\item \ruby{距離}{きょ|り}を測ることができる
				\item 0〜1023の値が入力される。計算でcm(センチメートル)に\ruby{変換}{へん|かん}する。
				\end{itemize}
				\smallskip
			\end{minipage} & 
			\begin{minipage}[t]{\linewidth}
				\smallskip
				\centering
				\includegraphics[width=0.8\linewidth]{images/chap05/text05-img023.jpg}
				\smallskip
			\end{minipage} &
			\pageref{distance}\\ \hline

			デジタル入力装置 & 照度センサー / \#109 & 
			\begin{minipage}[t]{\linewidth}
			%\smallskip
				\begin{itemize}
					\item 明るさの変化を測ることができる
					\item 0〜1023の値が入力される。暗いほど値が大きくなる。
				\end{itemize}
				\smallskip
			\end{minipage} & 
			\begin{minipage}[t]{\linewidth}
				\smallskip
				\centering
				\includegraphics[width=0.8\linewidth]{images/chap05/text05-img024.png}
				\smallskip
			\end{minipage} &
			\pageref{light}\\ \hline

			その他 & \ruby{有機}{ゆう|き}ELディスプレイ / \#214 & 
			\begin{minipage}[t]{\linewidth}
				%\smallskip
				\begin{itemize}
					\item 文字を画面に\ruby{表示}{ひょう|じ}させることができる(アルファベットと記号のみ)
					\item oled “文字”で表示する
				\end{itemize}
				\smallskip
			\end{minipage} & 
			\begin{minipage}[t]{\linewidth}
				\smallskip
				\centering
				\includegraphics[width=0.8\linewidth]{images/chap05/text05-img025.png}
				\smallskip
			\end{minipage} &
			\pageref{oled}\\ \hline
		\end{tabular}
  \end{widerrows}	
\end{table}
