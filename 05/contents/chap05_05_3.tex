\subsection{デジタル入力装置}
\subsubsection{ボタン}\label{button}
\begin{table}[H]
	\begin{tabular}{|p{\colF}|p{\colG}|}	\hline
	名称 & ボタン(ぼたん)\\ \hline
	接続箇所 & デジタルコネクタ (3pin)\\ \hline
	機能概要 & ボタンが押されているか、押されていないかを調べる\\ \hline
  \end{tabular}
\end{table}

\begin{table}[H]
	\begin{tabular}{|p{\colF}|p{\colG}|}	\hline
	サンプルコードの場所 & 05/digin.hsp\\ \hline
	raspiへの入力 & ボタンが押されていると0、押されていないと1の値になる。\\ \hline
	raspiへの入力方法 & val = gpioin(GPIO番号)\\ \hline
	raspiからの出力 & なし\\ \hline
	raspiからの出力方法 & なし\\ \hline
  \end{tabular}
\end{table}

\begin{table}[H]
	\begin{tabular}{|p{\colF}|p{\colG}|} \hline
	使い道 & ゲームのコントローラーのボタン\\ \hline
	注意事項 & 強く押して壊さないように注意\\ \hline
	補足 & ボタンを押すことによって、中に入っている金属が繋がり電気が流れます。\\ \hline
  \end{tabular}
\end{table}

\begin{figure}[H]
	\begin{tabular}{|p{\colH}|p{\colI}|p{\colH}|p{\colI}|} \hline
	外観 & 
	\begin{minipage}[t]{\linewidth}
    \smallskip
      \centering
      \includegraphics[width=\linewidth]{images/chap05/text05-img028.png}
      \caption{ボタン}
      \smallskip
    \end{minipage} &
    回路記号 & 
    \begin{minipage}[t]{\linewidth}
    \smallskip
      \centering
      \includegraphics[width=\linewidth]{images/chap05/text05-img045.png}
      \caption{ボタンの回路図}
      \smallskip
    \end{minipage}\\ \hline
  \end{tabular}
\end{figure}
