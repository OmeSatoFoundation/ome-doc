\subsection{ピン番号、ケーブル}
FaBoのシールドにはブリックとRaspberry Piを繋ぐためのピンがあります。ピンにはGPIO、A、I2Cがあります。下の表のように使い方が分けられているので、間違えないように気をつけましょう。\\
\begin{figure}[H]
\begin{center}
    \includegraphics[scale=0.6]{images/chap05/text05-img012.png}
    \caption{Faboシールドのピンはいち}
\end{center}
\end{figure}
\begin{table}[H]
 \centering
 \begin{tabular}{|c|l|} \hline
  GPIO 12 & \multirow{8}{*}{ボタンやスイッチなどのデジタル用のピン} \\ \cline{1-1}
  GPIO 16 & \\ \cline{1-1} GPIO 19 & \\ \cline{1-1} GPIO 20 & \\ \cline{1-1} GPIO 23 & \\ \cline{1-1}
  GPIO 24 & \\ \cline{1-1} GPIO 25 & \\ \cline{1-1} GPIO 26 & \\ \hline
  A0 & \multirow{4}{*}{ボリュームや距離センサーなどのアナログ用のピン} \\ \cline{1-1}
  A1 & \\ \cline{1-1} A2 & \\ \cline{1-1} A3 & \\ \hline
  I2C & ゆうきELディスプレイのような特別な通信のためのピン \\ \hline
 \end{tabular}
\end{table}
今まで使ってきたセンサーボードではGPIOの17, 18, 22, 24がLEDなど、あらかじめ番号が決められていました。しかしFaBoのGPIOはどの番号につなげても使えます。ただし、プログラムも番号に合わせて書く必要があります。\\
ケーブルは3ピンと4ピンの2種類があります。今回4ピンはゆうきELディスプレイで使います。\\
\begin{figure}[h]
\begin{center}
    \includegraphics[scale=0.6]{images/chap05/text05-img013.png}
    \caption{4ピンケーブルと3ピンケーブル(全体)}
\end{center}
\end{figure}
\begin{figure}[H]
  \begin{minipage}[t]{0.48\columnwidth}
    \centering
 \includegraphics[width=0.8\hsize]{images/chap05/text05-img014.jpg}
    \caption{4ピンケーブル(コネクタ)}
  \end{minipage}
  \hspace{0.04\columnwidth} % ここで隙間作成
  \begin{minipage}[t]{0.48\columnwidth}
    \centering
    \includegraphics[width=0.8\hsize]{images/chap05/text05-img015.jpg}
    \caption{3ピンケーブル(コネクタ)}
  \end{minipage}
\end{figure}











