\newpage
\section{赤外線送受信のまとめ}
\begin{enumerate}
\item \code{sudo service lircd stop}
\item \code{mode2 -d /dev/lirc0| tee ファイル名1.txt}
\item \code{convert\_pattern ファイル名1.txt > ファイル名2.pattern}
\item \code{template.lircd.confを書きかえてファイル名3.lircd.confを作成}
\item \code{sudo cp ファイル名3.lircd.conf /etc/lirc/lircd.conf.d/}
\item \code{sudo service lircd restart}
\item \code{irsend SEND\_ONCE <リモコンの名前> <信号の名前>}
\end{enumerate}

\begin{tcolorbox}[title=\useOmetoi]
\begin{enumerate}
\addex{これは宿題です。自分の家にある家電をRaspberry Piで制御してみましょう。エアコン、\ruby{扇風機}{せん|ぷう|き}、テレビなど赤外線を使って\ruby{操作}{そう|さ}する家電を選びましょう。}
\end{enumerate}
\end{tcolorbox}



\subsection{おまけ:信号を2つ以上登録する}
\ruby{複数}{ふく|すう}の動作を\ruby{制御}{せい|ぎょ}したい場合は、動作ごとに信号を登録します。template.lircd.confの中の“begin raw\_codes” と “end raw\_codes”の間にname <名前>と信号を\ruby{並}{なら}べて複数の信号を登録できます。ただし、ひとつのファイルの中にあるすべての信号の数の合計は255\ruby{個}{こ}以下でなければなりません。このファイルの場合、onoffに\ruby{記述}{き|じゅつ}される数字の数と、powerに記述される数字の数の合計が255以下でなければなりません。\\

\begin{lstlisting}[caption=2つの信号を登録するときのtemplate.lircd.comf,label=2つの信号を登録するときのtemplate.lircd.comf]
begin remote
  name fan
  flags RAW_CODES
  eps 30
  aeps 100

  gap 200000
  toggle_bit_mask 0x0

  begin raw_codes

      name onoff
      1207 588 447 838 447 823 475 812 1218…

      <#red#name power
      1364 363 752 743 372 836 164 614 ...#>

  end raw_codes

end remote
\end{lstlisting}

