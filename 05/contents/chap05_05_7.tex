\subsection{アナログ入力そうち}\label{analog_in}
\subsubsection{感圧センサー}\label{touch}
\begin{table}[H]
  \begin{widerrows}
    \begin{tabular}{|p{\colF}|p{\colG}|}	\hline
    名称 & 感圧センサー(かんあつせんさー)\\ \hline
    接続箇所 & アナログコネクタ (3pin)\\ \hline
    機能概要 & 圧力を測定する\\ \hline
    \end{tabular}
  \end{widerrows} 
\end{table}

\begin{table}[H]
  \begin{widerrows}
    \begin{tabular}{|p{\colF}|p{\colG}|}	\hline
    サンプルコードの場所 & 05/anain.hsp\\ \hline
    raspiへの入力 & 0~1023の値で圧力を測定します。圧力がかかるほど値が小さくなります。\\ \hline
    raspiへの入力方法 & val = spiget(ピン番号, チャンネル番号)\\ \hline
    raspiからの出力 & なし\\ \hline
    raspiからの出力方法 & なし\\ \hline
    \end{tabular}
  \end{widerrows} 
\end{table}

\begin{table}[H]
  \begin{widerrows}
    \begin{tabular}{|p{\colF}|p{\colG}|} \hline
    使い道 & 圧力を調べる、重さをはかる\\ \hline
    注意事項 & 圧力を感知する部分が壊れやすいです。\\ \hline
    補足 & なし\\ \hline
    \end{tabular}
  \end{widerrows} 
\end{table}

\begin{figure}[H]
  \begin{widerrows}
    \begin{tabular}{|p{\colH}|p{\colI}|p{\colH}|p{\colI}|} \hline
    外観 & 
    \begin{minipage}[t]{\linewidth}
      \smallskip
        \centering
        \includegraphics[width=0.5\linewidth]{images/chap05/text05-img021.jpg}
        \caption{感圧センサー}
        \smallskip
      \end{minipage} &
      回路記号 & 
      \begin{minipage}[t]{\linewidth}
      \smallskip
        \centering
        \includegraphics[width=0.5\linewidth]{images/chap05/text05-img049.png}
        \caption{感圧センサーの回路図}
        \smallskip
      \end{minipage}\\ \hline
    \end{tabular}
  \end{widerrows} 
\end{figure}