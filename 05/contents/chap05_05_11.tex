\subsection{その他}
\subsubsection{有機ELディスプレイ}\label{oled}
\begin{table}[H]
	{\renewcommand\arraystretch{1.4}
		\begin{tabular}{|p{\colF}|p{\colG}|}	\hline
		名称 & 有機ELディスプレイ(ゆうきいーえるでぃすぷれい)\\ \hline
		接続箇所 & I2C (4pin)\\ \hline
		機能概要 & 文字の表示\\ \hline
		\end{tabular}
	}
\end{table}

\begin{table}[H]
	{\renewcommand\arraystretch{1.4}
		\begin{tabular}{|p{\colF}|p{\colG}|}	\hline
		サンプルコードの場所 & 05/oled.hsp\\ \hline
		raspiへの入力 & なし\\ \hline
		raspiへの入力方法 & なし\\ \hline
		raspiからの出力 & プログラムで決めた文字や記号を表示する\\ \hline
		raspiからの出力方法 & oled “表示させたい文字”\\ \hline
		\end{tabular}
	}
\end{table}

\begin{table}[H]
	{\renewcommand\arraystretch{1.4}
		\begin{tabular}{|p{\colF}|p{\colG}|} \hline
		使い道 & テレビ\\ \hline
		注意事項 & 日本語表示させることはできないので注意\\ \hline
		補足 & なし\\ \hline
		\end{tabular}
	}
\end{table}

\begin{figure}[H]
	{\renewcommand\arraystretch{1.4}
		\begin{tabular}{|p{\colH}|p{\colI}|p{\colH}|p{\colI}|} \hline
		外観 & 
		\begin{minipage}[t]{\linewidth}
			\smallskip
				\centering
				\includegraphics[width=0.5\linewidth]{images/chap05/text05-img032.png}
				\caption{有機ELディスプレイ}
				\smallskip
			\end{minipage} &
			回路記号 & なし\\ \hline
		\end{tabular}
	}
\end{figure}
