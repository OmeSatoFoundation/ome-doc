\section{赤外線で信号を送る}
\subsection{赤外線ってなんだろう?}

\begin{figure}[H]
\centering
\includesvg[width=0.8\linewidth]{images/chap05/electromagnetic_wave.svg}
    \caption{\ruby{電磁}{でん|じ}スペクトル図}

\end{figure}

\ruby{物質}{ぶっ|しつ}や生物は電磁波を発しています。X線、\ruby{紫外線}{し|がい|せん}、光、マイクロ波などは電磁波の一種です。電磁波は波であり、波の長さで種類をわけることができます。わたしたちは\ruby{可視光線}{か|し|こう|せん}と\ruby{呼}{よ}ばれる光が目に入ることで、色などを目で感知することができます。マイクロ波は電子レンジに使われています。マイクロ波によって物質を\ruby{振動}{しん|どう}させることで、物質の温度をあげています。\\

赤外線は人の目には見えません。けれどすべての物体が赤外線を発しています。特に温度の高いものは赤外線を強く発しています。Raspberry Piは発熱しながら動いています。Raspberry Piに\ruby{触}{ふ}れなくても、手を近づけるだけで熱を感じるのは、赤外線がRaspberry Piから発せられていて、その赤外線を手で感知しているからです。\\

赤外線を使って信号を送ったり、受け取ったりすることもできます。例えば赤外線が送られているときは電気をつける、赤外線が送られていないときは電気を消す、などとすることができます。しかし赤外線が送られているかどうかを判定するだけではつける、消すの2つの動作しかできません。ロボットのように歩く、止まる、右に曲がる、左に曲がるなど多くの動きを\ruby{制御}{せい|ぎょ}したいときは、赤外線のONとOFFの組み合わせで信号作って送ります。

\begin{figure}[H]
\centering
\includesvg[scale=0.6]{images/chap05/text05-img037.svg}
\caption{赤外線で信号を送る方法}
\label{ir}
\end{figure}

図\ref{ir}を見てみましょう。電球が点いているときは赤外線を送り、電球が消えていないときは赤外線を送っていないという意味です。例えば1/1000000秒の\ruby{間隔}{かん|かく}で赤外線を送ります。つけたり消したりを組み合わせて、いろいろなパターンを作ることができます。そのパターンによって動作を決めています。送る、送らない、送らない、送らない、送る、送る、送らないと信号が来たときは”歩く”動作をします。
