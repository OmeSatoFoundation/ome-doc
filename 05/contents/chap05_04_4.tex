\newpage
\section{HSPを使って赤外線を出力するプログラムを作ってみよう}
HSPスクリプトエディタで\textasciitilde /05/ir.hspを開いてみましょう。\\
HSPではexec命令を使ってターミナルで使うコマンドを実行することができます。\\
\code{exec "irsend SEND\_ONCE fan onoff"}\\
exec命令は、\code{exec “コマンド”}とターミナルのコマンドを指定します。ここでのコマンドは先ほどターミナルで実行した\code{irsend SEND\_ONCE \textcolor{red}{fan onoff}}です。\code{\textcolor{red}{fan}}はリモコンの名前、\code{\textcolor{red}{onoff}}は信号の名前でした。\code{\textcolor{red}{fan}}と\code{\textcolor{red}{onoff}}は自分で\ruby{設定}{せっ|てい}ファイルに書いた名前に変えましょう。\\

\begin{lstlisting}[caption=ir.hsp,label=ir.hsp]
#include "hsp3dish.as"
#include "rpz-gpio.as"

BUTTON_PIN = 24
prev = 0
status=0

*main
        gpmes "ボタンPUSHで赤外線照射!"
        goto *edge

*edge
        current = gpioin(BUTTON_PIN)
        if (prev=0) & (current=1) {		<#blue#;ボタンが押されたら#>
            exec "irsend SEND_ONCE fan onoff"	<#blue#;irsendコマンドを実行する#>
        }
        prev = current
        update 1
        goto *edge
\end{lstlisting}

\begin{tcolorbox}[title=\useOmetoi]
\begin{enumerate}
\addex{ボタンをGPIO24につなげ、自分でコピーした赤外線を出力してみましょう。\code{\textcolor{red}{fan}}と\code{\textcolor{red}{onoff}}は自分で設定ファイルに書いた名前に変えましょう。}
\end{enumerate}
\end{tcolorbox}