\documentclass[a4paper,12pt]{jarticle}
\input ./chap01_preamble.tex
\graphicspath{%
  {../text01-img/}%
}
% !TEX root = ./chap01_08.tex
\begin{document}
\section{\ruby{質問}{しつもん}フォーム}
わからないこと、\ruby{質問}{しつもん}したいことがあれば\ruby{質問}{しつもん}フォームから\ruby{質問}{しつもん}しよう!

\ruby{質問}{しつもん}フォームでは、\ruby{授業}{じゅぎょう}でわからなかったことや\ruby{確認}{かくにん}したいことなどをホームページを通して\ruby{質問}{しつもん}できます。
たくさん\ruby{質問}{しつもん}してわからないことを\ruby{解決}{かいけつ}しよう。

1.
ブラウザで\textbf{\ruby{授業}{じゅぎょう}で使うホームページリスト}を開こう。

\textbf{\ruby{授業}{じゅぎょう}で使うホームページリスト}は01フォルダの中のlinks.htmlです

リストの4番目の子ども\textbf{IT\ruby{未来塾}{みらいじゅく}\ruby{質問}{しつもん}フォーム}をクリックして開いてください(Figure~\ref{seq:refFigure46})。


\bigskip

{\bfseries
  スマートフォン等からの場合は\url{https://bit.ly/2NHiVgi}を開いてください。}



\centering
\begin{minipage}{9.781cm}
  {\upshape
    \includegraphics[width=11.231cm,height=7.613cm]{textbook-img245.png}
    \flushleft

    \bigskip
    Figure {\refstepcounter{Figure}\theFigure\label{seq:refFigure46}}:
    ホームページリストより\ruby{質問}{しつもん}フォームを開く}
\end{minipage}

\bigskip

\bigskip

2.
\ruby{質問}{しつもん}フォーム(Figure~\ref{seq:refFigure47})が\ruby{表示}{ひょうじ}されます

\centering
\begin{minipage}{11.915cm}
  {\upshape
    \includegraphics[width=13.263cm,height=14.233cm]{textbook-img246.png}
    \flushleft

    \bigskip
    Figure {\refstepcounter{Figure}\theFigure\label{seq:refFigure47}}: \ruby{質問}{しつもん}フォーム}
\end{minipage}
\flushleft

\bigskip

メールアドレス、お名前、グループの先生の名前、ぎもんに思っていることを入力します。
必要に\ruby{応}{おう}じて、ぎもんに思っている場所の\ruby{画像}{がぞう}を追加してください。

{\bfseries
送信\textmd{を\ruby{押}{お}すと\ruby{質問}{しつもん}ができます。Figure~\ref{seq:refFigure48}の画面が出たら\ruby{質問}{しつもん}は\ruby{完了}{かんりょう}です。}}



\centering
\begin{minipage}{8.871cm}
  {\upshape
    \includegraphics[width=8.871cm,height=4.374cm]{textbook-img247.png}
    \flushleft

    \bigskip
    Figure {\refstepcounter{Figure}\theFigure\label{seq:refFigure48}}: \ruby{質問}{しつもん}送信画面}
\end{minipage}
\flushleft
メールアドレスのらんに入力したアドレスへグループの先生から回答メールがきます。
メールが来るまで、しばらくお待ちください。数日かかる場合があります。
\end{document}