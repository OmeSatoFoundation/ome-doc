% \documentclass[dvipdfmx]{beamer}

% \usepackage{beamerthemesplit}
% \usepackage[]{graphicx}
\graphicspath{%
{./slide08-img/}%
{./text08-img/}%
}

% \usepackage{listings}
% \usepackage{hyperref}
% \usepackage{pxjahyper}
% \usepackage{nameref} % これが\zexternaldocumentの前までに必要
% \usepackage{zref-xr}
% \usepackage{color}

\zxrsetup{toltxlabel} % 通常のLaTeXスタイルの\refを使う(\zexternaldocumentより前におく)
\zexternaldocument*[1:]{text08} % \zのついたexternaldocumentを使う

\setbeamertemplate{footline}[frame number]
\title{子どもIT未来塾 第8回}
\author{塾長 清水尚彦}

\def\quiz{1}


\frame{
   \begin{center}
    \huge{子どもIT未来塾}\\

    \vspace{48pt}
	   \Large{第8回}\\
	   {\huge\bf スクレイピングを学ぼう}\\
    \vspace{24pt}
    \large{塾長 清水尚彦}\\
    \vspace{10pt}
    \large{\the\year 年 9月24日}
  \end{center}
}
\begin{frame}[fragile]
	\frametitle{\large{発表会に向けて、自分が作りたいプログラムを考えよう}~~~\raisebox{-3mm}{\includegraphics[width=0.1\textwidth]{raspberry}}}
    \begin{itemize}
        \item 今まで習ったセンサーや Fabo、OpenJtalk、Julius、ゲーム、スクレイピング、CGIなど組み合わせられるか考えてみましょう。
        \item 自分の作りたいプログラムのアイデアをP.\pageref{1:P:challenge}のアイデアノートなどにメモしておきましょう。        
    \end{itemize}
\end{frame}

