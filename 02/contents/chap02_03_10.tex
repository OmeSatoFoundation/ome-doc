\clearpage

\subsection{例題8 文字の\ruby{表示位置}{ひょう|じ|い|ち}を変えてみる}

\subsubsection*{考え方}

mes命令で表示する文字の表示位置を変えるにはpos命令を使います。

\begin{description}
    \item (例)
    \item pos 160,200
\end{description}

パラメーターの数字2つを、増やしたり減らしたりしながら、文字の位置がどのように変わるか\ruby{試}{ため}してみましょう。

\subsection*{例題8 答え}

必ず、mes命令が実行されるよりも前(上の行)にpos命令を書きます。

\begin{description}
    \item (例)
    \item pos 160,200
\end{description}

「160」の部分が左からの位置。
「200」の部分が上からの位置になります。
位置を指定するためには、たてよこの2つ数字が必要です。
「\ruby{横}{よこ}の位置」「\ruby{縦}{たて}の位置」は、左上からの点(ドット)の数なので覚えておきましょう。

\begin{description}
    \item (HSPのルール)
\end{description}

\begin{description}
    \item 文字を出す位置を変えるにはpos命令を使います
    \item posの後はスペースに続けて2つの数字を指定します
    \item 2つの数字は「,」で区切ること
    \item 2つの数字は「\ruby{横}{よこ}の位置」「\ruby{縦}{たて}の位置」を指定する
\end{description}

実際に試してみて、文字の表示位置が変わったらTAや周りの友達にも見せてあげましょう。