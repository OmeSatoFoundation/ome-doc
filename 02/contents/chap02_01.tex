\chapter{ゲームを\ruby{改造}{かい|ぞう}してみよう}
\section{今回の\ruby{授業}{じゅ|ぎょう}}
\subsection{\ruby{目標}{もく|ひょう}}
\begin{itemize}
  \item プログラムについて知ろう
  \item センサーとゲームで遊んでみよう
\end{itemize}

\subsection{\ruby{授業内容}{じゅ|ぎょう|ない|よう}}
\begin{enumerate}
  \item プログラミングのじゅんびをしよう
  \item ゲームで遊んでみよう
  \item センサーボードを使ってみよう
  \item プログラムを\ruby{改造}{かい|ぞう}してみよう
\end{enumerate}

\subsection{\ruby{注意点}{ちゅう|い|てん}}
\begin{itemize}
  \item 授業の\ruby{合間}{あい|ま}のきゅうけいでは、遠くのものをながめたりして目を休めましょう
  \item \ruby{水分}{すい|ぶん}ほきゅうはこまめにしましょう
  \item 先生が\ruby{説明}{せつ|めい}しているときは先生の話を聞きましょう
  \item わからないことがあったらTAの先生方にすぐ聞きましょう
\end{itemize}

\subsection{ラズベリーパイを使うときの\ruby{注意}{ちゅう|い}}
\begin{itemize}
  \item 水などぬれているものをラズベリーパイ\ruby{本体}{ほん|たい}につけないようにしましょう
  \item ラズベリーパイをはじめコンピュータは熱に弱いのですごく\ruby{暑}{あつ}い\ruby{部屋}{へ|や}では使わないようにしましょう
  \item ラズベリーパイなどは\ruby{静電気}{せい|でん|き}によわいので注意しましょう
  \item ラズベリーパイをらんぼうに\ruby{扱}{あつか}うのはやめましょう
\end{itemize}
\clearpage

