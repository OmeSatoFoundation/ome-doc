\clearpage

\subsection{例題7 文字のサイズを変えてみる}

\subsection*{考え方}

mes命令で表示する文字のサイズを変えるにはfont命令を使います。

\begin{description}
    \item (例)
    \item font “”,100
\end{description}

font命令の後にスペースに続けて「””」、さらに「,」と大きさが書かれています。
たとえば、この「100」をもっと大きな数字にしてみて、文字が大きくなるのを確認しましょう。

\subsubsection*{例題7 答え}

必ず、mes命令が実行されるよりも前(上の行)にfont命令を書きます。
font命令で指定するパラメーターのは文字のサイズで大きい数になるほど文字も大きくなります。
\ruby{逆}{ぎゃく}に小さな数にすることで、小さな文字になります。

\begin{description}
    \item (HSPのルール)
\end{description}

\begin{description}
    \item font命令はフォントの形と大きさを決めるための命令
    \item 「“”,70」、の70はそのサイズ
\end{description}

実際に試してみて、文字のサイズが変わったらTAや周りの友達にも見せてあげましょう。

