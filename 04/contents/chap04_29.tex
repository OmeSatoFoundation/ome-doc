%2844
\newpage
\subsection{例題4-15 センサーボードを使った応用例を考えてみよう}


\begin{description}
    \item \textgt{\bf 考え方}
\end{description}

今まで覚えてきたことを組み合わせて、何ができるか考えてみましょう。家に帰ってから、考えてみて答えを書いてきてください。センサーボードで何ができるか、プログラムでどんなことができるか、もう一度思い出してみて、考えてみましょう。

考えがまとまらない人は、いままでに動かしたプログラムを改造してみたいこと、センサーボードでやってみたいことを書いてみましょう。実際にプログラムを作る必要はありません。いっぱい考えついた人は、例題4-16と例題4-17の答えの欄に書いてみましょう。

大切なことは、自分で何でも試してみることです。わからないことがあっても、実際に実行すれば結果がわかります。立ち止まらずに、どうなるか想像しながらプログラムを書いて実行することが、早く上達するコツになります。それでもわからない時は、先生に聞いてみましょう。

\begin{description}
    \item \textgt{\bf 例題4-15 答え}
\end{description}

自分で考えた応用例を下に書いておきましょう。

応用例ができたら、今まで覚えてきた命令でできることか考えてみましょう。

実際に作れる人は、自分でプログラムを作ってみてください。

%2879



