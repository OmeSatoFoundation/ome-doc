%8/31
%1158
\newpage
\subsection{例題4-5 自分で用意した絵を表示する}

\begin{description}
    \item \textgt{\bf \ \ 考え方}
\end{description}


絵と文字を表示するプログラム(celput.hsp)を改造して自分で用意した絵を出してみましょう。

準備する画像のファイルは、プログラム(.hsp)がある場所と同じフォルダに入れておく必要があります。

\ \ celput.hsp

\ \ sozai1.jpg

が同じディレクトリにあることを確認しましょう。

同じように「04」フォルダの中に、皆さんが自分で画像を作ったり、コピーして使用することもできます。

\begin{description}
    \item \textgt{\bf \ \ 例題4-5 答え}
\end{description}


[F5]キーを押して正しい画像が表示されるかどうか確認しましょう。

改造ができたらTAや周りの友達にも見せてあげましょう。

プログラムをあまり改造しすぎると、動かなくなったり、パソコンの動きが遅くなってしまうことがあります。その時は、先生に聞くか、元のファイル(celput.hsp)をもう一度コピーして使ってみてください。

%1218







