%2491

条件判断の方法をもう一度思い出してください。

if命令の使い方、ルールを覚えていますか?

\begin{description}
    \item \textgt{\bf \ \ (HSPのルール)}
    \item \textgt{\bf \ \ if命令により条件を判断することができる}
    \item \textgt{\bf \ \ ifの後にスペースに続けて\ruby{条件式}{じょう|けん|しき}を指定します}
    \item \textgt{\bf \ \ その後で「:」に続けて条件が正しい時に実行される命令を書きます}
\end{description}

\ \ (条件式はいくつか書き方があります)


\ \ \ \ 条件式 \ \ \ \ \ \ \ \ \ 意味

\ \ \ \ {}-{}-{}-{}-{}-{}-{}-{}-{}-{}-{}-{}-{}-{}-{}-{}-{}-{}-{}-{}-{}-{}-{}-{}-{}-{}-{}-{}-{}-{}-{}-{}-{}-{}-{}-{}-{}-{}-{}-{}-{}-{}-{}-{}-{}-{}-{}-{}-{}-{}-{}-{}-

\ \ \ \ 変数名 =
数値\ \ 変数の内容と数値が同じである

\ \ \ \ 変数名 !
数値\ \ 変数の内容と数値が同じではない

\ \ \ \ 変数名 {\textless}
数値\ \ 変数の内容より数値の方が大きい

\ \ \ \ 変数名 {\textgreater}
数値\ \ 変数の内容より数値の方が小さい

下のプログラムがどのような意味か考えてみましょう。


\begin{description}
    \item \textgt{\bf \ \ getkey a,’A’}
    \item \textgt{\bf \ \ if a=1 : x=x-4}
\end{description}

最初は、条件判断の仕組みがわかりにくいかもしれませんが、プログラムの1行目から、コンピューターが実行することを想像しながら、1つ1つ確認することが大切です。

やがて、プログラムを見て、どのように動くのか想像ができるようになります。

最初はちょっと難しく感じるかもしれませんが、慣れれば誰でも理解できるようになります。

あせらず、わからない所はまわりの先生や友達に聞きながら、進んでいきましょう。

\subsection{例題に挑戦しよう}

ゲームが遊べた人は、以下の例題にも挑戦してみよう。

・センサーボードのボタンを使ってみよう

・照度センサーの値をもとに絵を動かそう

・照度センサーを使ったゲームを遊んでみよう

・センサーボードを使った応用例を考えてみよう


例題の考え方がわからない時は、近くのTAか先生に聞いてください。

わからない所は、そのままにせず、必ず答えを見つけてから先に進みましょう。

%2586