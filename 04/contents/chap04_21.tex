%2083
\newpage
\section{プログラミングの応用}

\subsection{サンプルフォルダを準備しよう}

上にあるバーのアイコンからファイルマネージャーを開いてみましょう。

\begin{figure}[H]
    \begin{center}
      \includegraphics[keepaspectratio,width=5.898cm,height=1.242cm]{text04-img/s_filemanager.png}
      \caption{ファイルマネージャーを開くアイコン}
    \end{center}
    \label{fig:prog_menu}
\end{figure}

これからHSPで使うことのできるサンプルプログラムをコピーしてもらいます。「/home/ユーザー名」のフォルダに、「sample」フォルダをコピーすることから始めます。

まず、「/usr/local/share/OpenHSP」という場所を開いてみましょう。

この中に、「sample」という名前のフォルダがあります。これから、この「sample」というフォルダを自分の作業フォルダにコピーします。

「sample」という名前のフォルダ上でマウスの右クリックを押して「コピー」を選んでください。


\begin{figure}[H]
    \begin{center}
      \includegraphics[keepaspectratio,width=11.232cm,height=8.424cm]{text04-img/s_ome04e.png}
      \caption{sampleフォルダでコピーを選ぶところ}
    \end{center}
    \label{fig:prog_menu}
\end{figure}

この後、コピーしたい場所を選びます。コピー先の場所は、「/home/ユーザー名」のフォルダになります。

コピー先の場所を開いたら、何もない場所でマウスの右クリックを押して「貼り付け」を選んでください。

これで先ほどの「/usr/local/share/OpenHSP/sample」フォルダが、「/home/ユーザー名/sample」にコピーされます。

ファイルには、さまざまなプログラムとデータが入っています。

すべてのファイルを\ruby{紹介}{しょう|かい}できませんが、\ruby{興味}{きょう|み}がある人は時間がある時に読み込んで動かしてみましょう。

%2146


