% \title{\Huge\bf 子どもIT未来塾 第2回}
% \author{
% \huge\bf ゲームを改造してみよう!
% \vspace{15mm}
% }
% \date{ \Huge おにたま(武田 寧) 先生 }

% \chapter{子どもIT未来塾 第2回}
% \section{ゲームを改造してみよう!}
% \subsection{おにたま(武田 寧) 先生}

\begin{center}
    \par
    \par
    \par
    \par
    \vspace{30mm}
    {\Huge      }
    {\Huge{\bf 子どもIT未来塾 第4回}}
    \par
    \par
    \par
    \par
    \vspace{15mm}
    {\huge{\bf プログラムで、いろいろ自動化してみよう}}
    \par
    \par
    \par
    \par
    \vspace{15mm}
    {\Huge おにたま(武田 寧) 先生 }
\end{center}
\clearpage


\section{今回の授業}
\subsection{目標}
\ \ \ \ プログラムについて知ろう

\ \ \ \ センサーとゲームで遊んでみよう

\subsection{授業内容}
%\liststyleLii
\begin{enumerate}
\item プログラミングの準備をしよう
\item ゲームで遊んでみよう
\item センサーボードを使ってみよう
\item プログラムを改造してみよう
\end{enumerate}
\subsection{注意点}
%\liststyleLiii
\begin{itemize}
\item
授業の合間のきゅうけいでは、遠くのものをながめたりして目を休めましょう
\item 水分ほきゅうはこまめにしましょう
\item
先生が説明中は先生の話を聞きましょう
\item
わからないことがあったらTAの先生方にすぐ聞きましょう
\end{itemize}

\subsection{ラズベリーパイを使うときの注意}
%\liststyleLvii
\begin{itemize}
\item
水などぬれているものをラズベリーパイ本体につけないようにしましょう
\end{itemize}
%\liststyleLviii
\begin{itemize}
\item
ラズベリーパイをはじめコンピュータなどは熱に弱いのですごく暑い部屋では使わないようにしましょう
\end{itemize}
%\liststyleLix
\begin{itemize}
\item
ラズベリーパイなどは静電気によわいので注意しましょう
\end{itemize}
%\liststyleLx
\begin{itemize}
\item
ラズベリーパイをらんぼうに扱うのはやめましょう
\end{itemize}
%\liststyleLxi


  % Main Part:


\clearpage