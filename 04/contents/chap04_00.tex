\chapter{プログラムでいろいろ\ruby{自動化}{じ|どう|か}してみよう}
\section{今回の授業}
\subsection{\ruby{目標}{もく|ひょう}}
\ruby{対戦}{たい|せん}できるゲームで遊んでみよう

センサーのプログラムを作ろう

\subsection{\ruby{授業内容}{じゅ|ぎょう|ない|よう}}
%\liststyleLii
\begin{enumerate}
    \item プログラミングの\ruby{復習}{ふく|しゅう}をしよう
    \item \ruby{対戦}{たい|せん}ゲームで遊んでみよう
    \item センサーボードをプログラムで使おう
    \item ゲームの動きを\ruby{改造}{かい|ぞう}してみよう
\end{enumerate}

\subsection{注意点}
%\liststyleLiii
\begin{itemize}
    \item \ruby{授業}{じゅ|ぎょう}の\ruby{合間}{あい|ま}のきゅうけいでは、遠くのものをながめたりして目を休めましょう
    \item \ruby{水分}{すい|ぶん}ほきゅうはこまめにしましょう
    \item 先生が\ruby{説明}{せつ|めい}しているときは先生の話を聞きましょう
    \item わからないことがあったらTAの先生方にすぐ聞きましょう
\end{itemize}

\subsection{ラズベリーパイを使うときの\ruby{注意}{ちゅう|い}}
\begin{itemize}
  \item 水などぬれているものをラズベリーパイ\ruby{本体}{ほん|たい}につけないようにしましょう
  \item ラズベリーパイをはじめコンピュータは熱に弱いのですごく\ruby{暑}{あつ}い\ruby{部屋}{へ|や}では使わないようにしましょう
  \item ラズベリーパイなどは\ruby{静電気}{せい|でん|き}によわいので注意しましょう
  \item ラズベリーパイをらんぼうに\ruby{扱}{あつか}うのはやめましょう
\end{itemize}
\clearpage

